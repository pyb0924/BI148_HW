\documentclass[11pt]{article}
\usepackage{fontspec, xunicode, xltxtra}
    \setmainfont{Microsoft YaHei}
    \usepackage[breakable]{tcolorbox}
    \usepackage{parskip} % Stop auto-indenting (to mimic markdown behaviour)
    
    \usepackage{iftex}
    \ifPDFTeX
    	\usepackage[T1]{fontenc}
    	\usepackage{mathpazo}
    \else
    	\usepackage{fontspec}
    \fi

    % Basic figure setup, for now with no caption control since it's done
    % automatically by Pandoc (which extracts ![](path) syntax from Markdown).
    \usepackage{graphicx}
    % Maintain compatibility with old templates. Remove in nbconvert 6.0
    \let\Oldincludegraphics\includegraphics
    % Ensure that by default, figures have no caption (until we provide a
    % proper Figure object with a Caption API and a way to capture that
    % in the conversion process - todo).
    \usepackage{caption}
    \DeclareCaptionFormat{nocaption}{}
    \captionsetup{format=nocaption,aboveskip=0pt,belowskip=0pt}

    \usepackage[Export]{adjustbox} % Used to constrain images to a maximum size
    \adjustboxset{max size={0.9\linewidth}{0.9\paperheight}}
    \usepackage{float}
    \floatplacement{figure}{H} % forces figures to be placed at the correct location
    \usepackage{xcolor} % Allow colors to be defined
    \usepackage{enumerate} % Needed for markdown enumerations to work
    \usepackage{geometry} % Used to adjust the document margins
    \usepackage{amsmath} % Equations
    \usepackage{amssymb} % Equations
    \usepackage{textcomp} % defines textquotesingle
    % Hack from http://tex.stackexchange.com/a/47451/13684:
    \AtBeginDocument{%
        \def\PYZsq{\textquotesingle}% Upright quotes in Pygmentized code
    }
    \usepackage{upquote} % Upright quotes for verbatim code
    \usepackage{eurosym} % defines \euro
    \usepackage[mathletters]{ucs} % Extended unicode (utf-8) support
    \usepackage{fancyvrb} % verbatim replacement that allows latex
    \usepackage{grffile} % extends the file name processing of package graphics 
                         % to support a larger range
    \makeatletter % fix for grffile with XeLaTeX
    \def\Gread@@xetex#1{%
      \IfFileExists{"\Gin@base".bb}%
      {\Gread@eps{\Gin@base.bb}}%
      {\Gread@@xetex@aux#1}%
    }
    \makeatother

    % The hyperref package gives us a pdf with properly built
    % internal navigation ('pdf bookmarks' for the table of contents,
    % internal cross-reference links, web links for URLs, etc.)
    \usepackage{hyperref}
    % The default LaTeX title has an obnoxious amount of whitespace. By default,
    % titling removes some of it. It also provides customization options.
    \usepackage{titling}
    \usepackage{longtable} % longtable support required by pandoc >1.10
    \usepackage{booktabs}  % table support for pandoc > 1.12.2
    \usepackage[inline]{enumitem} % IRkernel/repr support (it uses the enumerate* environment)
    \usepackage[normalem]{ulem} % ulem is needed to support strikethroughs (\sout)
                                % normalem makes italics be italics, not underlines
    \usepackage{mathrsfs}
    

    
    % Colors for the hyperref package
    \definecolor{urlcolor}{rgb}{0,.145,.698}
    \definecolor{linkcolor}{rgb}{.71,0.21,0.01}
    \definecolor{citecolor}{rgb}{.12,.54,.11}

    % ANSI colors
    \definecolor{ansi-black}{HTML}{3E424D}
    \definecolor{ansi-black-intense}{HTML}{282C36}
    \definecolor{ansi-red}{HTML}{E75C58}
    \definecolor{ansi-red-intense}{HTML}{B22B31}
    \definecolor{ansi-green}{HTML}{00A250}
    \definecolor{ansi-green-intense}{HTML}{007427}
    \definecolor{ansi-yellow}{HTML}{DDB62B}
    \definecolor{ansi-yellow-intense}{HTML}{B27D12}
    \definecolor{ansi-blue}{HTML}{208FFB}
    \definecolor{ansi-blue-intense}{HTML}{0065CA}
    \definecolor{ansi-magenta}{HTML}{D160C4}
    \definecolor{ansi-magenta-intense}{HTML}{A03196}
    \definecolor{ansi-cyan}{HTML}{60C6C8}
    \definecolor{ansi-cyan-intense}{HTML}{258F8F}
    \definecolor{ansi-white}{HTML}{C5C1B4}
    \definecolor{ansi-white-intense}{HTML}{A1A6B2}
    \definecolor{ansi-default-inverse-fg}{HTML}{FFFFFF}
    \definecolor{ansi-default-inverse-bg}{HTML}{000000}

    % commands and environments needed by pandoc snippets
    % extracted from the output of `pandoc -s`
    \providecommand{\tightlist}{%
      \setlength{\itemsep}{0pt}\setlength{\parskip}{0pt}}
    \DefineVerbatimEnvironment{Highlighting}{Verbatim}{commandchars=\\\{\}}
    % Add ',fontsize=\small' for more characters per line
    \newenvironment{Shaded}{}{}
    \newcommand{\KeywordTok}[1]{\textcolor[rgb]{0.00,0.44,0.13}{\textbf{{#1}}}}
    \newcommand{\DataTypeTok}[1]{\textcolor[rgb]{0.56,0.13,0.00}{{#1}}}
    \newcommand{\DecValTok}[1]{\textcolor[rgb]{0.25,0.63,0.44}{{#1}}}
    \newcommand{\BaseNTok}[1]{\textcolor[rgb]{0.25,0.63,0.44}{{#1}}}
    \newcommand{\FloatTok}[1]{\textcolor[rgb]{0.25,0.63,0.44}{{#1}}}
    \newcommand{\CharTok}[1]{\textcolor[rgb]{0.25,0.44,0.63}{{#1}}}
    \newcommand{\StringTok}[1]{\textcolor[rgb]{0.25,0.44,0.63}{{#1}}}
    \newcommand{\CommentTok}[1]{\textcolor[rgb]{0.38,0.63,0.69}{\textit{{#1}}}}
    \newcommand{\OtherTok}[1]{\textcolor[rgb]{0.00,0.44,0.13}{{#1}}}
    \newcommand{\AlertTok}[1]{\textcolor[rgb]{1.00,0.00,0.00}{\textbf{{#1}}}}
    \newcommand{\FunctionTok}[1]{\textcolor[rgb]{0.02,0.16,0.49}{{#1}}}
    \newcommand{\RegionMarkerTok}[1]{{#1}}
    \newcommand{\ErrorTok}[1]{\textcolor[rgb]{1.00,0.00,0.00}{\textbf{{#1}}}}
    \newcommand{\NormalTok}[1]{{#1}}
    
    % Additional commands for more recent versions of Pandoc
    \newcommand{\ConstantTok}[1]{\textcolor[rgb]{0.53,0.00,0.00}{{#1}}}
    \newcommand{\SpecialCharTok}[1]{\textcolor[rgb]{0.25,0.44,0.63}{{#1}}}
    \newcommand{\VerbatimStringTok}[1]{\textcolor[rgb]{0.25,0.44,0.63}{{#1}}}
    \newcommand{\SpecialStringTok}[1]{\textcolor[rgb]{0.73,0.40,0.53}{{#1}}}
    \newcommand{\ImportTok}[1]{{#1}}
    \newcommand{\DocumentationTok}[1]{\textcolor[rgb]{0.73,0.13,0.13}{\textit{{#1}}}}
    \newcommand{\AnnotationTok}[1]{\textcolor[rgb]{0.38,0.63,0.69}{\textbf{\textit{{#1}}}}}
    \newcommand{\CommentVarTok}[1]{\textcolor[rgb]{0.38,0.63,0.69}{\textbf{\textit{{#1}}}}}
    \newcommand{\VariableTok}[1]{\textcolor[rgb]{0.10,0.09,0.49}{{#1}}}
    \newcommand{\ControlFlowTok}[1]{\textcolor[rgb]{0.00,0.44,0.13}{\textbf{{#1}}}}
    \newcommand{\OperatorTok}[1]{\textcolor[rgb]{0.40,0.40,0.40}{{#1}}}
    \newcommand{\BuiltInTok}[1]{{#1}}
    \newcommand{\ExtensionTok}[1]{{#1}}
    \newcommand{\PreprocessorTok}[1]{\textcolor[rgb]{0.74,0.48,0.00}{{#1}}}
    \newcommand{\AttributeTok}[1]{\textcolor[rgb]{0.49,0.56,0.16}{{#1}}}
    \newcommand{\InformationTok}[1]{\textcolor[rgb]{0.38,0.63,0.69}{\textbf{\textit{{#1}}}}}
    \newcommand{\WarningTok}[1]{\textcolor[rgb]{0.38,0.63,0.69}{\textbf{\textit{{#1}}}}}
    
    
    % Define a nice break command that doesn't care if a line doesn't already
    % exist.
    \def\br{\hspace*{\fill} \\* }
    % Math Jax compatibility definitions
    \def\gt{>}
    \def\lt{<}
    \let\Oldtex\TeX
    \let\Oldlatex\LaTeX
    \renewcommand{\TeX}{\textrm{\Oldtex}}
    \renewcommand{\LaTeX}{\textrm{\Oldlatex}}
    % Document parameters
    % Document title
    \title{Unit5HW}
    
    
    
    
    
% Pygments definitions
\makeatletter
\def\PY@reset{\let\PY@it=\relax \let\PY@bf=\relax%
    \let\PY@ul=\relax \let\PY@tc=\relax%
    \let\PY@bc=\relax \let\PY@ff=\relax}
\def\PY@tok#1{\csname PY@tok@#1\endcsname}
\def\PY@toks#1+{\ifx\relax#1\empty\else%
    \PY@tok{#1}\expandafter\PY@toks\fi}
\def\PY@do#1{\PY@bc{\PY@tc{\PY@ul{%
    \PY@it{\PY@bf{\PY@ff{#1}}}}}}}
\def\PY#1#2{\PY@reset\PY@toks#1+\relax+\PY@do{#2}}

\expandafter\def\csname PY@tok@w\endcsname{\def\PY@tc##1{\textcolor[rgb]{0.73,0.73,0.73}{##1}}}
\expandafter\def\csname PY@tok@c\endcsname{\let\PY@it=\textit\def\PY@tc##1{\textcolor[rgb]{0.25,0.50,0.50}{##1}}}
\expandafter\def\csname PY@tok@cp\endcsname{\def\PY@tc##1{\textcolor[rgb]{0.74,0.48,0.00}{##1}}}
\expandafter\def\csname PY@tok@k\endcsname{\let\PY@bf=\textbf\def\PY@tc##1{\textcolor[rgb]{0.00,0.50,0.00}{##1}}}
\expandafter\def\csname PY@tok@kp\endcsname{\def\PY@tc##1{\textcolor[rgb]{0.00,0.50,0.00}{##1}}}
\expandafter\def\csname PY@tok@kt\endcsname{\def\PY@tc##1{\textcolor[rgb]{0.69,0.00,0.25}{##1}}}
\expandafter\def\csname PY@tok@o\endcsname{\def\PY@tc##1{\textcolor[rgb]{0.40,0.40,0.40}{##1}}}
\expandafter\def\csname PY@tok@ow\endcsname{\let\PY@bf=\textbf\def\PY@tc##1{\textcolor[rgb]{0.67,0.13,1.00}{##1}}}
\expandafter\def\csname PY@tok@nb\endcsname{\def\PY@tc##1{\textcolor[rgb]{0.00,0.50,0.00}{##1}}}
\expandafter\def\csname PY@tok@nf\endcsname{\def\PY@tc##1{\textcolor[rgb]{0.00,0.00,1.00}{##1}}}
\expandafter\def\csname PY@tok@nc\endcsname{\let\PY@bf=\textbf\def\PY@tc##1{\textcolor[rgb]{0.00,0.00,1.00}{##1}}}
\expandafter\def\csname PY@tok@nn\endcsname{\let\PY@bf=\textbf\def\PY@tc##1{\textcolor[rgb]{0.00,0.00,1.00}{##1}}}
\expandafter\def\csname PY@tok@ne\endcsname{\let\PY@bf=\textbf\def\PY@tc##1{\textcolor[rgb]{0.82,0.25,0.23}{##1}}}
\expandafter\def\csname PY@tok@nv\endcsname{\def\PY@tc##1{\textcolor[rgb]{0.10,0.09,0.49}{##1}}}
\expandafter\def\csname PY@tok@no\endcsname{\def\PY@tc##1{\textcolor[rgb]{0.53,0.00,0.00}{##1}}}
\expandafter\def\csname PY@tok@nl\endcsname{\def\PY@tc##1{\textcolor[rgb]{0.63,0.63,0.00}{##1}}}
\expandafter\def\csname PY@tok@ni\endcsname{\let\PY@bf=\textbf\def\PY@tc##1{\textcolor[rgb]{0.60,0.60,0.60}{##1}}}
\expandafter\def\csname PY@tok@na\endcsname{\def\PY@tc##1{\textcolor[rgb]{0.49,0.56,0.16}{##1}}}
\expandafter\def\csname PY@tok@nt\endcsname{\let\PY@bf=\textbf\def\PY@tc##1{\textcolor[rgb]{0.00,0.50,0.00}{##1}}}
\expandafter\def\csname PY@tok@nd\endcsname{\def\PY@tc##1{\textcolor[rgb]{0.67,0.13,1.00}{##1}}}
\expandafter\def\csname PY@tok@s\endcsname{\def\PY@tc##1{\textcolor[rgb]{0.73,0.13,0.13}{##1}}}
\expandafter\def\csname PY@tok@sd\endcsname{\let\PY@it=\textit\def\PY@tc##1{\textcolor[rgb]{0.73,0.13,0.13}{##1}}}
\expandafter\def\csname PY@tok@si\endcsname{\let\PY@bf=\textbf\def\PY@tc##1{\textcolor[rgb]{0.73,0.40,0.53}{##1}}}
\expandafter\def\csname PY@tok@se\endcsname{\let\PY@bf=\textbf\def\PY@tc##1{\textcolor[rgb]{0.73,0.40,0.13}{##1}}}
\expandafter\def\csname PY@tok@sr\endcsname{\def\PY@tc##1{\textcolor[rgb]{0.73,0.40,0.53}{##1}}}
\expandafter\def\csname PY@tok@ss\endcsname{\def\PY@tc##1{\textcolor[rgb]{0.10,0.09,0.49}{##1}}}
\expandafter\def\csname PY@tok@sx\endcsname{\def\PY@tc##1{\textcolor[rgb]{0.00,0.50,0.00}{##1}}}
\expandafter\def\csname PY@tok@m\endcsname{\def\PY@tc##1{\textcolor[rgb]{0.40,0.40,0.40}{##1}}}
\expandafter\def\csname PY@tok@gh\endcsname{\let\PY@bf=\textbf\def\PY@tc##1{\textcolor[rgb]{0.00,0.00,0.50}{##1}}}
\expandafter\def\csname PY@tok@gu\endcsname{\let\PY@bf=\textbf\def\PY@tc##1{\textcolor[rgb]{0.50,0.00,0.50}{##1}}}
\expandafter\def\csname PY@tok@gd\endcsname{\def\PY@tc##1{\textcolor[rgb]{0.63,0.00,0.00}{##1}}}
\expandafter\def\csname PY@tok@gi\endcsname{\def\PY@tc##1{\textcolor[rgb]{0.00,0.63,0.00}{##1}}}
\expandafter\def\csname PY@tok@gr\endcsname{\def\PY@tc##1{\textcolor[rgb]{1.00,0.00,0.00}{##1}}}
\expandafter\def\csname PY@tok@ge\endcsname{\let\PY@it=\textit}
\expandafter\def\csname PY@tok@gs\endcsname{\let\PY@bf=\textbf}
\expandafter\def\csname PY@tok@gp\endcsname{\let\PY@bf=\textbf\def\PY@tc##1{\textcolor[rgb]{0.00,0.00,0.50}{##1}}}
\expandafter\def\csname PY@tok@go\endcsname{\def\PY@tc##1{\textcolor[rgb]{0.53,0.53,0.53}{##1}}}
\expandafter\def\csname PY@tok@gt\endcsname{\def\PY@tc##1{\textcolor[rgb]{0.00,0.27,0.87}{##1}}}
\expandafter\def\csname PY@tok@err\endcsname{\def\PY@bc##1{\setlength{\fboxsep}{0pt}\fcolorbox[rgb]{1.00,0.00,0.00}{1,1,1}{\strut ##1}}}
\expandafter\def\csname PY@tok@kc\endcsname{\let\PY@bf=\textbf\def\PY@tc##1{\textcolor[rgb]{0.00,0.50,0.00}{##1}}}
\expandafter\def\csname PY@tok@kd\endcsname{\let\PY@bf=\textbf\def\PY@tc##1{\textcolor[rgb]{0.00,0.50,0.00}{##1}}}
\expandafter\def\csname PY@tok@kn\endcsname{\let\PY@bf=\textbf\def\PY@tc##1{\textcolor[rgb]{0.00,0.50,0.00}{##1}}}
\expandafter\def\csname PY@tok@kr\endcsname{\let\PY@bf=\textbf\def\PY@tc##1{\textcolor[rgb]{0.00,0.50,0.00}{##1}}}
\expandafter\def\csname PY@tok@bp\endcsname{\def\PY@tc##1{\textcolor[rgb]{0.00,0.50,0.00}{##1}}}
\expandafter\def\csname PY@tok@fm\endcsname{\def\PY@tc##1{\textcolor[rgb]{0.00,0.00,1.00}{##1}}}
\expandafter\def\csname PY@tok@vc\endcsname{\def\PY@tc##1{\textcolor[rgb]{0.10,0.09,0.49}{##1}}}
\expandafter\def\csname PY@tok@vg\endcsname{\def\PY@tc##1{\textcolor[rgb]{0.10,0.09,0.49}{##1}}}
\expandafter\def\csname PY@tok@vi\endcsname{\def\PY@tc##1{\textcolor[rgb]{0.10,0.09,0.49}{##1}}}
\expandafter\def\csname PY@tok@vm\endcsname{\def\PY@tc##1{\textcolor[rgb]{0.10,0.09,0.49}{##1}}}
\expandafter\def\csname PY@tok@sa\endcsname{\def\PY@tc##1{\textcolor[rgb]{0.73,0.13,0.13}{##1}}}
\expandafter\def\csname PY@tok@sb\endcsname{\def\PY@tc##1{\textcolor[rgb]{0.73,0.13,0.13}{##1}}}
\expandafter\def\csname PY@tok@sc\endcsname{\def\PY@tc##1{\textcolor[rgb]{0.73,0.13,0.13}{##1}}}
\expandafter\def\csname PY@tok@dl\endcsname{\def\PY@tc##1{\textcolor[rgb]{0.73,0.13,0.13}{##1}}}
\expandafter\def\csname PY@tok@s2\endcsname{\def\PY@tc##1{\textcolor[rgb]{0.73,0.13,0.13}{##1}}}
\expandafter\def\csname PY@tok@sh\endcsname{\def\PY@tc##1{\textcolor[rgb]{0.73,0.13,0.13}{##1}}}
\expandafter\def\csname PY@tok@s1\endcsname{\def\PY@tc##1{\textcolor[rgb]{0.73,0.13,0.13}{##1}}}
\expandafter\def\csname PY@tok@mb\endcsname{\def\PY@tc##1{\textcolor[rgb]{0.40,0.40,0.40}{##1}}}
\expandafter\def\csname PY@tok@mf\endcsname{\def\PY@tc##1{\textcolor[rgb]{0.40,0.40,0.40}{##1}}}
\expandafter\def\csname PY@tok@mh\endcsname{\def\PY@tc##1{\textcolor[rgb]{0.40,0.40,0.40}{##1}}}
\expandafter\def\csname PY@tok@mi\endcsname{\def\PY@tc##1{\textcolor[rgb]{0.40,0.40,0.40}{##1}}}
\expandafter\def\csname PY@tok@il\endcsname{\def\PY@tc##1{\textcolor[rgb]{0.40,0.40,0.40}{##1}}}
\expandafter\def\csname PY@tok@mo\endcsname{\def\PY@tc##1{\textcolor[rgb]{0.40,0.40,0.40}{##1}}}
\expandafter\def\csname PY@tok@ch\endcsname{\let\PY@it=\textit\def\PY@tc##1{\textcolor[rgb]{0.25,0.50,0.50}{##1}}}
\expandafter\def\csname PY@tok@cm\endcsname{\let\PY@it=\textit\def\PY@tc##1{\textcolor[rgb]{0.25,0.50,0.50}{##1}}}
\expandafter\def\csname PY@tok@cpf\endcsname{\let\PY@it=\textit\def\PY@tc##1{\textcolor[rgb]{0.25,0.50,0.50}{##1}}}
\expandafter\def\csname PY@tok@c1\endcsname{\let\PY@it=\textit\def\PY@tc##1{\textcolor[rgb]{0.25,0.50,0.50}{##1}}}
\expandafter\def\csname PY@tok@cs\endcsname{\let\PY@it=\textit\def\PY@tc##1{\textcolor[rgb]{0.25,0.50,0.50}{##1}}}

\def\PYZbs{\char`\\}
\def\PYZus{\char`\_}
\def\PYZob{\char`\{}
\def\PYZcb{\char`\}}
\def\PYZca{\char`\^}
\def\PYZam{\char`\&}
\def\PYZlt{\char`\<}
\def\PYZgt{\char`\>}
\def\PYZsh{\char`\#}
\def\PYZpc{\char`\%}
\def\PYZdl{\char`\$}
\def\PYZhy{\char`\-}
\def\PYZsq{\char`\'}
\def\PYZdq{\char`\"}
\def\PYZti{\char`\~}
% for compatibility with earlier versions
\def\PYZat{@}
\def\PYZlb{[}
\def\PYZrb{]}
\makeatother


    % For linebreaks inside Verbatim environment from package fancyvrb. 
    \makeatletter
        \newbox\Wrappedcontinuationbox 
        \newbox\Wrappedvisiblespacebox 
        \newcommand*\Wrappedvisiblespace {\textcolor{red}{\textvisiblespace}} 
        \newcommand*\Wrappedcontinuationsymbol {\textcolor{red}{\llap{\tiny$\m@th\hookrightarrow$}}} 
        \newcommand*\Wrappedcontinuationindent {3ex } 
        \newcommand*\Wrappedafterbreak {\kern\Wrappedcontinuationindent\copy\Wrappedcontinuationbox} 
        % Take advantage of the already applied Pygments mark-up to insert 
        % potential linebreaks for TeX processing. 
        %        {, <, #, %, $, ' and ": go to next line. 
        %        _, }, ^, &, >, - and ~: stay at end of broken line. 
        % Use of \textquotesingle for straight quote. 
        \newcommand*\Wrappedbreaksatspecials {% 
            \def\PYGZus{\discretionary{\char`\_}{\Wrappedafterbreak}{\char`\_}}% 
            \def\PYGZob{\discretionary{}{\Wrappedafterbreak\char`\{}{\char`\{}}% 
            \def\PYGZcb{\discretionary{\char`\}}{\Wrappedafterbreak}{\char`\}}}% 
            \def\PYGZca{\discretionary{\char`\^}{\Wrappedafterbreak}{\char`\^}}% 
            \def\PYGZam{\discretionary{\char`\&}{\Wrappedafterbreak}{\char`\&}}% 
            \def\PYGZlt{\discretionary{}{\Wrappedafterbreak\char`\<}{\char`\<}}% 
            \def\PYGZgt{\discretionary{\char`\>}{\Wrappedafterbreak}{\char`\>}}% 
            \def\PYGZsh{\discretionary{}{\Wrappedafterbreak\char`\#}{\char`\#}}% 
            \def\PYGZpc{\discretionary{}{\Wrappedafterbreak\char`\%}{\char`\%}}% 
            \def\PYGZdl{\discretionary{}{\Wrappedafterbreak\char`\$}{\char`\$}}% 
            \def\PYGZhy{\discretionary{\char`\-}{\Wrappedafterbreak}{\char`\-}}% 
            \def\PYGZsq{\discretionary{}{\Wrappedafterbreak\textquotesingle}{\textquotesingle}}% 
            \def\PYGZdq{\discretionary{}{\Wrappedafterbreak\char`\"}{\char`\"}}% 
            \def\PYGZti{\discretionary{\char`\~}{\Wrappedafterbreak}{\char`\~}}% 
        } 
        % Some characters . , ; ? ! / are not pygmentized. 
        % This macro makes them "active" and they will insert potential linebreaks 
        \newcommand*\Wrappedbreaksatpunct {% 
            \lccode`\~`\.\lowercase{\def~}{\discretionary{\hbox{\char`\.}}{\Wrappedafterbreak}{\hbox{\char`\.}}}% 
            \lccode`\~`\,\lowercase{\def~}{\discretionary{\hbox{\char`\,}}{\Wrappedafterbreak}{\hbox{\char`\,}}}% 
            \lccode`\~`\;\lowercase{\def~}{\discretionary{\hbox{\char`\;}}{\Wrappedafterbreak}{\hbox{\char`\;}}}% 
            \lccode`\~`\:\lowercase{\def~}{\discretionary{\hbox{\char`\:}}{\Wrappedafterbreak}{\hbox{\char`\:}}}% 
            \lccode`\~`\?\lowercase{\def~}{\discretionary{\hbox{\char`\?}}{\Wrappedafterbreak}{\hbox{\char`\?}}}% 
            \lccode`\~`\!\lowercase{\def~}{\discretionary{\hbox{\char`\!}}{\Wrappedafterbreak}{\hbox{\char`\!}}}% 
            \lccode`\~`\/\lowercase{\def~}{\discretionary{\hbox{\char`\/}}{\Wrappedafterbreak}{\hbox{\char`\/}}}% 
            \catcode`\.\active
            \catcode`\,\active 
            \catcode`\;\active
            \catcode`\:\active
            \catcode`\?\active
            \catcode`\!\active
            \catcode`\/\active 
            \lccode`\~`\~ 	
        }
    \makeatother

    \let\OriginalVerbatim=\Verbatim
    \makeatletter
    \renewcommand{\Verbatim}[1][1]{%
        %\parskip\z@skip
        \sbox\Wrappedcontinuationbox {\Wrappedcontinuationsymbol}%
        \sbox\Wrappedvisiblespacebox {\FV@SetupFont\Wrappedvisiblespace}%
        \def\FancyVerbFormatLine ##1{\hsize\linewidth
            \vtop{\raggedright\hyphenpenalty\z@\exhyphenpenalty\z@
                \doublehyphendemerits\z@\finalhyphendemerits\z@
                \strut ##1\strut}%
        }%
        % If the linebreak is at a space, the latter will be displayed as visible
        % space at end of first line, and a continuation symbol starts next line.
        % Stretch/shrink are however usually zero for typewriter font.
        \def\FV@Space {%
            \nobreak\hskip\z@ plus\fontdimen3\font minus\fontdimen4\font
            \discretionary{\copy\Wrappedvisiblespacebox}{\Wrappedafterbreak}
            {\kern\fontdimen2\font}%
        }%
        
        % Allow breaks at special characters using \PYG... macros.
        \Wrappedbreaksatspecials
        % Breaks at punctuation characters . , ; ? ! and / need catcode=\active 	
        \OriginalVerbatim[#1,codes*=\Wrappedbreaksatpunct]%
    }
    \makeatother

    % Exact colors from NB
    \definecolor{incolor}{HTML}{303F9F}
    \definecolor{outcolor}{HTML}{D84315}
    \definecolor{cellborder}{HTML}{CFCFCF}
    \definecolor{cellbackground}{HTML}{F7F7F7}
    
    % prompt
    \makeatletter
    \newcommand{\boxspacing}{\kern\kvtcb@left@rule\kern\kvtcb@boxsep}
    \makeatother
    \newcommand{\prompt}[4]{
        \ttfamily\llap{{\color{#2}[#3]:\hspace{3pt}#4}}\vspace{-\baselineskip}
    }
    

    
    % Prevent overflowing lines due to hard-to-break entities
    \sloppy 
    % Setup hyperref package
    \hypersetup{
      breaklinks=true,  % so long urls are correctly broken across lines
      colorlinks=true,
      urlcolor=urlcolor,
      linkcolor=linkcolor,
      citecolor=citecolor,
      }
    % Slightly bigger margins than the latex defaults
    
    \geometry{verbose,tmargin=1in,bmargin=1in,lmargin=1in,rmargin=1in}
    
    

\begin{document}
    
    \maketitle
    
    

    
    \hypertarget{unit5ux4f5cux4e1a}{%
\section{Unit5作业}\label{unit5ux4f5cux4e1a}}

    \begin{tcolorbox}[breakable, size=fbox, boxrule=1pt, pad at break*=1mm,colback=cellbackground, colframe=cellborder]
\prompt{In}{incolor}{18}{\boxspacing}
\begin{Verbatim}[commandchars=\\\{\}]
\PY{k+kn}{from} \PY{n+nn}{IPython}\PY{n+nn}{.}\PY{n+nn}{core}\PY{n+nn}{.}\PY{n+nn}{interactiveshell} \PY{k+kn}{import} \PY{n}{InteractiveShell}
\PY{n}{InteractiveShell}\PY{o}{.}\PY{n}{ast\PYZus{}node\PYZus{}interactivity} \PY{o}{=} \PY{l+s+s1}{\PYZsq{}}\PY{l+s+s1}{all}\PY{l+s+s1}{\PYZsq{}}
\PY{k+kn}{import} \PY{n+nn}{scipy}\PY{n+nn}{.}\PY{n+nn}{stats} \PY{k}{as} \PY{n+nn}{stats}
\PY{k+kn}{import} \PY{n+nn}{numpy} \PY{k}{as} \PY{n+nn}{np}
\PY{k+kn}{import} \PY{n+nn}{statsmodels}\PY{n+nn}{.}\PY{n+nn}{stats}\PY{n+nn}{.}\PY{n+nn}{proportion} \PY{k}{as} \PY{n+nn}{proportion}
\PY{k+kn}{import} \PY{n+nn}{statsmodels}\PY{n+nn}{.}\PY{n+nn}{formula}\PY{n+nn}{.}\PY{n+nn}{api} \PY{k}{as} \PY{n+nn}{smf}
\PY{k+kn}{import} \PY{n+nn}{matplotlib}\PY{n+nn}{.}\PY{n+nn}{pyplot} \PY{k}{as} \PY{n+nn}{plt}
\PY{k+kn}{import} \PY{n+nn}{seaborn} \PY{k}{as} \PY{n+nn}{sns}
\PY{k+kn}{import} \PY{n+nn}{pandas} \PY{k}{as} \PY{n+nn}{pd}
\PY{n}{sns}\PY{o}{.}\PY{n}{set\PYZus{}style}\PY{p}{(}\PY{l+s+s1}{\PYZsq{}}\PY{l+s+s1}{darkgrid}\PY{l+s+s1}{\PYZsq{}}\PY{p}{)}
\end{Verbatim}
\end{tcolorbox}

    \hypertarget{hw-u5-1ux5bf9ux4e8ediabetes.csvux6570ux636eux8bf7ux5229ux7528ux534fux65b9ux5deeux548cpearson-ux76f8ux5173ux7cfbux6570ux5206ux6790glucoseux4e0ebloodpressureux7684ux5173ux7cfb-0.5ux5206-ux5e76ux753bux51faux8840ux7cd6-ux8840ux538bux6563ux70b9ux56fe0.5ux5206}{%
\paragraph{HW-U5-1:对于Diabetes.csv数据,请利用协方差和pearson
相关系数分析Glucose与BloodPressure的关系 (0.5分),
并画出血糖-血压散点图(0.5分)}\label{hw-u5-1ux5bf9ux4e8ediabetes.csvux6570ux636eux8bf7ux5229ux7528ux534fux65b9ux5deeux548cpearson-ux76f8ux5173ux7cfbux6570ux5206ux6790glucoseux4e0ebloodpressureux7684ux5173ux7cfb-0.5ux5206-ux5e76ux753bux51faux8840ux7cd6-ux8840ux538bux6563ux70b9ux56fe0.5ux5206}}

    \begin{tcolorbox}[breakable, size=fbox, boxrule=1pt, pad at break*=1mm,colback=cellbackground, colframe=cellborder]
\prompt{In}{incolor}{19}{\boxspacing}
\begin{Verbatim}[commandchars=\\\{\}]
\PY{n}{diabetes}\PY{o}{=}\PY{n}{pd}\PY{o}{.}\PY{n}{read\PYZus{}csv}\PY{p}{(}\PY{l+s+s1}{\PYZsq{}}\PY{l+s+s1}{Diabetes.csv}\PY{l+s+s1}{\PYZsq{}}\PY{p}{)}
\PY{n}{diabetes}\PY{o}{.}\PY{n}{head}\PY{p}{(}\PY{l+m+mi}{5}\PY{p}{)}
\end{Verbatim}
\end{tcolorbox}

            \begin{tcolorbox}[breakable, size=fbox, boxrule=.5pt, pad at break*=1mm, opacityfill=0]
\prompt{Out}{outcolor}{19}{\boxspacing}
\begin{Verbatim}[commandchars=\\\{\}]
   Pregnancies  Glucose  BloodPressure  SkinThickness  Insulin   BMI  \textbackslash{}
0            6      148             72             35        0  33.6
1            1       85             66             29        0  26.6
2            8      183             64              0        0  23.3
3            1       89             66             23       94  28.1
4            0      137             40             35      168  43.1

   DiabetesPedigreeFunction  Age  Outcome
0                     0.627   50        1
1                     0.351   31        0
2                     0.672   32        1
3                     0.167   21        0
4                     2.288   33        1
\end{Verbatim}
\end{tcolorbox}
        
    \begin{tcolorbox}[breakable, size=fbox, boxrule=1pt, pad at break*=1mm,colback=cellbackground, colframe=cellborder]
\prompt{In}{incolor}{20}{\boxspacing}
\begin{Verbatim}[commandchars=\\\{\}]
\PY{n}{diabetes}\PY{o}{=}\PY{n}{diabetes}\PY{p}{[}\PY{n}{diabetes}\PY{p}{[}\PY{l+s+s1}{\PYZsq{}}\PY{l+s+s1}{Glucose}\PY{l+s+s1}{\PYZsq{}}\PY{p}{]}\PY{o}{!=}\PY{l+m+mi}{0}\PY{p}{]}
\PY{n}{diabetes}\PY{o}{=}\PY{n}{diabetes}\PY{p}{[}\PY{n}{diabetes}\PY{p}{[}\PY{l+s+s1}{\PYZsq{}}\PY{l+s+s1}{BloodPressure}\PY{l+s+s1}{\PYZsq{}}\PY{p}{]}\PY{o}{!=}\PY{l+m+mi}{0}\PY{p}{]}
\PY{n}{glu}\PY{o}{=}\PY{n}{diabetes}\PY{p}{[}\PY{l+s+s1}{\PYZsq{}}\PY{l+s+s1}{Glucose}\PY{l+s+s1}{\PYZsq{}}\PY{p}{]}
\PY{n}{bp}\PY{o}{=}\PY{n}{diabetes}\PY{p}{[}\PY{l+s+s1}{\PYZsq{}}\PY{l+s+s1}{BloodPressure}\PY{l+s+s1}{\PYZsq{}}\PY{p}{]}
\PY{n}{plt}\PY{o}{.}\PY{n}{scatter}\PY{p}{(}\PY{n}{glu}\PY{p}{,}\PY{n}{bp}\PY{p}{)}
\end{Verbatim}
\end{tcolorbox}

            \begin{tcolorbox}[breakable, size=fbox, boxrule=.5pt, pad at break*=1mm, opacityfill=0]
\prompt{Out}{outcolor}{20}{\boxspacing}
\begin{Verbatim}[commandchars=\\\{\}]
<matplotlib.collections.PathCollection at 0x1f489300508>
\end{Verbatim}
\end{tcolorbox}
        
    \begin{center}
    \adjustimage{max size={0.9\linewidth}{0.9\paperheight}}{output_4_1.png}
    \end{center}
    { \hspace*{\fill} \\}
    
    \begin{tcolorbox}[breakable, size=fbox, boxrule=1pt, pad at break*=1mm,colback=cellbackground, colframe=cellborder]
\prompt{In}{incolor}{21}{\boxspacing}
\begin{Verbatim}[commandchars=\\\{\}]
\PY{n}{np}\PY{o}{.}\PY{n}{cov}\PY{p}{(}\PY{n}{glu}\PY{p}{,}\PY{n}{bp}\PY{p}{)}
\PY{n}{stats}\PY{o}{.}\PY{n}{pearsonr}\PY{p}{(}\PY{n}{glu}\PY{p}{,}\PY{n}{bp}\PY{p}{)}
\end{Verbatim}
\end{tcolorbox}

            \begin{tcolorbox}[breakable, size=fbox, boxrule=.5pt, pad at break*=1mm, opacityfill=0]
\prompt{Out}{outcolor}{21}{\boxspacing}
\begin{Verbatim}[commandchars=\\\{\}]
array([[941.21371888,  84.81198513],
       [ 84.81198513, 153.4157062 ]])
\end{Verbatim}
\end{tcolorbox}
        
            \begin{tcolorbox}[breakable, size=fbox, boxrule=.5pt, pad at break*=1mm, opacityfill=0]
\prompt{Out}{outcolor}{21}{\boxspacing}
\begin{Verbatim}[commandchars=\\\{\}]
(0.223191778249542, 1.138581203805524e-09)
\end{Verbatim}
\end{tcolorbox}
        
    pearson's
r\(=0.22\),p\(=1.14\times 10^{-9}.\)该结果显示Glucose与BloodPressure之间没有明显的相关关系。

    \hypertarget{hw-5-2-ux5bf9ux4e8etitantic.csvux6570ux636eux53c2ux89c1ux524dux9762ux5355ux5143ux4f5cux4e1aux8bf7ux7528ux5206ux522bux7528pearson-r-spearman-rho-kendalls-tauux5206ux522bux8ba1ux7b97ux4e58ux5ba2ux5e74ux9f84ux4e0eux4e70ux7684ux7968ux7684ux7b49ux7ea7ux7684ux76f8ux5173ux7cfbux6570-1ux5206}{%
\paragraph{HW-5-2:
对于Titantic.csv数据(参见前面单元作业),请用分别用pearson r, spearman
rho, kendall's tau分别计算乘客年龄与买的票的等级的相关系数
(1分)}\label{hw-5-2-ux5bf9ux4e8etitantic.csvux6570ux636eux53c2ux89c1ux524dux9762ux5355ux5143ux4f5cux4e1aux8bf7ux7528ux5206ux522bux7528pearson-r-spearman-rho-kendalls-tauux5206ux522bux8ba1ux7b97ux4e58ux5ba2ux5e74ux9f84ux4e0eux4e70ux7684ux7968ux7684ux7b49ux7ea7ux7684ux76f8ux5173ux7cfbux6570-1ux5206}}

    \begin{tcolorbox}[breakable, size=fbox, boxrule=1pt, pad at break*=1mm,colback=cellbackground, colframe=cellborder]
\prompt{In}{incolor}{22}{\boxspacing}
\begin{Verbatim}[commandchars=\\\{\}]
\PY{n}{titantic}\PY{o}{=}\PY{n}{pd}\PY{o}{.}\PY{n}{read\PYZus{}csv}\PY{p}{(}\PY{l+s+s1}{\PYZsq{}}\PY{l+s+s1}{Titanic.csv}\PY{l+s+s1}{\PYZsq{}}\PY{p}{)}
\PY{n}{titantic}\PY{o}{.}\PY{n}{head}\PY{p}{(}\PY{l+m+mi}{5}\PY{p}{)}
\PY{n}{titantic}\PY{o}{=}\PY{n}{titantic}\PY{o}{.}\PY{n}{dropna}\PY{p}{(}\PY{p}{)}
\end{Verbatim}
\end{tcolorbox}

            \begin{tcolorbox}[breakable, size=fbox, boxrule=.5pt, pad at break*=1mm, opacityfill=0]
\prompt{Out}{outcolor}{22}{\boxspacing}
\begin{Verbatim}[commandchars=\\\{\}]
                                            Name PClass    Age     Sex  \textbackslash{}
0                   Allen, Miss Elisabeth Walton    1st  29.00  female
1                    Allison, Miss Helen Loraine    1st   2.00  female
2            Allison, Mr Hudson Joshua Creighton    1st  30.00    male
3  Allison, Mrs Hudson JC (Bessie Waldo Daniels)    1st  25.00  female
4                  Allison, Master Hudson Trevor    1st   0.92    male

   Survived
0         1
1         0
2         0
3         0
4         1
\end{Verbatim}
\end{tcolorbox}
        
    \begin{tcolorbox}[breakable, size=fbox, boxrule=1pt, pad at break*=1mm,colback=cellbackground, colframe=cellborder]
\prompt{In}{incolor}{23}{\boxspacing}
\begin{Verbatim}[commandchars=\\\{\}]
\PY{n}{age}\PY{o}{=}\PY{n}{titantic}\PY{p}{[}\PY{l+s+s1}{\PYZsq{}}\PY{l+s+s1}{Age}\PY{l+s+s1}{\PYZsq{}}\PY{p}{]}
\PY{n}{pclass}\PY{o}{=}\PY{n}{titantic}\PY{p}{[}\PY{l+s+s1}{\PYZsq{}}\PY{l+s+s1}{PClass}\PY{l+s+s1}{\PYZsq{}}\PY{p}{]}
\PY{n}{pclass}\PY{o}{=}\PY{n}{pclass}\PY{o}{.}\PY{n}{replace}\PY{p}{(}\PY{p}{[}\PY{l+s+s1}{\PYZsq{}}\PY{l+s+s1}{1st}\PY{l+s+s1}{\PYZsq{}}\PY{p}{,}\PY{l+s+s1}{\PYZsq{}}\PY{l+s+s1}{2nd}\PY{l+s+s1}{\PYZsq{}}\PY{p}{,}\PY{l+s+s1}{\PYZsq{}}\PY{l+s+s1}{3rd}\PY{l+s+s1}{\PYZsq{}}\PY{p}{]}\PY{p}{,}\PY{p}{[}\PY{l+m+mi}{1}\PY{p}{,}\PY{l+m+mi}{2}\PY{p}{,}\PY{l+m+mi}{3}\PY{p}{]}\PY{p}{)}
\end{Verbatim}
\end{tcolorbox}

    \begin{tcolorbox}[breakable, size=fbox, boxrule=1pt, pad at break*=1mm,colback=cellbackground, colframe=cellborder]
\prompt{In}{incolor}{24}{\boxspacing}
\begin{Verbatim}[commandchars=\\\{\}]
\PY{n}{stats}\PY{o}{.}\PY{n}{pearsonr}\PY{p}{(}\PY{n}{age}\PY{p}{,}\PY{n}{pclass}\PY{p}{)}
\PY{n}{stats}\PY{o}{.}\PY{n}{spearmanr}\PY{p}{(}\PY{n}{age}\PY{p}{,}\PY{n}{pclass}\PY{p}{)}
\PY{n}{stats}\PY{o}{.}\PY{n}{kendalltau}\PY{p}{(}\PY{n}{age}\PY{p}{,}\PY{n}{pclass}\PY{p}{)}
\end{Verbatim}
\end{tcolorbox}

            \begin{tcolorbox}[breakable, size=fbox, boxrule=.5pt, pad at break*=1mm, opacityfill=0]
\prompt{Out}{outcolor}{24}{\boxspacing}
\begin{Verbatim}[commandchars=\\\{\}]
(-0.4141214595264922, 1.0969903610990536e-32)
\end{Verbatim}
\end{tcolorbox}
        
            \begin{tcolorbox}[breakable, size=fbox, boxrule=.5pt, pad at break*=1mm, opacityfill=0]
\prompt{Out}{outcolor}{24}{\boxspacing}
\begin{Verbatim}[commandchars=\\\{\}]
SpearmanrResult(correlation=-0.39366216507025165, pvalue=1.9756955531661058e-29)
\end{Verbatim}
\end{tcolorbox}
        
            \begin{tcolorbox}[breakable, size=fbox, boxrule=.5pt, pad at break*=1mm, opacityfill=0]
\prompt{Out}{outcolor}{24}{\boxspacing}
\begin{Verbatim}[commandchars=\\\{\}]
KendalltauResult(correlation=-0.3103724477828569, pvalue=8.600565143832718e-28)
\end{Verbatim}
\end{tcolorbox}
        
    以上三个结果的p-value均很小,均反映了Age与PClass之间没有显著的相关关系.

    \hypertarget{hw-5-3-ux9488ux5bf9ux6c7dux8f66ux6570ux636emtcars.csv}{%
\paragraph{HW-5-3 :
针对汽车数据mtcars.csv}\label{hw-5-3-ux9488ux5bf9ux6c7dux8f66ux6570ux636emtcars.csv}}

    \hypertarget{ux753bux51fawt-mpgux6563ux70b9ux56feux7528ux7b80ux5355ux7ebfux6027ux56deux5f52ux5206ux6790mpgux56e0ux53d8ux91cfux548cwtux81eaux53d8ux91cfux7684ux5173ux7cfbux5e76ux6839ux636eux56deux5f52ux7ed3ux679cux4e2dux7684ux622aux8dddux548cux659cux7387ux53caux5176ux663eux8457ux6027ux6c34ux5e73pux503c-ux5bf9ux7ed3ux679cux8fdbux884cux89e3ux91ca-ux5e76ux89e3ux91car-square.1.5ux5206}{%
\subparagraph{(1)画出wt\textasciitilde{}
mpg散点图;用简单线性回归分析mpg(因变量),和wt(自变量)的关系,并
根据回归结果中的截距和斜率及其显著性水平(p值),对结果进行解释;并解释Rsquare.(1.5分)}\label{ux753bux51fawt-mpgux6563ux70b9ux56feux7528ux7b80ux5355ux7ebfux6027ux56deux5f52ux5206ux6790mpgux56e0ux53d8ux91cfux548cwtux81eaux53d8ux91cfux7684ux5173ux7cfbux5e76ux6839ux636eux56deux5f52ux7ed3ux679cux4e2dux7684ux622aux8dddux548cux659cux7387ux53caux5176ux663eux8457ux6027ux6c34ux5e73pux503c-ux5bf9ux7ed3ux679cux8fdbux884cux89e3ux91ca-ux5e76ux89e3ux91car-square.1.5ux5206}}

    \begin{tcolorbox}[breakable, size=fbox, boxrule=1pt, pad at break*=1mm,colback=cellbackground, colframe=cellborder]
\prompt{In}{incolor}{25}{\boxspacing}
\begin{Verbatim}[commandchars=\\\{\}]
\PY{n}{df}\PY{o}{=}\PY{n}{pd}\PY{o}{.}\PY{n}{read\PYZus{}csv}\PY{p}{(}\PY{l+s+s1}{\PYZsq{}}\PY{l+s+s1}{mtcars.csv}\PY{l+s+s1}{\PYZsq{}}\PY{p}{)}
\PY{n}{df}\PY{o}{.}\PY{n}{head}\PY{p}{(}\PY{l+m+mi}{5}\PY{p}{)}
\end{Verbatim}
\end{tcolorbox}

            \begin{tcolorbox}[breakable, size=fbox, boxrule=.5pt, pad at break*=1mm, opacityfill=0]
\prompt{Out}{outcolor}{25}{\boxspacing}
\begin{Verbatim}[commandchars=\\\{\}]
          Unnamed: 0   mpg  cyl   disp   hp  drat     wt   qsec  vs  am  gear  \textbackslash{}
0          Mazda RX4  21.0    6  160.0  110  3.90  2.620  16.46   0   1     4
1      Mazda RX4 Wag  21.0    6  160.0  110  3.90  2.875  17.02   0   1     4
2         Datsun 710  22.8    4  108.0   93  3.85  2.320  18.61   1   1     4
3     Hornet 4 Drive  21.4    6  258.0  110  3.08  3.215  19.44   1   0     3
4  Hornet Sportabout  18.7    8  360.0  175  3.15  3.440  17.02   0   0     3

   carb
0     4
1     4
2     1
3     1
4     2
\end{Verbatim}
\end{tcolorbox}
        
    \begin{tcolorbox}[breakable, size=fbox, boxrule=1pt, pad at break*=1mm,colback=cellbackground, colframe=cellborder]
\prompt{In}{incolor}{26}{\boxspacing}
\begin{Verbatim}[commandchars=\\\{\}]
\PY{n}{sns}\PY{o}{.}\PY{n}{scatterplot}\PY{p}{(}\PY{n}{x}\PY{o}{=}\PY{l+s+s1}{\PYZsq{}}\PY{l+s+s1}{wt}\PY{l+s+s1}{\PYZsq{}}\PY{p}{,}\PY{n}{y}\PY{o}{=}\PY{l+s+s1}{\PYZsq{}}\PY{l+s+s1}{mpg}\PY{l+s+s1}{\PYZsq{}}\PY{p}{,}\PY{n}{data}\PY{o}{=}\PY{n}{df}\PY{p}{)}
\end{Verbatim}
\end{tcolorbox}

            \begin{tcolorbox}[breakable, size=fbox, boxrule=.5pt, pad at break*=1mm, opacityfill=0]
\prompt{Out}{outcolor}{26}{\boxspacing}
\begin{Verbatim}[commandchars=\\\{\}]
<matplotlib.axes.\_subplots.AxesSubplot at 0x1f489864948>
\end{Verbatim}
\end{tcolorbox}
        
    \begin{center}
    \adjustimage{max size={0.9\linewidth}{0.9\paperheight}}{output_15_1.png}
    \end{center}
    { \hspace*{\fill} \\}
    
    \begin{tcolorbox}[breakable, size=fbox, boxrule=1pt, pad at break*=1mm,colback=cellbackground, colframe=cellborder]
\prompt{In}{incolor}{27}{\boxspacing}
\begin{Verbatim}[commandchars=\\\{\}]
\PY{n}{result1}\PY{o}{=}\PY{n}{smf}\PY{o}{.}\PY{n}{ols}\PY{p}{(}\PY{l+s+s1}{\PYZsq{}}\PY{l+s+s1}{mpg\PYZti{}wt}\PY{l+s+s1}{\PYZsq{}}\PY{p}{,}\PY{n}{data}\PY{o}{=}\PY{n}{df}\PY{p}{)}\PY{o}{.}\PY{n}{fit}\PY{p}{(}\PY{p}{)}
\PY{n}{result1}\PY{o}{.}\PY{n}{summary}\PY{p}{(}\PY{p}{)}
\end{Verbatim}
\end{tcolorbox}

            \begin{tcolorbox}[breakable, size=fbox, boxrule=.5pt, pad at break*=1mm, opacityfill=0]
\prompt{Out}{outcolor}{27}{\boxspacing}
\begin{Verbatim}[commandchars=\\\{\}]
<class 'statsmodels.iolib.summary.Summary'>
"""
                            OLS Regression Results
==============================================================================
Dep. Variable:                    mpg   R-squared:                       0.753
Model:                            OLS   Adj. R-squared:                  0.745
Method:                 Least Squares   F-statistic:                     91.38
Date:                Tue, 19 May 2020   Prob (F-statistic):           1.29e-10
Time:                        20:04:26   Log-Likelihood:                -80.015
No. Observations:                  32   AIC:                             164.0
Df Residuals:                      30   BIC:                             167.0
Df Model:                           1
Covariance Type:            nonrobust
==============================================================================
                 coef    std err          t      P>|t|      [0.025      0.975]
------------------------------------------------------------------------------
Intercept     37.2851      1.878     19.858      0.000      33.450      41.120
wt            -5.3445      0.559     -9.559      0.000      -6.486      -4.203
==============================================================================
Omnibus:                        2.988   Durbin-Watson:                   1.252
Prob(Omnibus):                  0.225   Jarque-Bera (JB):                2.399
Skew:                           0.668   Prob(JB):                        0.301
Kurtosis:                       2.877   Cond. No.                         12.7
==============================================================================

Warnings:
[1] Standard Errors assume that the covariance matrix of the errors is correctly
specified.
"""
\end{Verbatim}
\end{tcolorbox}
        
    回归结果\(mpg=-5.3445wt+37.2851\),截距和斜率的p值均很小,说明截距和斜率均有显著意义.\(R^2=0.753\)说明mpg与wt之间有比较强的线性关系(strong).

    \hypertarget{ux7528ux591aux5143ux7ebfux6027ux56deux5f52ux5206ux6790mpgux56e0ux53d8ux91cfux548cwtux81eaux53d8ux91cfhpux81eaux53d8ux91cfux7684ux5173ux7cfbux5e76ux6839ux636eux56deux5f52ux7ed3ux679cux4e2dux7684ux5404ux4e2aux81eaux53d8ux91cfux7684ux7cfbux6570ux53caux5176ux663eux8457ux6027ux6c34ux5e73pux503c-ux5bf9ux56deux5f52ux7ed3ux679cux8fdbux884cux89e3ux91ca-ux5e76ux89e3ux91car-square-1.5ux5206}{%
\subparagraph{(2)用多元线性回归分析mpg(因变量),和wt(自变量)、hp(自变量)的关系,并根据回归
结果中的各个自变量的系数及其显著性水平(p值),
对回归结果进行解释; 并解释R-square
(1.5分)}\label{ux7528ux591aux5143ux7ebfux6027ux56deux5f52ux5206ux6790mpgux56e0ux53d8ux91cfux548cwtux81eaux53d8ux91cfhpux81eaux53d8ux91cfux7684ux5173ux7cfbux5e76ux6839ux636eux56deux5f52ux7ed3ux679cux4e2dux7684ux5404ux4e2aux81eaux53d8ux91cfux7684ux7cfbux6570ux53caux5176ux663eux8457ux6027ux6c34ux5e73pux503c-ux5bf9ux56deux5f52ux7ed3ux679cux8fdbux884cux89e3ux91ca-ux5e76ux89e3ux91car-square-1.5ux5206}}

    \begin{tcolorbox}[breakable, size=fbox, boxrule=1pt, pad at break*=1mm,colback=cellbackground, colframe=cellborder]
\prompt{In}{incolor}{28}{\boxspacing}
\begin{Verbatim}[commandchars=\\\{\}]
\PY{n}{result2}\PY{o}{=}\PY{n}{smf}\PY{o}{.}\PY{n}{ols}\PY{p}{(}\PY{l+s+s1}{\PYZsq{}}\PY{l+s+s1}{mpg\PYZti{}wt+hp}\PY{l+s+s1}{\PYZsq{}}\PY{p}{,}\PY{n}{data}\PY{o}{=}\PY{n}{df}\PY{p}{)}\PY{o}{.}\PY{n}{fit}\PY{p}{(}\PY{p}{)}
\PY{n}{result2}\PY{o}{.}\PY{n}{summary}\PY{p}{(}\PY{p}{)}
\end{Verbatim}
\end{tcolorbox}

            \begin{tcolorbox}[breakable, size=fbox, boxrule=.5pt, pad at break*=1mm, opacityfill=0]
\prompt{Out}{outcolor}{28}{\boxspacing}
\begin{Verbatim}[commandchars=\\\{\}]
<class 'statsmodels.iolib.summary.Summary'>
"""
                            OLS Regression Results
==============================================================================
Dep. Variable:                    mpg   R-squared:                       0.827
Model:                            OLS   Adj. R-squared:                  0.815
Method:                 Least Squares   F-statistic:                     69.21
Date:                Tue, 19 May 2020   Prob (F-statistic):           9.11e-12
Time:                        20:04:26   Log-Likelihood:                -74.326
No. Observations:                  32   AIC:                             154.7
Df Residuals:                      29   BIC:                             159.0
Df Model:                           2
Covariance Type:            nonrobust
==============================================================================
                 coef    std err          t      P>|t|      [0.025      0.975]
------------------------------------------------------------------------------
Intercept     37.2273      1.599     23.285      0.000      33.957      40.497
wt            -3.8778      0.633     -6.129      0.000      -5.172      -2.584
hp            -0.0318      0.009     -3.519      0.001      -0.050      -0.013
==============================================================================
Omnibus:                        5.303   Durbin-Watson:                   1.362
Prob(Omnibus):                  0.071   Jarque-Bera (JB):                4.046
Skew:                           0.855   Prob(JB):                        0.132
Kurtosis:                       3.332   Cond. No.                         588.
==============================================================================

Warnings:
[1] Standard Errors assume that the covariance matrix of the errors is correctly
specified.
"""
\end{Verbatim}
\end{tcolorbox}
        
    回归结果\(mpg=-3.8778wt-0.0318hp+37.2273\),三个系数的p值均小于0.05,说明结果具有显著
    意义。mpg与wt和hp都成负相关,但且wt对mpg的影响要高于hp。
    
    \(R^2=0.827\), 说明mpg与wt和hp之间有很强的二元线性关系。相比单变量回归,\(R^2\)值提高了,结合\(Adj-R^2\),也没有出现明显的过拟合现象,因此模型较单变量效果更好。

    \hypertarget{hw-5-4mtcars.csv-ux6570ux636eux6c7dux8f66ux7684ux79bbux5408am-ux624bux52a81ux81eaux52a80ux4e0eux6c7dux8f66ux7684ux6cb9ux8017mpgux9a6cux529bhpux662fux5f88ux76f8ux5173ux7684ux8bf7}{%
\paragraph{HW-5-4:mtcars.csv 数据),汽车的离合(am:
手动(1)/自动(0))与汽车的油耗(mpg),
马力(hp)是很相关的,请:}\label{hw-5-4mtcars.csv-ux6570ux636eux6c7dux8f66ux7684ux79bbux5408am-ux624bux52a81ux81eaux52a80ux4e0eux6c7dux8f66ux7684ux6cb9ux8017mpgux9a6cux529bhpux662fux5f88ux76f8ux5173ux7684ux8bf7}}

\hypertarget{ux57faux4e8eux5168ux90e8ux6570ux636eux7528mpg-hpux4f5cux4e3aux81eaux53d8ux91cfamux4f5cux4e3aux56e0ux53d8ux91cfux5efaux7acbux5bf9ux5e94ux7684ux903bux8f91ux56deux5f52ux6a21ux578b-ux5e76ux4f5cux51faux89e3ux91ca-1.0ux5206}{%
\subparagraph{(1)基于全部数据用mpg,hp作为自变量,am作为因变量,建立对应的逻辑回归模型 ,并作出解释
(1.0分)}\label{ux57faux4e8eux5168ux90e8ux6570ux636eux7528mpg-hpux4f5cux4e3aux81eaux53d8ux91cfamux4f5cux4e3aux56e0ux53d8ux91cfux5efaux7acbux5bf9ux5e94ux7684ux903bux8f91ux56deux5f52ux6a21ux578b-ux5e76ux4f5cux51faux89e3ux91ca-1.0ux5206}}

    \begin{tcolorbox}[breakable, size=fbox, boxrule=1pt, pad at break*=1mm,colback=cellbackground, colframe=cellborder]
\prompt{In}{incolor}{29}{\boxspacing}
\begin{Verbatim}[commandchars=\\\{\}]
\PY{n}{log\PYZus{}res}\PY{o}{=}\PY{n}{smf}\PY{o}{.}\PY{n}{logit}\PY{p}{(}\PY{l+s+s1}{\PYZsq{}}\PY{l+s+s1}{am\PYZti{}mpg+hp}\PY{l+s+s1}{\PYZsq{}}\PY{p}{,}\PY{n}{data}\PY{o}{=}\PY{n}{df}\PY{p}{)}\PY{o}{.}\PY{n}{fit}\PY{p}{(}\PY{p}{)}
\PY{n}{log\PYZus{}res}\PY{o}{.}\PY{n}{summary}\PY{p}{(}\PY{p}{)}
\end{Verbatim}
\end{tcolorbox}

    \begin{Verbatim}[commandchars=\\\{\}]
Optimization terminated successfully.
         Current function value: 0.300509
         Iterations 9
    \end{Verbatim}

            \begin{tcolorbox}[breakable, size=fbox, boxrule=.5pt, pad at break*=1mm, opacityfill=0]
\prompt{Out}{outcolor}{29}{\boxspacing}
\begin{Verbatim}[commandchars=\\\{\}]
<class 'statsmodels.iolib.summary.Summary'>
"""
                           Logit Regression Results
==============================================================================
Dep. Variable:                     am   No. Observations:                   32
Model:                          Logit   Df Residuals:                       29
Method:                           MLE   Df Model:                            2
Date:                Tue, 19 May 2020   Pseudo R-squ.:                  0.5551
Time:                        20:04:26   Log-Likelihood:                -9.6163
converged:                       True   LL-Null:                       -21.615
Covariance Type:            nonrobust   LLR p-value:                 6.153e-06
==============================================================================
                 coef    std err          z      P>|z|      [0.025      0.975]
------------------------------------------------------------------------------
Intercept    -33.6052     15.077     -2.229      0.026     -63.156      -4.055
mpg            1.2596      0.567      2.220      0.026       0.147       2.372
hp             0.0550      0.027      2.045      0.041       0.002       0.108
==============================================================================

Possibly complete quasi-separation: A fraction 0.12 of observations can be
perfectly predicted. This might indicate that there is complete
quasi-separation. In this case some parameters will not be identified.
"""
\end{Verbatim}
\end{tcolorbox}
        
    mpg和hp的p值均小于0.05,说明这两者与am均有关联,且mpg在其中占的比重更大。

    \hypertarget{ux9644ux52a0ux98982ux5c06ux524d20ux6761ux8bb0ux5f55ux4f5cux4e3aux8badux7ec3ux6570ux636eux91cdux65b0ux5efaux7acbux4e0aux9762ux7684ux903bux8f91ux56deux5f52ux6a21ux578bux7136ux540eux7528ux540e12ux6761ux8bb0ux5f55ux4f5cux4e3aux6d4bux8bd5ux6570ux636eux518dux5bf9ux8be5ux6a21ux578bux8fdbux884cux6d4bux8bd5ux5e76ux5bf9ux7ed3ux679cux4f5cux51faux89e3ux91ca}{%
\subparagraph{附加题:(2)将前20条记录作为训练数据,重新建立上面的逻辑回归模型,然后用后12条记录
作为测试数据,再对该模型进行测试,并对结果作出解释。}\label{ux9644ux52a0ux98982ux5c06ux524d20ux6761ux8bb0ux5f55ux4f5cux4e3aux8badux7ec3ux6570ux636eux91cdux65b0ux5efaux7acbux4e0aux9762ux7684ux903bux8f91ux56deux5f52ux6a21ux578bux7136ux540eux7528ux540e12ux6761ux8bb0ux5f55ux4f5cux4e3aux6d4bux8bd5ux6570ux636eux518dux5bf9ux8be5ux6a21ux578bux8fdbux884cux6d4bux8bd5ux5e76ux5bf9ux7ed3ux679cux4f5cux51faux89e3ux91ca}}

    \begin{tcolorbox}[breakable, size=fbox, boxrule=1pt, pad at break*=1mm,colback=cellbackground, colframe=cellborder]
\prompt{In}{incolor}{34}{\boxspacing}
\begin{Verbatim}[commandchars=\\\{\}]
\PY{k+kn}{from} \PY{n+nn}{sklearn}\PY{n+nn}{.}\PY{n+nn}{model\PYZus{}selection} \PY{k+kn}{import} \PY{n}{train\PYZus{}test\PYZus{}split}
\PY{n}{train}\PY{p}{,} \PY{n}{test} \PY{o}{=} \PY{n}{train\PYZus{}test\PYZus{}split}\PY{p}{(}\PY{n}{df}\PY{p}{,} \PY{n}{test\PYZus{}size}\PY{o}{=}\PY{l+m+mi}{12}\PY{o}{/}\PY{l+m+mi}{32}\PY{p}{)}
\PY{c+c1}{\PYZsh{}\PYZsh{}\PYZsh{}Normalization}
\PY{c+c1}{\PYZsh{} Assuming same lines from your example}
\PY{n}{cols\PYZus{}to\PYZus{}norm} \PY{o}{=} \PY{l+s+s2}{\PYZdq{}}\PY{l+s+s2}{mpg}\PY{l+s+s2}{\PYZdq{}}\PY{p}{,}\PY{l+s+s2}{\PYZdq{}}\PY{l+s+s2}{hp}\PY{l+s+s2}{\PYZdq{}}
\PY{n}{train}\PY{p}{[}\PY{n}{cols\PYZus{}to\PYZus{}norm}\PY{p}{]} \PY{o}{=} \PY{n}{train}\PY{p}{[}\PY{n}{cols\PYZus{}to\PYZus{}norm}\PY{p}{]}\PY{o}{.}\PY{n}{apply}\PY{p}{(}\PY{k}{lambda} \PY{n}{x}\PY{p}{:} \PY{p}{(}\PY{n}{x} \PY{o}{\PYZhy{}} \PY{n}{x}\PY{o}{.}\PY{n}{mean}\PY{p}{(}\PY{p}{)}\PY{p}{)} \PY{o}{/} \PY{p}{(}\PY{n}{x}\PY{o}{.}\PY{n}{std}\PY{p}{(}\PY{p}{)}\PY{p}{)}\PY{p}{)}
\PY{n}{test}\PY{p}{[}\PY{n}{cols\PYZus{}to\PYZus{}norm}\PY{p}{]} \PY{o}{=} \PY{n}{test}\PY{p}{[}\PY{n}{cols\PYZus{}to\PYZus{}norm}\PY{p}{]}\PY{o}{.}\PY{n}{apply}\PY{p}{(}\PY{k}{lambda} \PY{n}{x}\PY{p}{:} \PY{p}{(}\PY{n}{x} \PY{o}{\PYZhy{}} \PY{n}{x}\PY{o}{.}\PY{n}{mean}\PY{p}{(}\PY{p}{)}\PY{p}{)} \PY{o}{/} \PY{p}{(}\PY{n}{x}\PY{o}{.}\PY{n}{std}\PY{p}{(}\PY{p}{)}\PY{p}{)}\PY{p}{)}
\end{Verbatim}
\end{tcolorbox}

    

    \begin{tcolorbox}[breakable, size=fbox, boxrule=1pt, pad at break*=1mm,colback=cellbackground, colframe=cellborder]
\prompt{In}{incolor}{31}{\boxspacing}
\begin{Verbatim}[commandchars=\\\{\}]
\PY{n}{model}\PY{o}{=}\PY{n}{smf}\PY{o}{.}\PY{n}{logit}\PY{p}{(}\PY{l+s+s1}{\PYZsq{}}\PY{l+s+s1}{am\PYZti{}mpg+hp}\PY{l+s+s1}{\PYZsq{}}\PY{p}{,}\PY{n}{data}\PY{o}{=}\PY{n}{train}\PY{p}{)}\PY{o}{.}\PY{n}{fit}\PY{p}{(}\PY{p}{)}
\PY{n}{model}\PY{o}{.}\PY{n}{summary}\PY{p}{(}\PY{p}{)}
\end{Verbatim}
\end{tcolorbox}

    \begin{Verbatim}[commandchars=\\\{\}]
Optimization terminated successfully.
         Current function value: 0.210510
         Iterations 10
    \end{Verbatim}

            \begin{tcolorbox}[breakable, size=fbox, boxrule=.5pt, pad at break*=1mm, opacityfill=0]
\prompt{Out}{outcolor}{31}{\boxspacing}
\begin{Verbatim}[commandchars=\\\{\}]
<class 'statsmodels.iolib.summary.Summary'>
"""
                           Logit Regression Results
==============================================================================
Dep. Variable:                     am   No. Observations:                   20
Model:                          Logit   Df Residuals:                       17
Method:                           MLE   Df Model:                            2
Date:                Tue, 19 May 2020   Pseudo R-squ.:                  0.6941
Time:                        20:04:26   Log-Likelihood:                -4.2102
converged:                       True   LL-Null:                       -13.763
Covariance Type:            nonrobust   LLR p-value:                 7.102e-05
==============================================================================
                 coef    std err          z      P>|z|      [0.025      0.975]
------------------------------------------------------------------------------
Intercept      1.8711      1.363      1.373      0.170      -0.800       4.542
mpg           11.8868      6.970      1.705      0.088      -1.774      25.547
hp             5.6484      3.435      1.644      0.100      -1.084      12.381
==============================================================================

Possibly complete quasi-separation: A fraction 0.30 of observations can be
perfectly predicted. This might indicate that there is complete
quasi-separation. In this case some parameters will not be identified.
"""
\end{Verbatim}
\end{tcolorbox}
        
    \begin{tcolorbox}[breakable, size=fbox, boxrule=1pt, pad at break*=1mm,colback=cellbackground, colframe=cellborder]
\prompt{In}{incolor}{32}{\boxspacing}
\begin{Verbatim}[commandchars=\\\{\}]
\PY{n}{trainingRes}\PY{o}{=}\PY{n}{pd}\PY{o}{.}\PY{n}{DataFrame}\PY{p}{(}\PY{n}{model}\PY{o}{.}\PY{n}{pred\PYZus{}table}\PY{p}{(}\PY{p}{)}\PY{p}{)}
\PY{n}{trainingRes}\PY{o}{.}\PY{n}{columns}\PY{o}{=}\PY{p}{[}\PY{l+s+s2}{\PYZdq{}}\PY{l+s+s2}{Predicted Outcome 0}\PY{l+s+s2}{\PYZdq{}}\PY{p}{,}\PY{l+s+s2}{\PYZdq{}}\PY{l+s+s2}{Predicted Outcome 1}\PY{l+s+s2}{\PYZdq{}}\PY{p}{]}
\PY{n}{trainingRes}\PY{o}{=}\PY{n}{trainingRes}\PY{o}{.}\PY{n}{rename}\PY{p}{(}\PY{n}{index}\PY{o}{=}\PY{p}{\PYZob{}}\PY{l+m+mi}{0}\PY{p}{:}\PY{l+s+s2}{\PYZdq{}}\PY{l+s+s2}{Actual Outcome 0}\PY{l+s+s2}{\PYZdq{}}\PY{p}{,} \PY{l+m+mi}{1}\PY{p}{:}\PY{l+s+s2}{\PYZdq{}}\PY{l+s+s2}{Actually Outcome 1}\PY{l+s+s2}{\PYZdq{}}\PY{p}{\PYZcb{}}\PY{p}{)}
\PY{n}{trainingRes}
\PY{n+nb}{print}\PY{p}{(}\PY{l+s+s2}{\PYZdq{}}\PY{l+s+s2}{The accuracy for the train data is :}\PY{l+s+s2}{\PYZdq{}}\PY{p}{,}\PY{p}{(}\PY{n}{trainingRes}\PY{o}{.}\PY{n}{iloc}\PY{p}{[}\PY{l+m+mi}{1}\PY{p}{,}\PY{l+m+mi}{1}\PY{p}{]}\PY{o}{+}\PY{n}{trainingRes}\PY{o}{.}\PY{n}{iloc}\PY{p}{[}\PY{l+m+mi}{0}\PY{p}{,}\PY{l+m+mi}{0}\PY{p}{]}\PY{p}{)}\PY{o}{/}\PY{l+m+mi}{20}\PY{p}{)}
\end{Verbatim}
\end{tcolorbox}

            \begin{tcolorbox}[breakable, size=fbox, boxrule=.5pt, pad at break*=1mm, opacityfill=0]
\prompt{Out}{outcolor}{32}{\boxspacing}
\begin{Verbatim}[commandchars=\\\{\}]
                    Predicted Outcome 0  Predicted Outcome 1
Actual Outcome 0                    8.0                  1.0
Actually Outcome 1                  1.0                 10.0
\end{Verbatim}
\end{tcolorbox}
        
    \begin{Verbatim}[commandchars=\\\{\}]
The accuracy for the train data is : 0.9
    \end{Verbatim}

    \begin{tcolorbox}[breakable, size=fbox, boxrule=1pt, pad at break*=1mm,colback=cellbackground, colframe=cellborder]
\prompt{In}{incolor}{33}{\boxspacing}
\begin{Verbatim}[commandchars=\\\{\}]
\PY{n}{pred\PYZus{}values} \PY{o}{=} \PY{n}{model}\PY{o}{.}\PY{n}{predict}\PY{p}{(}\PY{n}{test}\PY{p}{)}
\PY{n}{bins}\PY{o}{=}\PY{n}{np}\PY{o}{.}\PY{n}{array}\PY{p}{(}\PY{p}{[}\PY{l+m+mi}{0}\PY{p}{,}\PY{l+m+mf}{0.5}\PY{p}{,}\PY{l+m+mi}{1}\PY{p}{]}\PY{p}{)}
\PY{n}{cm} \PY{o}{=} \PY{n}{np}\PY{o}{.}\PY{n}{histogram2d}\PY{p}{(}\PY{n}{test}\PY{p}{[}\PY{l+s+s1}{\PYZsq{}}\PY{l+s+s1}{am}\PY{l+s+s1}{\PYZsq{}}\PY{p}{]}\PY{p}{,} \PY{n}{pred\PYZus{}values}\PY{p}{,} \PY{n}{bins}\PY{o}{=}\PY{n}{bins}\PY{p}{)}\PY{p}{[}\PY{l+m+mi}{0}\PY{p}{]}
\PY{n}{accuracy} \PY{o}{=} \PY{p}{(}\PY{n}{cm}\PY{p}{[}\PY{l+m+mi}{0}\PY{p}{,}\PY{l+m+mi}{0}\PY{p}{]}\PY{o}{+}\PY{n}{cm}\PY{p}{[}\PY{l+m+mi}{1}\PY{p}{,}\PY{l+m+mi}{1}\PY{p}{]}\PY{p}{)}\PY{o}{/}\PY{n}{cm}\PY{o}{.}\PY{n}{sum}\PY{p}{(}\PY{p}{)}
\PY{n+nb}{print}\PY{p}{(}\PY{l+s+s2}{\PYZdq{}}\PY{l+s+s2}{The prediction accuracy for the test data is :}\PY{l+s+s2}{\PYZdq{}}\PY{p}{,} \PY{n}{accuracy}\PY{p}{)}

\PY{n}{testRes}\PY{o}{=}\PY{n}{pd}\PY{o}{.}\PY{n}{DataFrame}\PY{p}{(}\PY{n}{cm}\PY{p}{)}
\PY{n}{testRes}\PY{o}{.}\PY{n}{columns}\PY{o}{=}\PY{p}{[}\PY{l+s+s2}{\PYZdq{}}\PY{l+s+s2}{Predicted Outcome 0}\PY{l+s+s2}{\PYZdq{}}\PY{p}{,}\PY{l+s+s2}{\PYZdq{}}\PY{l+s+s2}{Predicted Outcome 1}\PY{l+s+s2}{\PYZdq{}}\PY{p}{]}
\PY{n}{testRes}\PY{o}{=}\PY{n}{testRes}\PY{o}{.}\PY{n}{rename}\PY{p}{(}\PY{n}{index}\PY{o}{=}\PY{p}{\PYZob{}}\PY{l+m+mi}{0}\PY{p}{:}\PY{l+s+s2}{\PYZdq{}}\PY{l+s+s2}{Actual Outcome 0}\PY{l+s+s2}{\PYZdq{}}\PY{p}{,} \PY{l+m+mi}{1}\PY{p}{:}\PY{l+s+s2}{\PYZdq{}}\PY{l+s+s2}{Actually Outcome 1}\PY{l+s+s2}{\PYZdq{}}\PY{p}{\PYZcb{}}\PY{p}{)}
\PY{n}{testRes}
\end{Verbatim}
\end{tcolorbox}

    \begin{Verbatim}[commandchars=\\\{\}]
The prediction accuracy for the test data is : 0.6666666666666666
    \end{Verbatim}

            \begin{tcolorbox}[breakable, size=fbox, boxrule=.5pt, pad at break*=1mm, opacityfill=0]
\prompt{Out}{outcolor}{33}{\boxspacing}
\begin{Verbatim}[commandchars=\\\{\}]
                    Predicted Outcome 0  Predicted Outcome 1
Actual Outcome 0                    6.0                  4.0
Actually Outcome 1                  0.0                  2.0
\end{Verbatim}
\end{tcolorbox}
        
    从拟合的模型来看,由于训练集较之前规模减小\((32->20)\),因此模型精确度有所下降,三个p值
    均较大,可能存在一定的欠拟合情况。但从结果来看模型效果还不错,训练集的正确率为0.8,测试
    集的准确率有0.917。


    % Add a bibliography block to the postdoc
    
    
    
\end{document}

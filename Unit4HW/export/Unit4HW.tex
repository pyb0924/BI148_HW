\documentclass[11pt]{article}
\usepackage{fontspec, xunicode, xltxtra}
    \setmainfont{Microsoft YaHei}
    \usepackage[breakable]{tcolorbox}
    \usepackage{parskip} % Stop auto-indenting (to mimic markdown behaviour)
    
    \usepackage{iftex}
    \ifPDFTeX
    	\usepackage[T1]{fontenc}
    	\usepackage{mathpazo}
    \else
    	\usepackage{fontspec}
    \fi

    % Basic figure setup, for now with no caption control since it's done
    % automatically by Pandoc (which extracts ![](path) syntax from Markdown).
    \usepackage{graphicx}
    % Maintain compatibility with old templates. Remove in nbconvert 6.0
    \let\Oldincludegraphics\includegraphics
    % Ensure that by default, figures have no caption (until we provide a
    % proper Figure object with a Caption API and a way to capture that
    % in the conversion process - todo).
    \usepackage{caption}
    \DeclareCaptionFormat{nocaption}{}
    \captionsetup{format=nocaption,aboveskip=0pt,belowskip=0pt}

    \usepackage[Export]{adjustbox} % Used to constrain images to a maximum size
    \adjustboxset{max size={0.9\linewidth}{0.9\paperheight}}
    \usepackage{float}
    \floatplacement{figure}{H} % forces figures to be placed at the correct location
    \usepackage{xcolor} % Allow colors to be defined
    \usepackage{enumerate} % Needed for markdown enumerations to work
    \usepackage{geometry} % Used to adjust the document margins
    \usepackage{amsmath} % Equations
    \usepackage{amssymb} % Equations
    \usepackage{textcomp} % defines textquotesingle
    % Hack from http://tex.stackexchange.com/a/47451/13684:
    \AtBeginDocument{%
        \def\PYZsq{\textquotesingle}% Upright quotes in Pygmentized code
    }
    \usepackage{upquote} % Upright quotes for verbatim code
    \usepackage{eurosym} % defines \euro
    \usepackage[mathletters]{ucs} % Extended unicode (utf-8) support
    \usepackage{fancyvrb} % verbatim replacement that allows latex
    \usepackage{grffile} % extends the file name processing of package graphics 
                         % to support a larger range
    \makeatletter % fix for grffile with XeLaTeX
    \def\Gread@@xetex#1{%
      \IfFileExists{"\Gin@base".bb}%
      {\Gread@eps{\Gin@base.bb}}%
      {\Gread@@xetex@aux#1}%
    }
    \makeatother

    % The hyperref package gives us a pdf with properly built
    % internal navigation ('pdf bookmarks' for the table of contents,
    % internal cross-reference links, web links for URLs, etc.)
    \usepackage{hyperref}
    % The default LaTeX title has an obnoxious amount of whitespace. By default,
    % titling removes some of it. It also provides customization options.
    \usepackage{titling}
    \usepackage{longtable} % longtable support required by pandoc >1.10
    \usepackage{booktabs}  % table support for pandoc > 1.12.2
    \usepackage[inline]{enumitem} % IRkernel/repr support (it uses the enumerate* environment)
    \usepackage[normalem]{ulem} % ulem is needed to support strikethroughs (\sout)
                                % normalem makes italics be italics, not underlines
    \usepackage{mathrsfs}
    

    
    % Colors for the hyperref package
    \definecolor{urlcolor}{rgb}{0,.145,.698}
    \definecolor{linkcolor}{rgb}{.71,0.21,0.01}
    \definecolor{citecolor}{rgb}{.12,.54,.11}

    % ANSI colors
    \definecolor{ansi-black}{HTML}{3E424D}
    \definecolor{ansi-black-intense}{HTML}{282C36}
    \definecolor{ansi-red}{HTML}{E75C58}
    \definecolor{ansi-red-intense}{HTML}{B22B31}
    \definecolor{ansi-green}{HTML}{00A250}
    \definecolor{ansi-green-intense}{HTML}{007427}
    \definecolor{ansi-yellow}{HTML}{DDB62B}
    \definecolor{ansi-yellow-intense}{HTML}{B27D12}
    \definecolor{ansi-blue}{HTML}{208FFB}
    \definecolor{ansi-blue-intense}{HTML}{0065CA}
    \definecolor{ansi-magenta}{HTML}{D160C4}
    \definecolor{ansi-magenta-intense}{HTML}{A03196}
    \definecolor{ansi-cyan}{HTML}{60C6C8}
    \definecolor{ansi-cyan-intense}{HTML}{258F8F}
    \definecolor{ansi-white}{HTML}{C5C1B4}
    \definecolor{ansi-white-intense}{HTML}{A1A6B2}
    \definecolor{ansi-default-inverse-fg}{HTML}{FFFFFF}
    \definecolor{ansi-default-inverse-bg}{HTML}{000000}

    % commands and environments needed by pandoc snippets
    % extracted from the output of `pandoc -s`
    \providecommand{\tightlist}{%
      \setlength{\itemsep}{0pt}\setlength{\parskip}{0pt}}
    \DefineVerbatimEnvironment{Highlighting}{Verbatim}{commandchars=\\\{\}}
    % Add ',fontsize=\small' for more characters per line
    \newenvironment{Shaded}{}{}
    \newcommand{\KeywordTok}[1]{\textcolor[rgb]{0.00,0.44,0.13}{\textbf{{#1}}}}
    \newcommand{\DataTypeTok}[1]{\textcolor[rgb]{0.56,0.13,0.00}{{#1}}}
    \newcommand{\DecValTok}[1]{\textcolor[rgb]{0.25,0.63,0.44}{{#1}}}
    \newcommand{\BaseNTok}[1]{\textcolor[rgb]{0.25,0.63,0.44}{{#1}}}
    \newcommand{\FloatTok}[1]{\textcolor[rgb]{0.25,0.63,0.44}{{#1}}}
    \newcommand{\CharTok}[1]{\textcolor[rgb]{0.25,0.44,0.63}{{#1}}}
    \newcommand{\StringTok}[1]{\textcolor[rgb]{0.25,0.44,0.63}{{#1}}}
    \newcommand{\CommentTok}[1]{\textcolor[rgb]{0.38,0.63,0.69}{\textit{{#1}}}}
    \newcommand{\OtherTok}[1]{\textcolor[rgb]{0.00,0.44,0.13}{{#1}}}
    \newcommand{\AlertTok}[1]{\textcolor[rgb]{1.00,0.00,0.00}{\textbf{{#1}}}}
    \newcommand{\FunctionTok}[1]{\textcolor[rgb]{0.02,0.16,0.49}{{#1}}}
    \newcommand{\RegionMarkerTok}[1]{{#1}}
    \newcommand{\ErrorTok}[1]{\textcolor[rgb]{1.00,0.00,0.00}{\textbf{{#1}}}}
    \newcommand{\NormalTok}[1]{{#1}}
    
    % Additional commands for more recent versions of Pandoc
    \newcommand{\ConstantTok}[1]{\textcolor[rgb]{0.53,0.00,0.00}{{#1}}}
    \newcommand{\SpecialCharTok}[1]{\textcolor[rgb]{0.25,0.44,0.63}{{#1}}}
    \newcommand{\VerbatimStringTok}[1]{\textcolor[rgb]{0.25,0.44,0.63}{{#1}}}
    \newcommand{\SpecialStringTok}[1]{\textcolor[rgb]{0.73,0.40,0.53}{{#1}}}
    \newcommand{\ImportTok}[1]{{#1}}
    \newcommand{\DocumentationTok}[1]{\textcolor[rgb]{0.73,0.13,0.13}{\textit{{#1}}}}
    \newcommand{\AnnotationTok}[1]{\textcolor[rgb]{0.38,0.63,0.69}{\textbf{\textit{{#1}}}}}
    \newcommand{\CommentVarTok}[1]{\textcolor[rgb]{0.38,0.63,0.69}{\textbf{\textit{{#1}}}}}
    \newcommand{\VariableTok}[1]{\textcolor[rgb]{0.10,0.09,0.49}{{#1}}}
    \newcommand{\ControlFlowTok}[1]{\textcolor[rgb]{0.00,0.44,0.13}{\textbf{{#1}}}}
    \newcommand{\OperatorTok}[1]{\textcolor[rgb]{0.40,0.40,0.40}{{#1}}}
    \newcommand{\BuiltInTok}[1]{{#1}}
    \newcommand{\ExtensionTok}[1]{{#1}}
    \newcommand{\PreprocessorTok}[1]{\textcolor[rgb]{0.74,0.48,0.00}{{#1}}}
    \newcommand{\AttributeTok}[1]{\textcolor[rgb]{0.49,0.56,0.16}{{#1}}}
    \newcommand{\InformationTok}[1]{\textcolor[rgb]{0.38,0.63,0.69}{\textbf{\textit{{#1}}}}}
    \newcommand{\WarningTok}[1]{\textcolor[rgb]{0.38,0.63,0.69}{\textbf{\textit{{#1}}}}}
    
    
    % Define a nice break command that doesn't care if a line doesn't already
    % exist.
    \def\br{\hspace*{\fill} \\* }
    % Math Jax compatibility definitions
    \def\gt{>}
    \def\lt{<}
    \let\Oldtex\TeX
    \let\Oldlatex\LaTeX
    \renewcommand{\TeX}{\textrm{\Oldtex}}
    \renewcommand{\LaTeX}{\textrm{\Oldlatex}}
    % Document parameters
    % Document title
    \title{Unit4HW}
    
    
    
    
    
% Pygments definitions
\makeatletter
\def\PY@reset{\let\PY@it=\relax \let\PY@bf=\relax%
    \let\PY@ul=\relax \let\PY@tc=\relax%
    \let\PY@bc=\relax \let\PY@ff=\relax}
\def\PY@tok#1{\csname PY@tok@#1\endcsname}
\def\PY@toks#1+{\ifx\relax#1\empty\else%
    \PY@tok{#1}\expandafter\PY@toks\fi}
\def\PY@do#1{\PY@bc{\PY@tc{\PY@ul{%
    \PY@it{\PY@bf{\PY@ff{#1}}}}}}}
\def\PY#1#2{\PY@reset\PY@toks#1+\relax+\PY@do{#2}}

\expandafter\def\csname PY@tok@w\endcsname{\def\PY@tc##1{\textcolor[rgb]{0.73,0.73,0.73}{##1}}}
\expandafter\def\csname PY@tok@c\endcsname{\let\PY@it=\textit\def\PY@tc##1{\textcolor[rgb]{0.25,0.50,0.50}{##1}}}
\expandafter\def\csname PY@tok@cp\endcsname{\def\PY@tc##1{\textcolor[rgb]{0.74,0.48,0.00}{##1}}}
\expandafter\def\csname PY@tok@k\endcsname{\let\PY@bf=\textbf\def\PY@tc##1{\textcolor[rgb]{0.00,0.50,0.00}{##1}}}
\expandafter\def\csname PY@tok@kp\endcsname{\def\PY@tc##1{\textcolor[rgb]{0.00,0.50,0.00}{##1}}}
\expandafter\def\csname PY@tok@kt\endcsname{\def\PY@tc##1{\textcolor[rgb]{0.69,0.00,0.25}{##1}}}
\expandafter\def\csname PY@tok@o\endcsname{\def\PY@tc##1{\textcolor[rgb]{0.40,0.40,0.40}{##1}}}
\expandafter\def\csname PY@tok@ow\endcsname{\let\PY@bf=\textbf\def\PY@tc##1{\textcolor[rgb]{0.67,0.13,1.00}{##1}}}
\expandafter\def\csname PY@tok@nb\endcsname{\def\PY@tc##1{\textcolor[rgb]{0.00,0.50,0.00}{##1}}}
\expandafter\def\csname PY@tok@nf\endcsname{\def\PY@tc##1{\textcolor[rgb]{0.00,0.00,1.00}{##1}}}
\expandafter\def\csname PY@tok@nc\endcsname{\let\PY@bf=\textbf\def\PY@tc##1{\textcolor[rgb]{0.00,0.00,1.00}{##1}}}
\expandafter\def\csname PY@tok@nn\endcsname{\let\PY@bf=\textbf\def\PY@tc##1{\textcolor[rgb]{0.00,0.00,1.00}{##1}}}
\expandafter\def\csname PY@tok@ne\endcsname{\let\PY@bf=\textbf\def\PY@tc##1{\textcolor[rgb]{0.82,0.25,0.23}{##1}}}
\expandafter\def\csname PY@tok@nv\endcsname{\def\PY@tc##1{\textcolor[rgb]{0.10,0.09,0.49}{##1}}}
\expandafter\def\csname PY@tok@no\endcsname{\def\PY@tc##1{\textcolor[rgb]{0.53,0.00,0.00}{##1}}}
\expandafter\def\csname PY@tok@nl\endcsname{\def\PY@tc##1{\textcolor[rgb]{0.63,0.63,0.00}{##1}}}
\expandafter\def\csname PY@tok@ni\endcsname{\let\PY@bf=\textbf\def\PY@tc##1{\textcolor[rgb]{0.60,0.60,0.60}{##1}}}
\expandafter\def\csname PY@tok@na\endcsname{\def\PY@tc##1{\textcolor[rgb]{0.49,0.56,0.16}{##1}}}
\expandafter\def\csname PY@tok@nt\endcsname{\let\PY@bf=\textbf\def\PY@tc##1{\textcolor[rgb]{0.00,0.50,0.00}{##1}}}
\expandafter\def\csname PY@tok@nd\endcsname{\def\PY@tc##1{\textcolor[rgb]{0.67,0.13,1.00}{##1}}}
\expandafter\def\csname PY@tok@s\endcsname{\def\PY@tc##1{\textcolor[rgb]{0.73,0.13,0.13}{##1}}}
\expandafter\def\csname PY@tok@sd\endcsname{\let\PY@it=\textit\def\PY@tc##1{\textcolor[rgb]{0.73,0.13,0.13}{##1}}}
\expandafter\def\csname PY@tok@si\endcsname{\let\PY@bf=\textbf\def\PY@tc##1{\textcolor[rgb]{0.73,0.40,0.53}{##1}}}
\expandafter\def\csname PY@tok@se\endcsname{\let\PY@bf=\textbf\def\PY@tc##1{\textcolor[rgb]{0.73,0.40,0.13}{##1}}}
\expandafter\def\csname PY@tok@sr\endcsname{\def\PY@tc##1{\textcolor[rgb]{0.73,0.40,0.53}{##1}}}
\expandafter\def\csname PY@tok@ss\endcsname{\def\PY@tc##1{\textcolor[rgb]{0.10,0.09,0.49}{##1}}}
\expandafter\def\csname PY@tok@sx\endcsname{\def\PY@tc##1{\textcolor[rgb]{0.00,0.50,0.00}{##1}}}
\expandafter\def\csname PY@tok@m\endcsname{\def\PY@tc##1{\textcolor[rgb]{0.40,0.40,0.40}{##1}}}
\expandafter\def\csname PY@tok@gh\endcsname{\let\PY@bf=\textbf\def\PY@tc##1{\textcolor[rgb]{0.00,0.00,0.50}{##1}}}
\expandafter\def\csname PY@tok@gu\endcsname{\let\PY@bf=\textbf\def\PY@tc##1{\textcolor[rgb]{0.50,0.00,0.50}{##1}}}
\expandafter\def\csname PY@tok@gd\endcsname{\def\PY@tc##1{\textcolor[rgb]{0.63,0.00,0.00}{##1}}}
\expandafter\def\csname PY@tok@gi\endcsname{\def\PY@tc##1{\textcolor[rgb]{0.00,0.63,0.00}{##1}}}
\expandafter\def\csname PY@tok@gr\endcsname{\def\PY@tc##1{\textcolor[rgb]{1.00,0.00,0.00}{##1}}}
\expandafter\def\csname PY@tok@ge\endcsname{\let\PY@it=\textit}
\expandafter\def\csname PY@tok@gs\endcsname{\let\PY@bf=\textbf}
\expandafter\def\csname PY@tok@gp\endcsname{\let\PY@bf=\textbf\def\PY@tc##1{\textcolor[rgb]{0.00,0.00,0.50}{##1}}}
\expandafter\def\csname PY@tok@go\endcsname{\def\PY@tc##1{\textcolor[rgb]{0.53,0.53,0.53}{##1}}}
\expandafter\def\csname PY@tok@gt\endcsname{\def\PY@tc##1{\textcolor[rgb]{0.00,0.27,0.87}{##1}}}
\expandafter\def\csname PY@tok@err\endcsname{\def\PY@bc##1{\setlength{\fboxsep}{0pt}\fcolorbox[rgb]{1.00,0.00,0.00}{1,1,1}{\strut ##1}}}
\expandafter\def\csname PY@tok@kc\endcsname{\let\PY@bf=\textbf\def\PY@tc##1{\textcolor[rgb]{0.00,0.50,0.00}{##1}}}
\expandafter\def\csname PY@tok@kd\endcsname{\let\PY@bf=\textbf\def\PY@tc##1{\textcolor[rgb]{0.00,0.50,0.00}{##1}}}
\expandafter\def\csname PY@tok@kn\endcsname{\let\PY@bf=\textbf\def\PY@tc##1{\textcolor[rgb]{0.00,0.50,0.00}{##1}}}
\expandafter\def\csname PY@tok@kr\endcsname{\let\PY@bf=\textbf\def\PY@tc##1{\textcolor[rgb]{0.00,0.50,0.00}{##1}}}
\expandafter\def\csname PY@tok@bp\endcsname{\def\PY@tc##1{\textcolor[rgb]{0.00,0.50,0.00}{##1}}}
\expandafter\def\csname PY@tok@fm\endcsname{\def\PY@tc##1{\textcolor[rgb]{0.00,0.00,1.00}{##1}}}
\expandafter\def\csname PY@tok@vc\endcsname{\def\PY@tc##1{\textcolor[rgb]{0.10,0.09,0.49}{##1}}}
\expandafter\def\csname PY@tok@vg\endcsname{\def\PY@tc##1{\textcolor[rgb]{0.10,0.09,0.49}{##1}}}
\expandafter\def\csname PY@tok@vi\endcsname{\def\PY@tc##1{\textcolor[rgb]{0.10,0.09,0.49}{##1}}}
\expandafter\def\csname PY@tok@vm\endcsname{\def\PY@tc##1{\textcolor[rgb]{0.10,0.09,0.49}{##1}}}
\expandafter\def\csname PY@tok@sa\endcsname{\def\PY@tc##1{\textcolor[rgb]{0.73,0.13,0.13}{##1}}}
\expandafter\def\csname PY@tok@sb\endcsname{\def\PY@tc##1{\textcolor[rgb]{0.73,0.13,0.13}{##1}}}
\expandafter\def\csname PY@tok@sc\endcsname{\def\PY@tc##1{\textcolor[rgb]{0.73,0.13,0.13}{##1}}}
\expandafter\def\csname PY@tok@dl\endcsname{\def\PY@tc##1{\textcolor[rgb]{0.73,0.13,0.13}{##1}}}
\expandafter\def\csname PY@tok@s2\endcsname{\def\PY@tc##1{\textcolor[rgb]{0.73,0.13,0.13}{##1}}}
\expandafter\def\csname PY@tok@sh\endcsname{\def\PY@tc##1{\textcolor[rgb]{0.73,0.13,0.13}{##1}}}
\expandafter\def\csname PY@tok@s1\endcsname{\def\PY@tc##1{\textcolor[rgb]{0.73,0.13,0.13}{##1}}}
\expandafter\def\csname PY@tok@mb\endcsname{\def\PY@tc##1{\textcolor[rgb]{0.40,0.40,0.40}{##1}}}
\expandafter\def\csname PY@tok@mf\endcsname{\def\PY@tc##1{\textcolor[rgb]{0.40,0.40,0.40}{##1}}}
\expandafter\def\csname PY@tok@mh\endcsname{\def\PY@tc##1{\textcolor[rgb]{0.40,0.40,0.40}{##1}}}
\expandafter\def\csname PY@tok@mi\endcsname{\def\PY@tc##1{\textcolor[rgb]{0.40,0.40,0.40}{##1}}}
\expandafter\def\csname PY@tok@il\endcsname{\def\PY@tc##1{\textcolor[rgb]{0.40,0.40,0.40}{##1}}}
\expandafter\def\csname PY@tok@mo\endcsname{\def\PY@tc##1{\textcolor[rgb]{0.40,0.40,0.40}{##1}}}
\expandafter\def\csname PY@tok@ch\endcsname{\let\PY@it=\textit\def\PY@tc##1{\textcolor[rgb]{0.25,0.50,0.50}{##1}}}
\expandafter\def\csname PY@tok@cm\endcsname{\let\PY@it=\textit\def\PY@tc##1{\textcolor[rgb]{0.25,0.50,0.50}{##1}}}
\expandafter\def\csname PY@tok@cpf\endcsname{\let\PY@it=\textit\def\PY@tc##1{\textcolor[rgb]{0.25,0.50,0.50}{##1}}}
\expandafter\def\csname PY@tok@c1\endcsname{\let\PY@it=\textit\def\PY@tc##1{\textcolor[rgb]{0.25,0.50,0.50}{##1}}}
\expandafter\def\csname PY@tok@cs\endcsname{\let\PY@it=\textit\def\PY@tc##1{\textcolor[rgb]{0.25,0.50,0.50}{##1}}}

\def\PYZbs{\char`\\}
\def\PYZus{\char`\_}
\def\PYZob{\char`\{}
\def\PYZcb{\char`\}}
\def\PYZca{\char`\^}
\def\PYZam{\char`\&}
\def\PYZlt{\char`\<}
\def\PYZgt{\char`\>}
\def\PYZsh{\char`\#}
\def\PYZpc{\char`\%}
\def\PYZdl{\char`\$}
\def\PYZhy{\char`\-}
\def\PYZsq{\char`\'}
\def\PYZdq{\char`\"}
\def\PYZti{\char`\~}
% for compatibility with earlier versions
\def\PYZat{@}
\def\PYZlb{[}
\def\PYZrb{]}
\makeatother


    % For linebreaks inside Verbatim environment from package fancyvrb. 
    \makeatletter
        \newbox\Wrappedcontinuationbox 
        \newbox\Wrappedvisiblespacebox 
        \newcommand*\Wrappedvisiblespace {\textcolor{red}{\textvisiblespace}} 
        \newcommand*\Wrappedcontinuationsymbol {\textcolor{red}{\llap{\tiny$\m@th\hookrightarrow$}}} 
        \newcommand*\Wrappedcontinuationindent {3ex } 
        \newcommand*\Wrappedafterbreak {\kern\Wrappedcontinuationindent\copy\Wrappedcontinuationbox} 
        % Take advantage of the already applied Pygments mark-up to insert 
        % potential linebreaks for TeX processing. 
        %        {, <, #, %, $, ' and ": go to next line. 
        %        _, }, ^, &, >, - and ~: stay at end of broken line. 
        % Use of \textquotesingle for straight quote. 
        \newcommand*\Wrappedbreaksatspecials {% 
            \def\PYGZus{\discretionary{\char`\_}{\Wrappedafterbreak}{\char`\_}}% 
            \def\PYGZob{\discretionary{}{\Wrappedafterbreak\char`\{}{\char`\{}}% 
            \def\PYGZcb{\discretionary{\char`\}}{\Wrappedafterbreak}{\char`\}}}% 
            \def\PYGZca{\discretionary{\char`\^}{\Wrappedafterbreak}{\char`\^}}% 
            \def\PYGZam{\discretionary{\char`\&}{\Wrappedafterbreak}{\char`\&}}% 
            \def\PYGZlt{\discretionary{}{\Wrappedafterbreak\char`\<}{\char`\<}}% 
            \def\PYGZgt{\discretionary{\char`\>}{\Wrappedafterbreak}{\char`\>}}% 
            \def\PYGZsh{\discretionary{}{\Wrappedafterbreak\char`\#}{\char`\#}}% 
            \def\PYGZpc{\discretionary{}{\Wrappedafterbreak\char`\%}{\char`\%}}% 
            \def\PYGZdl{\discretionary{}{\Wrappedafterbreak\char`\$}{\char`\$}}% 
            \def\PYGZhy{\discretionary{\char`\-}{\Wrappedafterbreak}{\char`\-}}% 
            \def\PYGZsq{\discretionary{}{\Wrappedafterbreak\textquotesingle}{\textquotesingle}}% 
            \def\PYGZdq{\discretionary{}{\Wrappedafterbreak\char`\"}{\char`\"}}% 
            \def\PYGZti{\discretionary{\char`\~}{\Wrappedafterbreak}{\char`\~}}% 
        } 
        % Some characters . , ; ? ! / are not pygmentized. 
        % This macro makes them "active" and they will insert potential linebreaks 
        \newcommand*\Wrappedbreaksatpunct {% 
            \lccode`\~`\.\lowercase{\def~}{\discretionary{\hbox{\char`\.}}{\Wrappedafterbreak}{\hbox{\char`\.}}}% 
            \lccode`\~`\,\lowercase{\def~}{\discretionary{\hbox{\char`\,}}{\Wrappedafterbreak}{\hbox{\char`\,}}}% 
            \lccode`\~`\;\lowercase{\def~}{\discretionary{\hbox{\char`\;}}{\Wrappedafterbreak}{\hbox{\char`\;}}}% 
            \lccode`\~`\:\lowercase{\def~}{\discretionary{\hbox{\char`\:}}{\Wrappedafterbreak}{\hbox{\char`\:}}}% 
            \lccode`\~`\?\lowercase{\def~}{\discretionary{\hbox{\char`\?}}{\Wrappedafterbreak}{\hbox{\char`\?}}}% 
            \lccode`\~`\!\lowercase{\def~}{\discretionary{\hbox{\char`\!}}{\Wrappedafterbreak}{\hbox{\char`\!}}}% 
            \lccode`\~`\/\lowercase{\def~}{\discretionary{\hbox{\char`\/}}{\Wrappedafterbreak}{\hbox{\char`\/}}}% 
            \catcode`\.\active
            \catcode`\,\active 
            \catcode`\;\active
            \catcode`\:\active
            \catcode`\?\active
            \catcode`\!\active
            \catcode`\/\active 
            \lccode`\~`\~ 	
        }
    \makeatother

    \let\OriginalVerbatim=\Verbatim
    \makeatletter
    \renewcommand{\Verbatim}[1][1]{%
        %\parskip\z@skip
        \sbox\Wrappedcontinuationbox {\Wrappedcontinuationsymbol}%
        \sbox\Wrappedvisiblespacebox {\FV@SetupFont\Wrappedvisiblespace}%
        \def\FancyVerbFormatLine ##1{\hsize\linewidth
            \vtop{\raggedright\hyphenpenalty\z@\exhyphenpenalty\z@
                \doublehyphendemerits\z@\finalhyphendemerits\z@
                \strut ##1\strut}%
        }%
        % If the linebreak is at a space, the latter will be displayed as visible
        % space at end of first line, and a continuation symbol starts next line.
        % Stretch/shrink are however usually zero for typewriter font.
        \def\FV@Space {%
            \nobreak\hskip\z@ plus\fontdimen3\font minus\fontdimen4\font
            \discretionary{\copy\Wrappedvisiblespacebox}{\Wrappedafterbreak}
            {\kern\fontdimen2\font}%
        }%
        
        % Allow breaks at special characters using \PYG... macros.
        \Wrappedbreaksatspecials
        % Breaks at punctuation characters . , ; ? ! and / need catcode=\active 	
        \OriginalVerbatim[#1,codes*=\Wrappedbreaksatpunct]%
    }
    \makeatother

    % Exact colors from NB
    \definecolor{incolor}{HTML}{303F9F}
    \definecolor{outcolor}{HTML}{D84315}
    \definecolor{cellborder}{HTML}{CFCFCF}
    \definecolor{cellbackground}{HTML}{F7F7F7}
    
    % prompt
    \makeatletter
    \newcommand{\boxspacing}{\kern\kvtcb@left@rule\kern\kvtcb@boxsep}
    \makeatother
    \newcommand{\prompt}[4]{
        \ttfamily\llap{{\color{#2}[#3]:\hspace{3pt}#4}}\vspace{-\baselineskip}
    }
    

    
    % Prevent overflowing lines due to hard-to-break entities
    \sloppy 
    % Setup hyperref package
    \hypersetup{
      breaklinks=true,  % so long urls are correctly broken across lines
      colorlinks=true,
      urlcolor=urlcolor,
      linkcolor=linkcolor,
      citecolor=citecolor,
      }
    % Slightly bigger margins than the latex defaults
    
    \geometry{verbose,tmargin=1in,bmargin=1in,lmargin=1in,rmargin=1in}
    
    

\begin{document}
    
    \maketitle
    
    

    
    \hypertarget{unit4-ux4f5cux4e1a}{%
\section{Unit4 作业}\label{unit4-ux4f5cux4e1a}}

    \begin{tcolorbox}[breakable, size=fbox, boxrule=1pt, pad at break*=1mm,colback=cellbackground, colframe=cellborder]
\prompt{In}{incolor}{1}{\boxspacing}
\begin{Verbatim}[commandchars=\\\{\}]
\PY{k+kn}{import} \PY{n+nn}{scipy}\PY{n+nn}{.}\PY{n+nn}{stats} \PY{k}{as} \PY{n+nn}{stats}
\PY{k+kn}{import} \PY{n+nn}{numpy} \PY{k}{as} \PY{n+nn}{np}
\PY{k+kn}{import} \PY{n+nn}{statsmodels}\PY{n+nn}{.}\PY{n+nn}{stats}\PY{n+nn}{.}\PY{n+nn}{proportion} \PY{k}{as} \PY{n+nn}{proportion}
\PY{k+kn}{from} \PY{n+nn}{IPython}\PY{n+nn}{.}\PY{n+nn}{core}\PY{n+nn}{.}\PY{n+nn}{interactiveshell} \PY{k+kn}{import} \PY{n}{InteractiveShell}
\PY{k+kn}{import} \PY{n+nn}{matplotlib}\PY{n+nn}{.}\PY{n+nn}{pyplot} \PY{k}{as} \PY{n+nn}{plt}
\PY{k+kn}{import} \PY{n+nn}{seaborn} \PY{k}{as} \PY{n+nn}{sns}
\PY{n}{InteractiveShell}\PY{o}{.}\PY{n}{ast\PYZus{}node\PYZus{}interactivity} \PY{o}{=} \PY{l+s+s1}{\PYZsq{}}\PY{l+s+s1}{all}\PY{l+s+s1}{\PYZsq{}}
\end{Verbatim}
\end{tcolorbox}

    \hypertarget{hw-u4-1ux8bf7ux8bf4ux660eux4e0dux540cux7c7bux578bux7c7bux578bux7684ux5361ux65b9ux68c0ux9a8cux6761ux4ef6ux5e76ux5206ux522bux7ed9ux51faux4f8bux5b50-1ux5206}{%
\paragraph{HW-U4-1:请说明不同类型类型的卡方检验条件,并分别给出例子
(1分)}\label{hw-u4-1ux8bf7ux8bf4ux660eux4e0dux540cux7c7bux578bux7c7bux578bux7684ux5361ux65b9ux68c0ux9a8cux6761ux4ef6ux5e76ux5206ux522bux7ed9ux51faux4f8bux5b50-1ux5206}}

    \begin{enumerate}
\def\labelenumi{\arabic{enumi}.}
\tightlist
\item
  单样本适合度(GoF)卡方检验 条件:频数不能过低。 例子:检验一个骰子是否均匀

\item
  $R\times C$联立表。条件:80\%以上的频数不得小于5。例子:检验一道选择题一个班级的答案
  
  与标准答案的符合程度。
\item
  Yates卡方检验 条件:某个频数出现过少,需要进行不连续校正。


\end{enumerate}

    \hypertarget{hw-u4-2-ux5bf9ux4e24ux7ec4ux75c5ux4ebaux5206ux522bux8fdbux884cux4e24ux7ec4abux836fux7269ux5b9eux9a8c-ux7ed3ux679cux662fux6cbbux6108ux548cux672aux6cbbux6108ux6570ux636eux5982ux4e0b}{%
\paragraph{HW-U4-2: 对两组病人分别进行两组(A,B)药物实验,
结果是治愈和未治愈,数据如下:}\label{hw-u4-2-ux5bf9ux4e24ux7ec4ux75c5ux4ebaux5206ux522bux8fdbux884cux4e24ux7ec4abux836fux7269ux5b9eux9a8c-ux7ed3ux679cux662fux6cbbux6108ux548cux672aux6cbbux6108ux6570ux636eux5982ux4e0b}}

\begin{longtable}[]{@{}lll@{}}
\toprule
Drug & cured & uncured\tabularnewline
\midrule
\endhead
A & 510 & 100\tabularnewline
B & 110 & 50\tabularnewline
total & 620 & 150\tabularnewline
\bottomrule
\end{longtable}

    \hypertarget{ux8ba1ux7b97ux4e24ux79cdux836fux7269ux6cbbux6108ux7387ux768499ux7f6eux4fe1ux533aux95f40.5ux5206}{%
\subparagraph{(1)计算两种药物治愈率的99\%置信区间;(0.5分)}\label{ux8ba1ux7b97ux4e24ux79cdux836fux7269ux6cbbux6108ux7387ux768499ux7f6eux4fe1ux533aux95f40.5ux5206}}

    \begin{tcolorbox}[breakable, size=fbox, boxrule=1pt, pad at break*=1mm,colback=cellbackground, colframe=cellborder]
\prompt{In}{incolor}{3}{\boxspacing}
\begin{Verbatim}[commandchars=\\\{\}]
\PY{k}{def} \PY{n+nf}{proportion\PYZus{}ci\PYZus{}wilson}\PY{p}{(}\PY{n}{p}\PY{p}{,}\PY{n}{alpha}\PY{p}{,}\PY{n}{n}\PY{p}{)}\PY{p}{:}
    \PY{n}{za}\PY{o}{=}\PY{n}{stats}\PY{o}{.}\PY{n}{norm}\PY{o}{.}\PY{n}{isf}\PY{p}{(}\PY{n}{q}\PY{o}{=}\PY{n}{alpha}\PY{o}{/}\PY{l+m+mi}{2}\PY{p}{,}\PY{n}{loc}\PY{o}{=}\PY{l+m+mi}{0}\PY{p}{,}\PY{n}{scale}\PY{o}{=}\PY{l+m+mi}{1}\PY{p}{)}
    \PY{n}{dn}\PY{o}{=}\PY{l+m+mi}{2}\PY{o}{*}\PY{p}{(}\PY{n}{n}\PY{o}{+}\PY{n}{za}\PY{o}{*}\PY{o}{*}\PY{l+m+mi}{2}\PY{p}{)}
    \PY{n}{nl}\PY{o}{=}\PY{l+m+mi}{2}\PY{o}{*}\PY{p}{(}\PY{n}{n}\PY{o}{*}\PY{n}{p}\PY{o}{+}\PY{n}{za}\PY{o}{*}\PY{o}{*}\PY{l+m+mi}{2}\PY{p}{)}\PY{o}{\PYZhy{}}\PY{p}{(}\PY{n}{za}\PY{o}{*}\PY{n}{np}\PY{o}{.}\PY{n}{sqrt}\PY{p}{(}\PY{n}{za}\PY{o}{*}\PY{o}{*}\PY{l+m+mi}{2}\PY{o}{\PYZhy{}}\PY{l+m+mi}{1}\PY{o}{/}\PY{n}{n}\PY{o}{+}\PY{l+m+mi}{4}\PY{o}{*}\PY{n}{n}\PY{o}{*}\PY{n}{p}\PY{o}{*}\PY{p}{(}\PY{l+m+mi}{1}\PY{o}{\PYZhy{}}\PY{n}{p}\PY{p}{)}\PY{o}{+}\PY{l+m+mi}{4}\PY{o}{*}\PY{n}{p}\PY{o}{\PYZhy{}}\PY{l+m+mi}{2}\PY{p}{)}\PY{o}{+}\PY{l+m+mi}{1}\PY{p}{)}
    \PY{n}{nu}\PY{o}{=}\PY{l+m+mi}{2}\PY{o}{*}\PY{p}{(}\PY{n}{n}\PY{o}{*}\PY{n}{p}\PY{o}{+}\PY{n}{za}\PY{o}{*}\PY{o}{*}\PY{l+m+mi}{2}\PY{p}{)}\PY{o}{+}\PY{p}{(}\PY{n}{za}\PY{o}{*}\PY{n}{np}\PY{o}{.}\PY{n}{sqrt}\PY{p}{(}\PY{n}{za}\PY{o}{*}\PY{o}{*}\PY{l+m+mi}{2}\PY{o}{\PYZhy{}}\PY{l+m+mi}{1}\PY{o}{/}\PY{n}{n}\PY{o}{+}\PY{l+m+mi}{4}\PY{o}{*}\PY{n}{n}\PY{o}{*}\PY{n}{p}\PY{o}{*}\PY{p}{(}\PY{l+m+mi}{1}\PY{o}{\PYZhy{}}\PY{n}{p}\PY{p}{)}\PY{o}{\PYZhy{}}\PY{l+m+mi}{4}\PY{o}{*}\PY{n}{p}\PY{o}{\PYZhy{}}\PY{l+m+mi}{2}\PY{p}{)}\PY{o}{+}\PY{l+m+mi}{1}\PY{p}{)}
    \PY{n}{wl}\PY{o}{=}\PY{n+nb}{max}\PY{p}{(}\PY{l+m+mi}{0}\PY{p}{,}\PY{n}{nl}\PY{o}{/}\PY{n}{dn}\PY{p}{)}
    \PY{n}{wu}\PY{o}{=}\PY{n+nb}{min}\PY{p}{(}\PY{l+m+mi}{1}\PY{p}{,}\PY{n}{nu}\PY{o}{/}\PY{n}{dn}\PY{p}{)}
    \PY{k}{return} \PY{n}{wl}\PY{p}{,}\PY{n}{wu}
\PY{k}{def} \PY{n+nf}{proportion\PYZus{}ci\PYZus{}asym}\PY{p}{(}\PY{n}{p}\PY{p}{,}\PY{n}{alpha}\PY{p}{,}\PY{n}{n}\PY{p}{)}\PY{p}{:}
    \PY{n}{z\PYZus{}alpha005}\PY{o}{=}\PY{n}{stats}\PY{o}{.}\PY{n}{norm}\PY{o}{.}\PY{n}{isf}\PY{p}{(}\PY{n}{q}\PY{o}{=}\PY{n}{alpha}\PY{o}{/}\PY{l+m+mi}{2}\PY{p}{,}\PY{n}{loc}\PY{o}{=}\PY{l+m+mi}{0}\PY{p}{,}\PY{n}{scale}\PY{o}{=}\PY{l+m+mi}{1}\PY{p}{)}
    \PY{n}{sigmap}\PY{o}{=}\PY{n}{np}\PY{o}{.}\PY{n}{sqrt}\PY{p}{(}\PY{n}{p}\PY{o}{*}\PY{p}{(}\PY{l+m+mi}{1}\PY{o}{\PYZhy{}}\PY{n}{p}\PY{p}{)}\PY{o}{/}\PY{n}{n}\PY{p}{)}
    \PY{k}{return} \PY{n}{p}\PY{o}{\PYZhy{}}\PY{n}{z\PYZus{}alpha005}\PY{o}{*}\PY{n}{sigmap}\PY{p}{,}\PY{n}{p}\PY{o}{+}\PY{n}{z\PYZus{}alpha005}\PY{o}{*}\PY{n}{sigmap}

\PY{n}{proportion\PYZus{}ci\PYZus{}asym}\PY{p}{(}\PY{l+m+mi}{510}\PY{o}{/}\PY{l+m+mi}{610}\PY{p}{,}\PY{l+m+mf}{0.01}\PY{p}{,}\PY{l+m+mi}{610}\PY{p}{)}
\PY{n}{proportion\PYZus{}ci\PYZus{}wilson}\PY{p}{(}\PY{l+m+mi}{510}\PY{o}{/}\PY{l+m+mi}{610}\PY{p}{,}\PY{l+m+mf}{0.01}\PY{p}{,}\PY{l+m+mi}{610}\PY{p}{)}

\PY{n}{proportion\PYZus{}ci\PYZus{}asym}\PY{p}{(}\PY{l+m+mi}{110}\PY{o}{/}\PY{l+m+mi}{160}\PY{p}{,}\PY{l+m+mf}{0.01}\PY{p}{,}\PY{l+m+mi}{160}\PY{p}{)}
\PY{n}{proportion\PYZus{}ci\PYZus{}wilson}\PY{p}{(}\PY{l+m+mi}{110}\PY{o}{/}\PY{l+m+mi}{160}\PY{p}{,}\PY{l+m+mf}{0.01}\PY{p}{,}\PY{l+m+mi}{160}\PY{p}{)}
\end{Verbatim}
\end{tcolorbox}

            \begin{tcolorbox}[breakable, size=fbox, boxrule=.5pt, pad at break*=1mm, opacityfill=0]
\prompt{Out}{outcolor}{3}{\boxspacing}
\begin{Verbatim}[commandchars=\\\{\}]
(0.7974548969040598, 0.8746762506369239)
\end{Verbatim}
\end{tcolorbox}
        
            \begin{tcolorbox}[breakable, size=fbox, boxrule=.5pt, pad at break*=1mm, opacityfill=0]
\prompt{Out}{outcolor}{3}{\boxspacing}
\begin{Verbatim}[commandchars=\\\{\}]
(0.7983705221403479, 0.8769091055402819)
\end{Verbatim}
\end{tcolorbox}
        
            \begin{tcolorbox}[breakable, size=fbox, boxrule=.5pt, pad at break*=1mm, opacityfill=0]
\prompt{Out}{outcolor}{3}{\boxspacing}
\begin{Verbatim}[commandchars=\\\{\}]
(0.5931116378126502, 0.7818883621873498)
\end{Verbatim}
\end{tcolorbox}
        
            \begin{tcolorbox}[breakable, size=fbox, boxrule=.5pt, pad at break*=1mm, opacityfill=0]
\prompt{Out}{outcolor}{3}{\boxspacing}
\begin{Verbatim}[commandchars=\\\{\}]
(0.6039121674361304, 0.7941905124430988)
\end{Verbatim}
\end{tcolorbox}
        
    \hypertarget{ux7528ux5361ux65b9ux68c0ux9a8cux5206ux6790ux4e24ux79cdux836fux7269ux7597ux6548ux662fux5426ux6709ux5deeux5f02-1ux5206}{%
\subparagraph{(2) 用卡方检验分析两种药物疗效是否有差异;
(1分)}\label{ux7528ux5361ux65b9ux68c0ux9a8cux5206ux6790ux4e24ux79cdux836fux7269ux7597ux6548ux662fux5426ux6709ux5deeux5f02-1ux5206}}

    \begin{tcolorbox}[breakable, size=fbox, boxrule=1pt, pad at break*=1mm,colback=cellbackground, colframe=cellborder]
\prompt{In}{incolor}{6}{\boxspacing}
\begin{Verbatim}[commandchars=\\\{\}]
\PY{n}{stats}\PY{o}{.}\PY{n}{chisquare}\PY{p}{(}\PY{p}{[}\PY{l+m+mi}{510}\PY{p}{,}\PY{l+m+mi}{100}\PY{p}{]}\PY{p}{,}\PY{p}{[}\PY{l+m+mi}{110}\PY{p}{,}\PY{l+m+mi}{50}\PY{p}{]}\PY{p}{,}\PY{n}{ddof}\PY{o}{=}\PY{l+m+mi}{0}\PY{p}{)}
\end{Verbatim}
\end{tcolorbox}

            \begin{tcolorbox}[breakable, size=fbox, boxrule=.5pt, pad at break*=1mm, opacityfill=0]
\prompt{Out}{outcolor}{6}{\boxspacing}
\begin{Verbatim}[commandchars=\\\{\}]
Power\_divergenceResult(statistic=1504.5454545454545, pvalue=0.0)
\end{Verbatim}
\end{tcolorbox}
        
    p-value\textless{}0.05,说明两种药物疗效存在差异。

    \hypertarget{ux7528z-ux68c0ux9a8cux6bd4ux8f83ux4e24ux79cdux836fux7269ux7684ux6cbbux6108ux7387ux5deeux5f02-ux5e76ux4e0eux7f6eux4fe1ux533aux95f4ux65b9ux6cd5ux6bd4ux8f831ux5206}{%
\subparagraph{(3)用z-检验比较两种药物的治愈率差异,
并与置信区间方法比较。(1分)}\label{ux7528z-ux68c0ux9a8cux6bd4ux8f83ux4e24ux79cdux836fux7269ux7684ux6cbbux6108ux7387ux5deeux5f02-ux5e76ux4e0eux7f6eux4fe1ux533aux95f4ux65b9ux6cd5ux6bd4ux8f831ux5206}}

    \begin{tcolorbox}[breakable, size=fbox, boxrule=1pt, pad at break*=1mm,colback=cellbackground, colframe=cellborder]
\prompt{In}{incolor}{9}{\boxspacing}
\begin{Verbatim}[commandchars=\\\{\}]
\PY{n}{proportion}\PY{o}{.}\PY{n}{proportions\PYZus{}ztest}\PY{p}{(}\PY{p}{[}\PY{l+m+mi}{100}\PY{p}{,}\PY{l+m+mi}{50}\PY{p}{]}\PY{p}{,}\PY{p}{[}\PY{l+m+mi}{510}\PY{p}{,}\PY{l+m+mi}{110}\PY{p}{]}\PY{p}{)}
\end{Verbatim}
\end{tcolorbox}

            \begin{tcolorbox}[breakable, size=fbox, boxrule=.5pt, pad at break*=1mm, opacityfill=0]
\prompt{Out}{outcolor}{9}{\boxspacing}
\begin{Verbatim}[commandchars=\\\{\}]
(-5.741004111057862, 9.411680590755424e-09)
\end{Verbatim}
\end{tcolorbox}
        
    \hypertarget{hw-u4-3-ux7528ux5361ux65b9ux68c0ux9a8cux5206ux6790ux4e00ux4e2a4ux52173ux884cux7684rcux8054ux7acbux8868contingency-table-ux5176ux5361ux65b9ux68c0ux9a8cux7684ux81eaux7531ux5ea6ux662fux591aux5c11-0.5ux5206}{%
\paragraph{HW-U4-3:用卡方检验分析一个4列3行的RC联立表(contingency table), 其卡方检验的自由度是多少?
(0.5分)}\label{hw-u4-3-ux7528ux5361ux65b9ux68c0ux9a8cux5206ux6790ux4e00ux4e2a4ux52173ux884cux7684rcux8054ux7acbux8868contingency-table-ux5176ux5361ux65b9ux68c0ux9a8cux7684ux81eaux7531ux5ea6ux662fux591aux5c11-0.5ux5206}}

    \(dof=(4-1)\times(3-1)=6\)

    \hypertarget{hw-u4-4-ux5361ux65b9ux68c0ux9a8cux4e2dux5982ux679cux81eaux7531ux5ea6ux4e3a2ux5361ux65b9ux7edfux8ba1ux91cfux4e3a8.1ux7684ux65f6ux5019ux5bf9ux5e94ux7684pux503cux662fux591aux5c11-1ux5206}{%
\paragraph{HW-U4-4:卡方检验中,如果自由度为2,卡方统计量为8.1的时候,对应的p值是多少?
(1分)}\label{hw-u4-4-ux5361ux65b9ux68c0ux9a8cux4e2dux5982ux679cux81eaux7531ux5ea6ux4e3a2ux5361ux65b9ux7edfux8ba1ux91cfux4e3a8.1ux7684ux65f6ux5019ux5bf9ux5e94ux7684pux503cux662fux591aux5c11-1ux5206}}

    \begin{tcolorbox}[breakable, size=fbox, boxrule=1pt, pad at break*=1mm,colback=cellbackground, colframe=cellborder]
\prompt{In}{incolor}{10}{\boxspacing}
\begin{Verbatim}[commandchars=\\\{\}]
\PY{n}{stats}\PY{o}{.}\PY{n}{chi2}\PY{o}{.}\PY{n}{sf}\PY{p}{(}\PY{l+m+mi}{8}\PY{p}{,}\PY{n}{df}\PY{o}{=}\PY{l+m+mi}{2}\PY{p}{)}
\end{Verbatim}
\end{tcolorbox}

            \begin{tcolorbox}[breakable, size=fbox, boxrule=.5pt, pad at break*=1mm, opacityfill=0]
\prompt{Out}{outcolor}{10}{\boxspacing}
\begin{Verbatim}[commandchars=\\\{\}]
0.018315638888734182
\end{Verbatim}
\end{tcolorbox}
        
    对应的p值是0.018

    \hypertarget{hw-u4-5-ux5982ux679cux8981ux7814ux7a76ux5065ux5eb7ux6559ux80b2ux662fux5426ux4f1aux8ba9ux4ebaux516cux4f17ux63d0ux9ad8ux9632ux75abux610fux8bc6ux4eceux800cux66f4ux52a0ux6ce8ux91cdux52e4ux6d17ux624bux6234ux53e3ux7f69ux91c7ux7528ux4ec0ux4e48ux7edfux8ba1ux65b9ux6cd5ux5408ux90020.5ux5206}{%
\paragraph{HW-U4-5:
如果要研究健康教育是否会让人公众提高防疫意识,从而更加注重勤洗手/戴口罩,
采用什么统计方法合适?(0.5分)}\label{hw-u4-5-ux5982ux679cux8981ux7814ux7a76ux5065ux5eb7ux6559ux80b2ux662fux5426ux4f1aux8ba9ux4ebaux516cux4f17ux63d0ux9ad8ux9632ux75abux610fux8bc6ux4eceux800cux66f4ux52a0ux6ce8ux91cdux52e4ux6d17ux624bux6234ux53e3ux7f69ux91c7ux7528ux4ec0ux4e48ux7edfux8ba1ux65b9ux6cd5ux5408ux90020.5ux5206}}

    
    可以调查若干不同教育水平的勤洗手/戴口罩频率得到RC联立表,
    并采用卡方检验分析一个RC联立表的方法。

    \hypertarget{hw-u4-6-ux5206ux6790ux5168ux56fd34ux4e2aux7701ux81eaux6cbbux533aux76f4ux8f96ux5e02ux7279ux522bux884cux653fux533aux7684ux65b0ux51a0ux75c5ux4ebaux786eux8bcaux6570ux662fux5426ux7b26ux5408ux6b63ux6001ux5206ux5e03ux91c7ux7528ux5361ux65b9ux68c0ux9a8cux7684ux8bddux5bf9ux5e94ux7684ux81eaux7531ux5ea6ux662fux591aux5c11ux8bf4ux51faux7406ux75310.5ux5206}{%
\paragraph{HW-U4-6:
分析全国34个省、自治区、直辖市、特别行政区的新冠病人确诊数是否符合正态分布,
采用卡方检验的话,对应的自由度是多少,说出理由?(0.5分)}\label{hw-u4-6-ux5206ux6790ux5168ux56fd34ux4e2aux7701ux81eaux6cbbux533aux76f4ux8f96ux5e02ux7279ux522bux884cux653fux533aux7684ux65b0ux51a0ux75c5ux4ebaux786eux8bcaux6570ux662fux5426ux7b26ux5408ux6b63ux6001ux5206ux5e03ux91c7ux7528ux5361ux65b9ux68c0ux9a8cux7684ux8bddux5bf9ux5e94ux7684ux81eaux7531ux5ea6ux662fux591aux5c11ux8bf4ux51faux7406ux75310.5ux5206}}

    对应的自由度是\(34-3=31\)。

    \hypertarget{ux9644ux52a0ux9898}{%
\subsection{附加题}\label{ux9644ux52a0ux9898}}

\hypertarget{ux8bf7ux7528bootstrappingux65b9ux6cd5ux8ba1ux7b97ux5355ux4e2aux6bd4ux4f8bux6837ux672cp0n0ux7684ux7f6eux4fe1ux533aux95f4ux5e76ux4e0edemoux4e2dasymptotic-wilson-score-ux65b9ux6cd5ux6bd4ux8f83ux7136ux540eux6539ux53d8p0n0ux7ed9ux51faux89c2ux5bdfux7ed3ux8bba0.5ux5206}{%
\paragraph{1.
请用bootstrapping方法计算单个比例样本(p0,n0)的置信区间,并与demo中asymptotic
, wilson score
方法比较,然后改变p0,n0,给出观察结论(0.5分)。}\label{ux8bf7ux7528bootstrappingux65b9ux6cd5ux8ba1ux7b97ux5355ux4e2aux6bd4ux4f8bux6837ux672cp0n0ux7684ux7f6eux4fe1ux533aux95f4ux5e76ux4e0edemoux4e2dasymptotic-wilson-score-ux65b9ux6cd5ux6bd4ux8f83ux7136ux540eux6539ux53d8p0n0ux7ed9ux51faux89c2ux5bdfux7ed3ux8bba0.5ux5206}}

    \begin{tcolorbox}[breakable, size=fbox, boxrule=1pt, pad at break*=1mm,colback=cellbackground, colframe=cellborder]
\prompt{In}{incolor}{18}{\boxspacing}
\begin{Verbatim}[commandchars=\\\{\}]
\PY{k}{def} \PY{n+nf}{proportion\PYZus{}ci\PYZus{}bootstrap}\PY{p}{(}\PY{n}{p}\PY{p}{,}\PY{n}{alpha}\PY{p}{,}\PY{n}{n}\PY{p}{,}\PY{n}{n\PYZus{}boot}\PY{o}{=}\PY{l+m+mi}{200}\PY{p}{)}\PY{p}{:}
    \PY{n}{bootstrap\PYZus{}means}\PY{o}{=}\PY{p}{[}\PY{p}{]}
    \PY{n}{num}\PY{o}{=}\PY{n+nb}{round}\PY{p}{(}\PY{n}{n}\PY{o}{*}\PY{n}{p}\PY{p}{)}
    \PY{n}{data}\PY{o}{=}\PY{p}{[}\PY{l+m+mi}{1} \PY{k}{for} \PY{n}{i} \PY{o+ow}{in} \PY{n+nb}{range}\PY{p}{(}\PY{n}{num}\PY{p}{)}\PY{p}{]}\PY{o}{+}\PY{p}{[}\PY{l+m+mi}{0} \PY{k}{for} \PY{n}{i} \PY{o+ow}{in} \PY{n+nb}{range}\PY{p}{(}\PY{n}{n}\PY{o}{\PYZhy{}}\PY{n}{num}\PY{p}{)}\PY{p}{]}
    \PY{k}{for} \PY{n}{i} \PY{o+ow}{in} \PY{n+nb}{range}\PY{p}{(}\PY{n}{n\PYZus{}boot}\PY{p}{)}\PY{p}{:}
        \PY{n}{random\PYZus{}sample}\PY{o}{=}\PY{n}{np}\PY{o}{.}\PY{n}{random}\PY{o}{.}\PY{n}{choice}\PY{p}{(}\PY{n}{data}\PY{p}{,}\PY{n+nb}{len}\PY{p}{(}\PY{n}{data}\PY{p}{)}\PY{p}{,}\PY{n}{replace}\PY{o}{=}\PY{k+kc}{True}\PY{p}{)}
        \PY{n}{bootstrap\PYZus{}means}\PY{o}{.}\PY{n}{append}\PY{p}{(}\PY{n}{np}\PY{o}{.}\PY{n}{array}\PY{p}{(}\PY{n}{random\PYZus{}sample}\PY{p}{)}\PY{o}{.}\PY{n}{sum}\PY{p}{(}\PY{p}{)}\PY{o}{/}\PY{n+nb}{len}\PY{p}{(}\PY{n}{data}\PY{p}{)}\PY{o}{\PYZhy{}}\PY{n}{p}\PY{p}{)}
    \PY{n}{ci\PYZus{}l}\PY{p}{,}\PY{n}{ci\PYZus{}h}\PY{o}{=}\PY{n}{p}\PY{o}{+} \PY{n}{np}\PY{o}{.}\PY{n}{percentile}\PY{p}{(}\PY{n}{bootstrap\PYZus{}means}\PY{p}{,} \PY{p}{[}\PY{p}{(}\PY{l+m+mi}{1}\PY{o}{\PYZhy{}}\PY{n}{alpha}\PY{p}{)}\PY{o}{/}\PY{l+m+mi}{2}\PY{o}{*}\PY{l+m+mi}{100}\PY{p}{,}\PY{p}{(}\PY{l+m+mi}{1}\PY{o}{+}\PY{n}{alpha}\PY{p}{)}\PY{o}{/}\PY{l+m+mi}{2}\PY{o}{*}\PY{l+m+mi}{100}\PY{p}{]}\PY{p}{)}
    \PY{k}{return} \PY{n}{ci\PYZus{}l}\PY{p}{,}\PY{n}{ci\PYZus{}h}

\PY{n}{proportion\PYZus{}ci\PYZus{}asym}\PY{p}{(}\PY{l+m+mi}{510}\PY{o}{/}\PY{l+m+mi}{610}\PY{p}{,}\PY{l+m+mf}{0.05}\PY{p}{,}\PY{l+m+mi}{610}\PY{p}{)}
\PY{n}{proportion\PYZus{}ci\PYZus{}wilson}\PY{p}{(}\PY{l+m+mi}{510}\PY{o}{/}\PY{l+m+mi}{610}\PY{p}{,}\PY{l+m+mf}{0.05}\PY{p}{,}\PY{l+m+mi}{610}\PY{p}{)}
\PY{n}{proportion\PYZus{}ci\PYZus{}bootstrap}\PY{p}{(}\PY{l+m+mi}{510}\PY{o}{/}\PY{l+m+mi}{610}\PY{p}{,}\PY{l+m+mf}{0.05}\PY{p}{,}\PY{l+m+mi}{610}\PY{p}{)} 
\end{Verbatim}
\end{tcolorbox}

            \begin{tcolorbox}[breakable, size=fbox, boxrule=.5pt, pad at break*=1mm, opacityfill=0]
\prompt{Out}{outcolor}{18}{\boxspacing}
\begin{Verbatim}[commandchars=\\\{\}]
0.8360655737704918
\end{Verbatim}
\end{tcolorbox}
        
            \begin{tcolorbox}[breakable, size=fbox, boxrule=.5pt, pad at break*=1mm, opacityfill=0]
\prompt{Out}{outcolor}{18}{\boxspacing}
\begin{Verbatim}[commandchars=\\\{\}]
(0.8066864778914842, 0.8654446696494995)
\end{Verbatim}
\end{tcolorbox}
        
            \begin{tcolorbox}[breakable, size=fbox, boxrule=.5pt, pad at break*=1mm, opacityfill=0]
\prompt{Out}{outcolor}{18}{\boxspacing}
\begin{Verbatim}[commandchars=\\\{\}]
(0.8068562899634564, 0.8670355254537873)
\end{Verbatim}
\end{tcolorbox}
        
            \begin{tcolorbox}[breakable, size=fbox, boxrule=.5pt, pad at break*=1mm, opacityfill=0]
\prompt{Out}{outcolor}{18}{\boxspacing}
\begin{Verbatim}[commandchars=\\\{\}]
(0.8360655737704918, 0.8360655737704918)
\end{Verbatim}
\end{tcolorbox}
        
            \begin{tcolorbox}[breakable, size=fbox, boxrule=.5pt, pad at break*=1mm, opacityfill=0]
\prompt{Out}{outcolor}{18}{\boxspacing}
\begin{Verbatim}[commandchars=\\\{\}]
0.01639344262295082
\end{Verbatim}
\end{tcolorbox}
        
            \begin{tcolorbox}[breakable, size=fbox, boxrule=.5pt, pad at break*=1mm, opacityfill=0]
\prompt{Out}{outcolor}{18}{\boxspacing}
\begin{Verbatim}[commandchars=\\\{\}]
(0.006316495870918962, 0.02647038937498268)
\end{Verbatim}
\end{tcolorbox}
        
            \begin{tcolorbox}[breakable, size=fbox, boxrule=.5pt, pad at break*=1mm, opacityfill=0]
\prompt{Out}{outcolor}{18}{\boxspacing}
\begin{Verbatim}[commandchars=\\\{\}]
(0.011480874567673587, 0.03360064248304972)
\end{Verbatim}
\end{tcolorbox}
        
            \begin{tcolorbox}[breakable, size=fbox, boxrule=.5pt, pad at break*=1mm, opacityfill=0]
\prompt{Out}{outcolor}{18}{\boxspacing}
\begin{Verbatim}[commandchars=\\\{\}]
(0.014754098360655738, 0.01639344262295082)
\end{Verbatim}
\end{tcolorbox}
        
    \begin{tcolorbox}[breakable, size=fbox, boxrule=1pt, pad at break*=1mm,colback=cellbackground, colframe=cellborder]
\prompt{In}{incolor}{20}{\boxspacing}
\begin{Verbatim}[commandchars=\\\{\}]
\PY{c+c1}{\PYZsh{} 改变p0}
\PY{n}{proportion\PYZus{}ci\PYZus{}bootstrap}\PY{p}{(}\PY{l+m+mi}{310}\PY{o}{/}\PY{l+m+mi}{610}\PY{p}{,}\PY{l+m+mf}{0.05}\PY{p}{,}\PY{l+m+mi}{610}\PY{p}{)}
\PY{n}{proportion\PYZus{}ci\PYZus{}bootstrap}\PY{p}{(}\PY{l+m+mi}{110}\PY{o}{/}\PY{l+m+mi}{610}\PY{p}{,}\PY{l+m+mf}{0.05}\PY{p}{,}\PY{l+m+mi}{610}\PY{p}{)}
\PY{n}{proportion\PYZus{}ci\PYZus{}bootstrap}\PY{p}{(}\PY{l+m+mi}{10}\PY{o}{/}\PY{l+m+mi}{610}\PY{p}{,}\PY{l+m+mf}{0.05}\PY{p}{,}\PY{l+m+mi}{610}\PY{p}{)}
\end{Verbatim}
\end{tcolorbox}

            \begin{tcolorbox}[breakable, size=fbox, boxrule=.5pt, pad at break*=1mm, opacityfill=0]
\prompt{Out}{outcolor}{20}{\boxspacing}
\begin{Verbatim}[commandchars=\\\{\}]
(0.5114754098360655, 0.5138934426229508)
\end{Verbatim}
\end{tcolorbox}
        
            \begin{tcolorbox}[breakable, size=fbox, boxrule=.5pt, pad at break*=1mm, opacityfill=0]
\prompt{Out}{outcolor}{20}{\boxspacing}
\begin{Verbatim}[commandchars=\\\{\}]
(0.1819672131147541, 0.18360655737704917)
\end{Verbatim}
\end{tcolorbox}
        
            \begin{tcolorbox}[breakable, size=fbox, boxrule=.5pt, pad at break*=1mm, opacityfill=0]
\prompt{Out}{outcolor}{20}{\boxspacing}
\begin{Verbatim}[commandchars=\\\{\}]
0.01639344262295082
\end{Verbatim}
\end{tcolorbox}
        
            \begin{tcolorbox}[breakable, size=fbox, boxrule=.5pt, pad at break*=1mm, opacityfill=0]
\prompt{Out}{outcolor}{20}{\boxspacing}
\begin{Verbatim}[commandchars=\\\{\}]
(0.014754098360655738, 0.01639344262295082)
\end{Verbatim}
\end{tcolorbox}
        
    \begin{tcolorbox}[breakable, size=fbox, boxrule=1pt, pad at break*=1mm,colback=cellbackground, colframe=cellborder]
\prompt{In}{incolor}{21}{\boxspacing}
\begin{Verbatim}[commandchars=\\\{\}]
\PY{c+c1}{\PYZsh{} 改变n0}
\PY{n}{proportion\PYZus{}ci\PYZus{}bootstrap}\PY{p}{(}\PY{l+m+mi}{10}\PY{o}{/}\PY{l+m+mi}{610}\PY{p}{,}\PY{l+m+mf}{0.05}\PY{p}{,}\PY{l+m+mi}{610}\PY{p}{)}
\PY{n}{proportion\PYZus{}ci\PYZus{}bootstrap}\PY{p}{(}\PY{l+m+mi}{10}\PY{o}{/}\PY{l+m+mi}{610}\PY{p}{,}\PY{l+m+mf}{0.05}\PY{p}{,}\PY{l+m+mi}{310}\PY{p}{)}
\PY{n}{proportion\PYZus{}ci\PYZus{}bootstrap}\PY{p}{(}\PY{l+m+mi}{10}\PY{o}{/}\PY{l+m+mi}{610}\PY{p}{,}\PY{l+m+mf}{0.05}\PY{p}{,}\PY{l+m+mi}{110}\PY{p}{)}
\end{Verbatim}
\end{tcolorbox}

            \begin{tcolorbox}[breakable, size=fbox, boxrule=.5pt, pad at break*=1mm, opacityfill=0]
\prompt{Out}{outcolor}{21}{\boxspacing}
\begin{Verbatim}[commandchars=\\\{\}]
(0.014754098360655738, 0.015532786885245916)
\end{Verbatim}
\end{tcolorbox}
        
            \begin{tcolorbox}[breakable, size=fbox, boxrule=.5pt, pad at break*=1mm, opacityfill=0]
\prompt{Out}{outcolor}{21}{\boxspacing}
\begin{Verbatim}[commandchars=\\\{\}]
(0.016129032258064516, 0.016129032258064516)
\end{Verbatim}
\end{tcolorbox}
        
            \begin{tcolorbox}[breakable, size=fbox, boxrule=.5pt, pad at break*=1mm, opacityfill=0]
\prompt{Out}{outcolor}{21}{\boxspacing}
\begin{Verbatim}[commandchars=\\\{\}]
(0.01818181818181818, 0.01818181818181818)
\end{Verbatim}
\end{tcolorbox}
        
    结论: 
    
    1.可以看到bootstrap方法的置信区间范围要明显小于asym和wilson方法。 
    
    2.当p0增大时,bootstrap的置信区间范围减小。 
    
    3.由于重采样的次数n\_boots相同,n0越小,置信区间范围越小。

    \hypertarget{ux7528bootstrappingux65b9ux6cd5ux8ba1ux7b97ux4e24ux4e2aux6837ux672cp1r1n1-p2r2n2ux6bd4ux4f8bp1p2ux5deeux5f02ux7684ux7f6eux4fe1ux533aux95f4-0.5ux5206ux5e76ux4e0ez-ux5206ux5e03ux65b9ux6cd5ux8fdbux884cux6bd4ux8f830.5ux5206}{%
\paragraph{2. 用bootstrapping方法计算两个样本(p1=r1/n1,
p2=r2/n2)比例(p1,p2)差异的置信区间
(0.5分);并与z-分布方法进行比较(0.5分)}\label{ux7528bootstrappingux65b9ux6cd5ux8ba1ux7b97ux4e24ux4e2aux6837ux672cp1r1n1-p2r2n2ux6bd4ux4f8bp1p2ux5deeux5f02ux7684ux7f6eux4fe1ux533aux95f4-0.5ux5206ux5e76ux4e0ez-ux5206ux5e03ux65b9ux6cd5ux8fdbux884cux6bd4ux8f830.5ux5206}}

    \begin{tcolorbox}[breakable, size=fbox, boxrule=1pt, pad at break*=1mm,colback=cellbackground, colframe=cellborder]
\prompt{In}{incolor}{26}{\boxspacing}
\begin{Verbatim}[commandchars=\\\{\}]
\PY{k}{def} \PY{n+nf}{proportion\PYZus{}ci\PYZus{}bootstrap2}\PY{p}{(}\PY{n}{p1}\PY{p}{,}\PY{n}{n1}\PY{p}{,}\PY{n}{p2}\PY{p}{,}\PY{n}{n2}\PY{p}{,}\PY{n}{alpha}\PY{p}{,}\PY{n}{n\PYZus{}boot}\PY{o}{=}\PY{l+m+mi}{200}\PY{p}{)}\PY{p}{:}
    \PY{n}{bootstrap\PYZus{}means}\PY{o}{=}\PY{p}{[}\PY{p}{]}
    \PY{n}{num1}\PY{o}{=}\PY{n+nb}{round}\PY{p}{(}\PY{n}{n1}\PY{o}{*}\PY{n}{p1}\PY{p}{)}
    \PY{n}{num2}\PY{o}{=}\PY{n+nb}{round}\PY{p}{(}\PY{n}{n2}\PY{o}{*}\PY{n}{p2}\PY{p}{)}
    \PY{n}{data1}\PY{o}{=}\PY{p}{[}\PY{l+m+mi}{1} \PY{k}{for} \PY{n}{i} \PY{o+ow}{in} \PY{n+nb}{range}\PY{p}{(}\PY{n}{num1}\PY{p}{)}\PY{p}{]}\PY{o}{+}\PY{p}{[}\PY{l+m+mi}{0} \PY{k}{for} \PY{n}{i} \PY{o+ow}{in} \PY{n+nb}{range}\PY{p}{(}\PY{n}{n1}\PY{o}{\PYZhy{}}\PY{n}{num1}\PY{p}{)}\PY{p}{]}
    \PY{n}{data2}\PY{o}{=}\PY{p}{[}\PY{l+m+mi}{1} \PY{k}{for} \PY{n}{i} \PY{o+ow}{in} \PY{n+nb}{range}\PY{p}{(}\PY{n}{num2}\PY{p}{)}\PY{p}{]}\PY{o}{+}\PY{p}{[}\PY{l+m+mi}{0} \PY{k}{for} \PY{n}{i} \PY{o+ow}{in} \PY{n+nb}{range}\PY{p}{(}\PY{n}{n2}\PY{o}{\PYZhy{}}\PY{n}{num2}\PY{p}{)}\PY{p}{]}
    \PY{n}{length}\PY{o}{=}\PY{n}{n1}\PY{o}{+}\PY{n}{n2}
    \PY{k}{for} \PY{n}{i} \PY{o+ow}{in} \PY{n+nb}{range}\PY{p}{(}\PY{n}{n\PYZus{}boot}\PY{p}{)}\PY{p}{:}
        \PY{n}{random\PYZus{}sample1}\PY{o}{=}\PY{n}{np}\PY{o}{.}\PY{n}{random}\PY{o}{.}\PY{n}{choice}\PY{p}{(}\PY{n}{data1}\PY{p}{,}\PY{n}{length}\PY{p}{,}\PY{n}{replace}\PY{o}{=}\PY{k+kc}{True}\PY{p}{)}
        \PY{n}{random\PYZus{}sample2}\PY{o}{=}\PY{n}{np}\PY{o}{.}\PY{n}{random}\PY{o}{.}\PY{n}{choice}\PY{p}{(}\PY{n}{data2}\PY{p}{,}\PY{n}{length}\PY{p}{,}\PY{n}{replace}\PY{o}{=}\PY{k+kc}{True}\PY{p}{)}
        \PY{n}{random\PYZus{}sample}\PY{o}{=}\PY{n}{random\PYZus{}sample2}\PY{o}{\PYZhy{}}\PY{n}{random\PYZus{}sample1}
        \PY{n}{bootstrap\PYZus{}means}\PY{o}{.}\PY{n}{append}\PY{p}{(}\PY{n}{np}\PY{o}{.}\PY{n}{array}\PY{p}{(}\PY{n}{random\PYZus{}sample}\PY{p}{)}\PY{o}{.}\PY{n}{sum}\PY{p}{(}\PY{p}{)}\PY{o}{/}\PY{n}{length}\PY{o}{\PYZhy{}}\PY{p}{(}\PY{n}{p2}\PY{o}{\PYZhy{}}\PY{n}{p1}\PY{p}{)}\PY{p}{)}
    \PY{n}{ci\PYZus{}l}\PY{p}{,}\PY{n}{ci\PYZus{}h}\PY{o}{=}\PY{n}{p2}\PY{o}{\PYZhy{}}\PY{n}{p1}\PY{o}{+} \PY{n}{np}\PY{o}{.}\PY{n}{percentile}\PY{p}{(}\PY{n}{bootstrap\PYZus{}means}\PY{p}{,} \PY{p}{[}\PY{p}{(}\PY{l+m+mi}{1}\PY{o}{\PYZhy{}}\PY{n}{alpha}\PY{p}{)}\PY{o}{/}\PY{l+m+mi}{2}\PY{o}{*}\PY{l+m+mi}{100}\PY{p}{,}\PY{p}{(}\PY{l+m+mi}{1}\PY{o}{+}\PY{n}{alpha}\PY{p}{)}\PY{o}{/}\PY{l+m+mi}{2}\PY{o}{*}\PY{l+m+mi}{100}\PY{p}{]}\PY{p}{)}
    \PY{k}{return} \PY{n}{ci\PYZus{}l}\PY{p}{,}\PY{n}{ci\PYZus{}h}

\PY{n}{proportion\PYZus{}ci\PYZus{}bootstrap2}\PY{p}{(}\PY{l+m+mi}{510}\PY{o}{/}\PY{l+m+mi}{610}\PY{p}{,}\PY{l+m+mi}{610}\PY{p}{,}\PY{l+m+mi}{110}\PY{o}{/}\PY{l+m+mi}{160}\PY{p}{,}\PY{l+m+mi}{160}\PY{p}{,}\PY{l+m+mf}{0.05}\PY{p}{)}
\PY{n}{proportion}\PY{o}{.}\PY{n}{proportions\PYZus{}ztest}\PY{p}{(}\PY{p}{[}\PY{l+m+mi}{100}\PY{p}{,}\PY{l+m+mi}{50}\PY{p}{]}\PY{p}{,}\PY{p}{[}\PY{l+m+mi}{510}\PY{p}{,}\PY{l+m+mi}{110}\PY{p}{]}\PY{p}{)}
\end{Verbatim}
\end{tcolorbox}

            \begin{tcolorbox}[breakable, size=fbox, boxrule=.5pt, pad at break*=1mm, opacityfill=0]
\prompt{Out}{outcolor}{26}{\boxspacing}
\begin{Verbatim}[commandchars=\\\{\}]
(-0.14805194805194805, -0.14415584415584415)
\end{Verbatim}
\end{tcolorbox}
        
            \begin{tcolorbox}[breakable, size=fbox, boxrule=.5pt, pad at break*=1mm, opacityfill=0]
\prompt{Out}{outcolor}{26}{\boxspacing}
\begin{Verbatim}[commandchars=\\\{\}]
(-5.741004111057862, 9.411680590755424e-09)
\end{Verbatim}
\end{tcolorbox}
        
    同样,bootstrap方法的置信区间宽度要显著低于z-test方法。


    
\end{document}

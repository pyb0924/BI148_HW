\documentclass[11pt]{article}
\usepackage{fontspec, xunicode, xltxtra}
    \setmainfont{Microsoft YaHei}
    \usepackage[breakable]{tcolorbox}
    \usepackage{parskip} % Stop auto-indenting (to mimic markdown behaviour)
    
    \usepackage{iftex}
    \ifPDFTeX
    	\usepackage[T1]{fontenc}
    	\usepackage{mathpazo}
    \else
    	\usepackage{fontspec}
    \fi

    % Basic figure setup, for now with no caption control since it's done
    % automatically by Pandoc (which extracts ![](path) syntax from Markdown).
    \usepackage{graphicx}
    % Maintain compatibility with old templates. Remove in nbconvert 6.0
    \let\Oldincludegraphics\includegraphics
    % Ensure that by default, figures have no caption (until we provide a
    % proper Figure object with a Caption API and a way to capture that
    % in the conversion process - todo).
    \usepackage{caption}
    \DeclareCaptionFormat{nocaption}{}
    \captionsetup{format=nocaption,aboveskip=0pt,belowskip=0pt}

    \usepackage[Export]{adjustbox} % Used to constrain images to a maximum size
    \adjustboxset{max size={0.9\linewidth}{0.9\paperheight}}
    \usepackage{float}
    \floatplacement{figure}{H} % forces figures to be placed at the correct location
    \usepackage{xcolor} % Allow colors to be defined
    \usepackage{enumerate} % Needed for markdown enumerations to work
    \usepackage{geometry} % Used to adjust the document margins
    \usepackage{amsmath} % Equations
    \usepackage{amssymb} % Equations
    \usepackage{textcomp} % defines textquotesingle
    % Hack from http://tex.stackexchange.com/a/47451/13684:
    \AtBeginDocument{%
        \def\PYZsq{\textquotesingle}% Upright quotes in Pygmentized code
    }
    \usepackage{upquote} % Upright quotes for verbatim code
    \usepackage{eurosym} % defines \euro
    \usepackage[mathletters]{ucs} % Extended unicode (utf-8) support
    \usepackage{fancyvrb} % verbatim replacement that allows latex
    \usepackage{grffile} % extends the file name processing of package graphics 
                         % to support a larger range
    \makeatletter % fix for grffile with XeLaTeX
    \def\Gread@@xetex#1{%
      \IfFileExists{"\Gin@base".bb}%
      {\Gread@eps{\Gin@base.bb}}%
      {\Gread@@xetex@aux#1}%
    }
    \makeatother

    % The hyperref package gives us a pdf with properly built
    % internal navigation ('pdf bookmarks' for the table of contents,
    % internal cross-reference links, web links for URLs, etc.)
    \usepackage{hyperref}
    % The default LaTeX title has an obnoxious amount of whitespace. By default,
    % titling removes some of it. It also provides customization options.
    \usepackage{titling}
    \usepackage{longtable} % longtable support required by pandoc >1.10
    \usepackage{booktabs}  % table support for pandoc > 1.12.2
    \usepackage[inline]{enumitem} % IRkernel/repr support (it uses the enumerate* environment)
    \usepackage[normalem]{ulem} % ulem is needed to support strikethroughs (\sout)
                                % normalem makes italics be italics, not underlines
    \usepackage{mathrsfs}
    

    
    % Colors for the hyperref package
    \definecolor{urlcolor}{rgb}{0,.145,.698}
    \definecolor{linkcolor}{rgb}{.71,0.21,0.01}
    \definecolor{citecolor}{rgb}{.12,.54,.11}

    % ANSI colors
    \definecolor{ansi-black}{HTML}{3E424D}
    \definecolor{ansi-black-intense}{HTML}{282C36}
    \definecolor{ansi-red}{HTML}{E75C58}
    \definecolor{ansi-red-intense}{HTML}{B22B31}
    \definecolor{ansi-green}{HTML}{00A250}
    \definecolor{ansi-green-intense}{HTML}{007427}
    \definecolor{ansi-yellow}{HTML}{DDB62B}
    \definecolor{ansi-yellow-intense}{HTML}{B27D12}
    \definecolor{ansi-blue}{HTML}{208FFB}
    \definecolor{ansi-blue-intense}{HTML}{0065CA}
    \definecolor{ansi-magenta}{HTML}{D160C4}
    \definecolor{ansi-magenta-intense}{HTML}{A03196}
    \definecolor{ansi-cyan}{HTML}{60C6C8}
    \definecolor{ansi-cyan-intense}{HTML}{258F8F}
    \definecolor{ansi-white}{HTML}{C5C1B4}
    \definecolor{ansi-white-intense}{HTML}{A1A6B2}
    \definecolor{ansi-default-inverse-fg}{HTML}{FFFFFF}
    \definecolor{ansi-default-inverse-bg}{HTML}{000000}

    % commands and environments needed by pandoc snippets
    % extracted from the output of `pandoc -s`
    \providecommand{\tightlist}{%
      \setlength{\itemsep}{0pt}\setlength{\parskip}{0pt}}
    \DefineVerbatimEnvironment{Highlighting}{Verbatim}{commandchars=\\\{\}}
    % Add ',fontsize=\small' for more characters per line
    \newenvironment{Shaded}{}{}
    \newcommand{\KeywordTok}[1]{\textcolor[rgb]{0.00,0.44,0.13}{\textbf{{#1}}}}
    \newcommand{\DataTypeTok}[1]{\textcolor[rgb]{0.56,0.13,0.00}{{#1}}}
    \newcommand{\DecValTok}[1]{\textcolor[rgb]{0.25,0.63,0.44}{{#1}}}
    \newcommand{\BaseNTok}[1]{\textcolor[rgb]{0.25,0.63,0.44}{{#1}}}
    \newcommand{\FloatTok}[1]{\textcolor[rgb]{0.25,0.63,0.44}{{#1}}}
    \newcommand{\CharTok}[1]{\textcolor[rgb]{0.25,0.44,0.63}{{#1}}}
    \newcommand{\StringTok}[1]{\textcolor[rgb]{0.25,0.44,0.63}{{#1}}}
    \newcommand{\CommentTok}[1]{\textcolor[rgb]{0.38,0.63,0.69}{\textit{{#1}}}}
    \newcommand{\OtherTok}[1]{\textcolor[rgb]{0.00,0.44,0.13}{{#1}}}
    \newcommand{\AlertTok}[1]{\textcolor[rgb]{1.00,0.00,0.00}{\textbf{{#1}}}}
    \newcommand{\FunctionTok}[1]{\textcolor[rgb]{0.02,0.16,0.49}{{#1}}}
    \newcommand{\RegionMarkerTok}[1]{{#1}}
    \newcommand{\ErrorTok}[1]{\textcolor[rgb]{1.00,0.00,0.00}{\textbf{{#1}}}}
    \newcommand{\NormalTok}[1]{{#1}}
    
    % Additional commands for more recent versions of Pandoc
    \newcommand{\ConstantTok}[1]{\textcolor[rgb]{0.53,0.00,0.00}{{#1}}}
    \newcommand{\SpecialCharTok}[1]{\textcolor[rgb]{0.25,0.44,0.63}{{#1}}}
    \newcommand{\VerbatimStringTok}[1]{\textcolor[rgb]{0.25,0.44,0.63}{{#1}}}
    \newcommand{\SpecialStringTok}[1]{\textcolor[rgb]{0.73,0.40,0.53}{{#1}}}
    \newcommand{\ImportTok}[1]{{#1}}
    \newcommand{\DocumentationTok}[1]{\textcolor[rgb]{0.73,0.13,0.13}{\textit{{#1}}}}
    \newcommand{\AnnotationTok}[1]{\textcolor[rgb]{0.38,0.63,0.69}{\textbf{\textit{{#1}}}}}
    \newcommand{\CommentVarTok}[1]{\textcolor[rgb]{0.38,0.63,0.69}{\textbf{\textit{{#1}}}}}
    \newcommand{\VariableTok}[1]{\textcolor[rgb]{0.10,0.09,0.49}{{#1}}}
    \newcommand{\ControlFlowTok}[1]{\textcolor[rgb]{0.00,0.44,0.13}{\textbf{{#1}}}}
    \newcommand{\OperatorTok}[1]{\textcolor[rgb]{0.40,0.40,0.40}{{#1}}}
    \newcommand{\BuiltInTok}[1]{{#1}}
    \newcommand{\ExtensionTok}[1]{{#1}}
    \newcommand{\PreprocessorTok}[1]{\textcolor[rgb]{0.74,0.48,0.00}{{#1}}}
    \newcommand{\AttributeTok}[1]{\textcolor[rgb]{0.49,0.56,0.16}{{#1}}}
    \newcommand{\InformationTok}[1]{\textcolor[rgb]{0.38,0.63,0.69}{\textbf{\textit{{#1}}}}}
    \newcommand{\WarningTok}[1]{\textcolor[rgb]{0.38,0.63,0.69}{\textbf{\textit{{#1}}}}}
    
    
    % Define a nice break command that doesn't care if a line doesn't already
    % exist.
    \def\br{\hspace*{\fill} \\* }
    % Math Jax compatibility definitions
    \def\gt{>}
    \def\lt{<}
    \let\Oldtex\TeX
    \let\Oldlatex\LaTeX
    \renewcommand{\TeX}{\textrm{\Oldtex}}
    \renewcommand{\LaTeX}{\textrm{\Oldlatex}}
    % Document parameters
    % Document title
    \title{Unit2HW}
    
    
    
    
    
% Pygments definitions
\makeatletter
\def\PY@reset{\let\PY@it=\relax \let\PY@bf=\relax%
    \let\PY@ul=\relax \let\PY@tc=\relax%
    \let\PY@bc=\relax \let\PY@ff=\relax}
\def\PY@tok#1{\csname PY@tok@#1\endcsname}
\def\PY@toks#1+{\ifx\relax#1\empty\else%
    \PY@tok{#1}\expandafter\PY@toks\fi}
\def\PY@do#1{\PY@bc{\PY@tc{\PY@ul{%
    \PY@it{\PY@bf{\PY@ff{#1}}}}}}}
\def\PY#1#2{\PY@reset\PY@toks#1+\relax+\PY@do{#2}}

\expandafter\def\csname PY@tok@w\endcsname{\def\PY@tc##1{\textcolor[rgb]{0.73,0.73,0.73}{##1}}}
\expandafter\def\csname PY@tok@c\endcsname{\let\PY@it=\textit\def\PY@tc##1{\textcolor[rgb]{0.25,0.50,0.50}{##1}}}
\expandafter\def\csname PY@tok@cp\endcsname{\def\PY@tc##1{\textcolor[rgb]{0.74,0.48,0.00}{##1}}}
\expandafter\def\csname PY@tok@k\endcsname{\let\PY@bf=\textbf\def\PY@tc##1{\textcolor[rgb]{0.00,0.50,0.00}{##1}}}
\expandafter\def\csname PY@tok@kp\endcsname{\def\PY@tc##1{\textcolor[rgb]{0.00,0.50,0.00}{##1}}}
\expandafter\def\csname PY@tok@kt\endcsname{\def\PY@tc##1{\textcolor[rgb]{0.69,0.00,0.25}{##1}}}
\expandafter\def\csname PY@tok@o\endcsname{\def\PY@tc##1{\textcolor[rgb]{0.40,0.40,0.40}{##1}}}
\expandafter\def\csname PY@tok@ow\endcsname{\let\PY@bf=\textbf\def\PY@tc##1{\textcolor[rgb]{0.67,0.13,1.00}{##1}}}
\expandafter\def\csname PY@tok@nb\endcsname{\def\PY@tc##1{\textcolor[rgb]{0.00,0.50,0.00}{##1}}}
\expandafter\def\csname PY@tok@nf\endcsname{\def\PY@tc##1{\textcolor[rgb]{0.00,0.00,1.00}{##1}}}
\expandafter\def\csname PY@tok@nc\endcsname{\let\PY@bf=\textbf\def\PY@tc##1{\textcolor[rgb]{0.00,0.00,1.00}{##1}}}
\expandafter\def\csname PY@tok@nn\endcsname{\let\PY@bf=\textbf\def\PY@tc##1{\textcolor[rgb]{0.00,0.00,1.00}{##1}}}
\expandafter\def\csname PY@tok@ne\endcsname{\let\PY@bf=\textbf\def\PY@tc##1{\textcolor[rgb]{0.82,0.25,0.23}{##1}}}
\expandafter\def\csname PY@tok@nv\endcsname{\def\PY@tc##1{\textcolor[rgb]{0.10,0.09,0.49}{##1}}}
\expandafter\def\csname PY@tok@no\endcsname{\def\PY@tc##1{\textcolor[rgb]{0.53,0.00,0.00}{##1}}}
\expandafter\def\csname PY@tok@nl\endcsname{\def\PY@tc##1{\textcolor[rgb]{0.63,0.63,0.00}{##1}}}
\expandafter\def\csname PY@tok@ni\endcsname{\let\PY@bf=\textbf\def\PY@tc##1{\textcolor[rgb]{0.60,0.60,0.60}{##1}}}
\expandafter\def\csname PY@tok@na\endcsname{\def\PY@tc##1{\textcolor[rgb]{0.49,0.56,0.16}{##1}}}
\expandafter\def\csname PY@tok@nt\endcsname{\let\PY@bf=\textbf\def\PY@tc##1{\textcolor[rgb]{0.00,0.50,0.00}{##1}}}
\expandafter\def\csname PY@tok@nd\endcsname{\def\PY@tc##1{\textcolor[rgb]{0.67,0.13,1.00}{##1}}}
\expandafter\def\csname PY@tok@s\endcsname{\def\PY@tc##1{\textcolor[rgb]{0.73,0.13,0.13}{##1}}}
\expandafter\def\csname PY@tok@sd\endcsname{\let\PY@it=\textit\def\PY@tc##1{\textcolor[rgb]{0.73,0.13,0.13}{##1}}}
\expandafter\def\csname PY@tok@si\endcsname{\let\PY@bf=\textbf\def\PY@tc##1{\textcolor[rgb]{0.73,0.40,0.53}{##1}}}
\expandafter\def\csname PY@tok@se\endcsname{\let\PY@bf=\textbf\def\PY@tc##1{\textcolor[rgb]{0.73,0.40,0.13}{##1}}}
\expandafter\def\csname PY@tok@sr\endcsname{\def\PY@tc##1{\textcolor[rgb]{0.73,0.40,0.53}{##1}}}
\expandafter\def\csname PY@tok@ss\endcsname{\def\PY@tc##1{\textcolor[rgb]{0.10,0.09,0.49}{##1}}}
\expandafter\def\csname PY@tok@sx\endcsname{\def\PY@tc##1{\textcolor[rgb]{0.00,0.50,0.00}{##1}}}
\expandafter\def\csname PY@tok@m\endcsname{\def\PY@tc##1{\textcolor[rgb]{0.40,0.40,0.40}{##1}}}
\expandafter\def\csname PY@tok@gh\endcsname{\let\PY@bf=\textbf\def\PY@tc##1{\textcolor[rgb]{0.00,0.00,0.50}{##1}}}
\expandafter\def\csname PY@tok@gu\endcsname{\let\PY@bf=\textbf\def\PY@tc##1{\textcolor[rgb]{0.50,0.00,0.50}{##1}}}
\expandafter\def\csname PY@tok@gd\endcsname{\def\PY@tc##1{\textcolor[rgb]{0.63,0.00,0.00}{##1}}}
\expandafter\def\csname PY@tok@gi\endcsname{\def\PY@tc##1{\textcolor[rgb]{0.00,0.63,0.00}{##1}}}
\expandafter\def\csname PY@tok@gr\endcsname{\def\PY@tc##1{\textcolor[rgb]{1.00,0.00,0.00}{##1}}}
\expandafter\def\csname PY@tok@ge\endcsname{\let\PY@it=\textit}
\expandafter\def\csname PY@tok@gs\endcsname{\let\PY@bf=\textbf}
\expandafter\def\csname PY@tok@gp\endcsname{\let\PY@bf=\textbf\def\PY@tc##1{\textcolor[rgb]{0.00,0.00,0.50}{##1}}}
\expandafter\def\csname PY@tok@go\endcsname{\def\PY@tc##1{\textcolor[rgb]{0.53,0.53,0.53}{##1}}}
\expandafter\def\csname PY@tok@gt\endcsname{\def\PY@tc##1{\textcolor[rgb]{0.00,0.27,0.87}{##1}}}
\expandafter\def\csname PY@tok@err\endcsname{\def\PY@bc##1{\setlength{\fboxsep}{0pt}\fcolorbox[rgb]{1.00,0.00,0.00}{1,1,1}{\strut ##1}}}
\expandafter\def\csname PY@tok@kc\endcsname{\let\PY@bf=\textbf\def\PY@tc##1{\textcolor[rgb]{0.00,0.50,0.00}{##1}}}
\expandafter\def\csname PY@tok@kd\endcsname{\let\PY@bf=\textbf\def\PY@tc##1{\textcolor[rgb]{0.00,0.50,0.00}{##1}}}
\expandafter\def\csname PY@tok@kn\endcsname{\let\PY@bf=\textbf\def\PY@tc##1{\textcolor[rgb]{0.00,0.50,0.00}{##1}}}
\expandafter\def\csname PY@tok@kr\endcsname{\let\PY@bf=\textbf\def\PY@tc##1{\textcolor[rgb]{0.00,0.50,0.00}{##1}}}
\expandafter\def\csname PY@tok@bp\endcsname{\def\PY@tc##1{\textcolor[rgb]{0.00,0.50,0.00}{##1}}}
\expandafter\def\csname PY@tok@fm\endcsname{\def\PY@tc##1{\textcolor[rgb]{0.00,0.00,1.00}{##1}}}
\expandafter\def\csname PY@tok@vc\endcsname{\def\PY@tc##1{\textcolor[rgb]{0.10,0.09,0.49}{##1}}}
\expandafter\def\csname PY@tok@vg\endcsname{\def\PY@tc##1{\textcolor[rgb]{0.10,0.09,0.49}{##1}}}
\expandafter\def\csname PY@tok@vi\endcsname{\def\PY@tc##1{\textcolor[rgb]{0.10,0.09,0.49}{##1}}}
\expandafter\def\csname PY@tok@vm\endcsname{\def\PY@tc##1{\textcolor[rgb]{0.10,0.09,0.49}{##1}}}
\expandafter\def\csname PY@tok@sa\endcsname{\def\PY@tc##1{\textcolor[rgb]{0.73,0.13,0.13}{##1}}}
\expandafter\def\csname PY@tok@sb\endcsname{\def\PY@tc##1{\textcolor[rgb]{0.73,0.13,0.13}{##1}}}
\expandafter\def\csname PY@tok@sc\endcsname{\def\PY@tc##1{\textcolor[rgb]{0.73,0.13,0.13}{##1}}}
\expandafter\def\csname PY@tok@dl\endcsname{\def\PY@tc##1{\textcolor[rgb]{0.73,0.13,0.13}{##1}}}
\expandafter\def\csname PY@tok@s2\endcsname{\def\PY@tc##1{\textcolor[rgb]{0.73,0.13,0.13}{##1}}}
\expandafter\def\csname PY@tok@sh\endcsname{\def\PY@tc##1{\textcolor[rgb]{0.73,0.13,0.13}{##1}}}
\expandafter\def\csname PY@tok@s1\endcsname{\def\PY@tc##1{\textcolor[rgb]{0.73,0.13,0.13}{##1}}}
\expandafter\def\csname PY@tok@mb\endcsname{\def\PY@tc##1{\textcolor[rgb]{0.40,0.40,0.40}{##1}}}
\expandafter\def\csname PY@tok@mf\endcsname{\def\PY@tc##1{\textcolor[rgb]{0.40,0.40,0.40}{##1}}}
\expandafter\def\csname PY@tok@mh\endcsname{\def\PY@tc##1{\textcolor[rgb]{0.40,0.40,0.40}{##1}}}
\expandafter\def\csname PY@tok@mi\endcsname{\def\PY@tc##1{\textcolor[rgb]{0.40,0.40,0.40}{##1}}}
\expandafter\def\csname PY@tok@il\endcsname{\def\PY@tc##1{\textcolor[rgb]{0.40,0.40,0.40}{##1}}}
\expandafter\def\csname PY@tok@mo\endcsname{\def\PY@tc##1{\textcolor[rgb]{0.40,0.40,0.40}{##1}}}
\expandafter\def\csname PY@tok@ch\endcsname{\let\PY@it=\textit\def\PY@tc##1{\textcolor[rgb]{0.25,0.50,0.50}{##1}}}
\expandafter\def\csname PY@tok@cm\endcsname{\let\PY@it=\textit\def\PY@tc##1{\textcolor[rgb]{0.25,0.50,0.50}{##1}}}
\expandafter\def\csname PY@tok@cpf\endcsname{\let\PY@it=\textit\def\PY@tc##1{\textcolor[rgb]{0.25,0.50,0.50}{##1}}}
\expandafter\def\csname PY@tok@c1\endcsname{\let\PY@it=\textit\def\PY@tc##1{\textcolor[rgb]{0.25,0.50,0.50}{##1}}}
\expandafter\def\csname PY@tok@cs\endcsname{\let\PY@it=\textit\def\PY@tc##1{\textcolor[rgb]{0.25,0.50,0.50}{##1}}}

\def\PYZbs{\char`\\}
\def\PYZus{\char`\_}
\def\PYZob{\char`\{}
\def\PYZcb{\char`\}}
\def\PYZca{\char`\^}
\def\PYZam{\char`\&}
\def\PYZlt{\char`\<}
\def\PYZgt{\char`\>}
\def\PYZsh{\char`\#}
\def\PYZpc{\char`\%}
\def\PYZdl{\char`\$}
\def\PYZhy{\char`\-}
\def\PYZsq{\char`\'}
\def\PYZdq{\char`\"}
\def\PYZti{\char`\~}
% for compatibility with earlier versions
\def\PYZat{@}
\def\PYZlb{[}
\def\PYZrb{]}
\makeatother


    % For linebreaks inside Verbatim environment from package fancyvrb. 
    \makeatletter
        \newbox\Wrappedcontinuationbox 
        \newbox\Wrappedvisiblespacebox 
        \newcommand*\Wrappedvisiblespace {\textcolor{red}{\textvisiblespace}} 
        \newcommand*\Wrappedcontinuationsymbol {\textcolor{red}{\llap{\tiny$\m@th\hookrightarrow$}}} 
        \newcommand*\Wrappedcontinuationindent {3ex } 
        \newcommand*\Wrappedafterbreak {\kern\Wrappedcontinuationindent\copy\Wrappedcontinuationbox} 
        % Take advantage of the already applied Pygments mark-up to insert 
        % potential linebreaks for TeX processing. 
        %        {, <, #, %, $, ' and ": go to next line. 
        %        _, }, ^, &, >, - and ~: stay at end of broken line. 
        % Use of \textquotesingle for straight quote. 
        \newcommand*\Wrappedbreaksatspecials {% 
            \def\PYGZus{\discretionary{\char`\_}{\Wrappedafterbreak}{\char`\_}}% 
            \def\PYGZob{\discretionary{}{\Wrappedafterbreak\char`\{}{\char`\{}}% 
            \def\PYGZcb{\discretionary{\char`\}}{\Wrappedafterbreak}{\char`\}}}% 
            \def\PYGZca{\discretionary{\char`\^}{\Wrappedafterbreak}{\char`\^}}% 
            \def\PYGZam{\discretionary{\char`\&}{\Wrappedafterbreak}{\char`\&}}% 
            \def\PYGZlt{\discretionary{}{\Wrappedafterbreak\char`\<}{\char`\<}}% 
            \def\PYGZgt{\discretionary{\char`\>}{\Wrappedafterbreak}{\char`\>}}% 
            \def\PYGZsh{\discretionary{}{\Wrappedafterbreak\char`\#}{\char`\#}}% 
            \def\PYGZpc{\discretionary{}{\Wrappedafterbreak\char`\%}{\char`\%}}% 
            \def\PYGZdl{\discretionary{}{\Wrappedafterbreak\char`\$}{\char`\$}}% 
            \def\PYGZhy{\discretionary{\char`\-}{\Wrappedafterbreak}{\char`\-}}% 
            \def\PYGZsq{\discretionary{}{\Wrappedafterbreak\textquotesingle}{\textquotesingle}}% 
            \def\PYGZdq{\discretionary{}{\Wrappedafterbreak\char`\"}{\char`\"}}% 
            \def\PYGZti{\discretionary{\char`\~}{\Wrappedafterbreak}{\char`\~}}% 
        } 
        % Some characters . , ; ? ! / are not pygmentized. 
        % This macro makes them "active" and they will insert potential linebreaks 
        \newcommand*\Wrappedbreaksatpunct {% 
            \lccode`\~`\.\lowercase{\def~}{\discretionary{\hbox{\char`\.}}{\Wrappedafterbreak}{\hbox{\char`\.}}}% 
            \lccode`\~`\,\lowercase{\def~}{\discretionary{\hbox{\char`\,}}{\Wrappedafterbreak}{\hbox{\char`\,}}}% 
            \lccode`\~`\;\lowercase{\def~}{\discretionary{\hbox{\char`\;}}{\Wrappedafterbreak}{\hbox{\char`\;}}}% 
            \lccode`\~`\:\lowercase{\def~}{\discretionary{\hbox{\char`\:}}{\Wrappedafterbreak}{\hbox{\char`\:}}}% 
            \lccode`\~`\?\lowercase{\def~}{\discretionary{\hbox{\char`\?}}{\Wrappedafterbreak}{\hbox{\char`\?}}}% 
            \lccode`\~`\!\lowercase{\def~}{\discretionary{\hbox{\char`\!}}{\Wrappedafterbreak}{\hbox{\char`\!}}}% 
            \lccode`\~`\/\lowercase{\def~}{\discretionary{\hbox{\char`\/}}{\Wrappedafterbreak}{\hbox{\char`\/}}}% 
            \catcode`\.\active
            \catcode`\,\active 
            \catcode`\;\active
            \catcode`\:\active
            \catcode`\?\active
            \catcode`\!\active
            \catcode`\/\active 
            \lccode`\~`\~ 	
        }
    \makeatother

    \let\OriginalVerbatim=\Verbatim
    \makeatletter
    \renewcommand{\Verbatim}[1][1]{%
        %\parskip\z@skip
        \sbox\Wrappedcontinuationbox {\Wrappedcontinuationsymbol}%
        \sbox\Wrappedvisiblespacebox {\FV@SetupFont\Wrappedvisiblespace}%
        \def\FancyVerbFormatLine ##1{\hsize\linewidth
            \vtop{\raggedright\hyphenpenalty\z@\exhyphenpenalty\z@
                \doublehyphendemerits\z@\finalhyphendemerits\z@
                \strut ##1\strut}%
        }%
        % If the linebreak is at a space, the latter will be displayed as visible
        % space at end of first line, and a continuation symbol starts next line.
        % Stretch/shrink are however usually zero for typewriter font.
        \def\FV@Space {%
            \nobreak\hskip\z@ plus\fontdimen3\font minus\fontdimen4\font
            \discretionary{\copy\Wrappedvisiblespacebox}{\Wrappedafterbreak}
            {\kern\fontdimen2\font}%
        }%
        
        % Allow breaks at special characters using \PYG... macros.
        \Wrappedbreaksatspecials
        % Breaks at punctuation characters . , ; ? ! and / need catcode=\active 	
        \OriginalVerbatim[#1,codes*=\Wrappedbreaksatpunct]%
    }
    \makeatother

    % Exact colors from NB
    \definecolor{incolor}{HTML}{303F9F}
    \definecolor{outcolor}{HTML}{D84315}
    \definecolor{cellborder}{HTML}{CFCFCF}
    \definecolor{cellbackground}{HTML}{F7F7F7}
    
    % prompt
    \makeatletter
    \newcommand{\boxspacing}{\kern\kvtcb@left@rule\kern\kvtcb@boxsep}
    \makeatother
    \newcommand{\prompt}[4]{
        \ttfamily\llap{{\color{#2}[#3]:\hspace{3pt}#4}}\vspace{-\baselineskip}
    }
    

    
    % Prevent overflowing lines due to hard-to-break entities
    \sloppy 
    % Setup hyperref package
    \hypersetup{
      breaklinks=true,  % so long urls are correctly broken across lines
      colorlinks=true,
      urlcolor=urlcolor,
      linkcolor=linkcolor,
      citecolor=citecolor,
      }
    % Slightly bigger margins than the latex defaults
    
    \geometry{verbose,tmargin=1in,bmargin=1in,lmargin=1in,rmargin=1in}
    
    

\begin{document}
    
    \maketitle
    
    

    
    \begin{tcolorbox}[breakable, size=fbox, boxrule=1pt, pad at break*=1mm,colback=cellbackground, colframe=cellborder]
\prompt{In}{incolor}{62}{\boxspacing}
\begin{Verbatim}[commandchars=\\\{\}]
\PY{k+kn}{import} \PY{n+nn}{numpy} \PY{k}{as} \PY{n+nn}{np}
\PY{k+kn}{import} \PY{n+nn}{pandas} \PY{k}{as} \PY{n+nn}{pd}
\PY{k+kn}{import} \PY{n+nn}{scipy}\PY{n+nn}{.}\PY{n+nn}{stats} \PY{k}{as} \PY{n+nn}{stats}
\PY{k+kn}{import} \PY{n+nn}{statistics} \PY{k}{as} \PY{n+nn}{sta}
\PY{k+kn}{import} \PY{n+nn}{seaborn} \PY{k}{as} \PY{n+nn}{sns}
\PY{k+kn}{import} \PY{n+nn}{matplotlib}\PY{n+nn}{.}\PY{n+nn}{pyplot} \PY{k}{as} \PY{n+nn}{plt}
\PY{k+kn}{import} \PY{n+nn}{math}
\PY{k+kn}{from} \PY{n+nn}{IPython}\PY{n+nn}{.}\PY{n+nn}{core}\PY{n+nn}{.}\PY{n+nn}{interactiveshell} \PY{k+kn}{import} \PY{n}{InteractiveShell}
\PY{n}{InteractiveShell}\PY{o}{.}\PY{n}{ast\PYZus{}node\PYZus{}interactivity} \PY{o}{=} \PY{l+s+s1}{\PYZsq{}}\PY{l+s+s1}{all}\PY{l+s+s1}{\PYZsq{}}
\PY{n}{sns}\PY{o}{.}\PY{n}{set\PYZus{}style}\PY{p}{(}\PY{l+s+s2}{\PYZdq{}}\PY{l+s+s2}{darkgrid}\PY{l+s+s2}{\PYZdq{}}\PY{p}{)}
\end{Verbatim}
\end{tcolorbox}

    \hypertarget{hw-u2-1-ux5bf9ux4e8eux6837ux672cx252725242729292021253035}{%
\subsubsection{\texorpdfstring{\textbf{HW-U2-1}:
对于样本:x={[}25,27,25,24,27,29,29,20,21,25,30,35{]}}{HW-U2-1: 对于样本:x={[}25,27,25,24,27,29,29,20,21,25,30,35{]}}}\label{hw-u2-1-ux5bf9ux4e8eux6837ux672cx252725242729292021253035}}

    \begin{tcolorbox}[breakable, size=fbox, boxrule=1pt, pad at break*=1mm,colback=cellbackground, colframe=cellborder]
\prompt{In}{incolor}{2}{\boxspacing}
\begin{Verbatim}[commandchars=\\\{\}]
\PY{n}{x}\PY{o}{=}\PY{p}{[}\PY{l+m+mi}{25}\PY{p}{,}\PY{l+m+mi}{27}\PY{p}{,}\PY{l+m+mi}{25}\PY{p}{,}\PY{l+m+mi}{24}\PY{p}{,}\PY{l+m+mi}{27}\PY{p}{,}\PY{l+m+mi}{29}\PY{p}{,}\PY{l+m+mi}{29}\PY{p}{,}\PY{l+m+mi}{20}\PY{p}{,}\PY{l+m+mi}{21}\PY{p}{,}\PY{l+m+mi}{25}\PY{p}{,}\PY{l+m+mi}{30}\PY{p}{,}\PY{l+m+mi}{35}\PY{p}{]}
\PY{n}{x}
\end{Verbatim}
\end{tcolorbox}

            \begin{tcolorbox}[breakable, size=fbox, boxrule=.5pt, pad at break*=1mm, opacityfill=0]
\prompt{Out}{outcolor}{2}{\boxspacing}
\begin{Verbatim}[commandchars=\\\{\}]
[25, 27, 25, 24, 27, 29, 29, 20, 21, 25, 30, 35]
\end{Verbatim}
\end{tcolorbox}
        
    \hypertarget{ux5229ux7528pythonux5148ux5c06ux8fd9ux4e2aux5e8fux5217ux6309ux5347ux5e8fux6392ux5217ux7136ux540eux7528statisticsux6a21ux5757ux5206ux522bux7ed9ux51faux8fd9ux4e2aux5e8fux5217ux7684mean-mode-median-upper_median-low_median}{%
\paragraph{(1)利用Python先将这个序列按升序排列,然后用statistics模块,分别给出这个序列的mean,
mode, median, upper\_median,
low\_median;}\label{ux5229ux7528pythonux5148ux5c06ux8fd9ux4e2aux5e8fux5217ux6309ux5347ux5e8fux6392ux5217ux7136ux540eux7528statisticsux6a21ux5757ux5206ux522bux7ed9ux51faux8fd9ux4e2aux5e8fux5217ux7684mean-mode-median-upper_median-low_median}}

    \begin{tcolorbox}[breakable, size=fbox, boxrule=1pt, pad at break*=1mm,colback=cellbackground, colframe=cellborder]
\prompt{In}{incolor}{3}{\boxspacing}
\begin{Verbatim}[commandchars=\\\{\}]
\PY{n}{x}\PY{o}{.}\PY{n}{sort}\PY{p}{(}\PY{p}{)}
\PY{n}{x}
\end{Verbatim}
\end{tcolorbox}

            \begin{tcolorbox}[breakable, size=fbox, boxrule=.5pt, pad at break*=1mm, opacityfill=0]
\prompt{Out}{outcolor}{3}{\boxspacing}
\begin{Verbatim}[commandchars=\\\{\}]
[20, 21, 24, 25, 25, 25, 27, 27, 29, 29, 30, 35]
\end{Verbatim}
\end{tcolorbox}
        
    \begin{tcolorbox}[breakable, size=fbox, boxrule=1pt, pad at break*=1mm,colback=cellbackground, colframe=cellborder]
\prompt{In}{incolor}{4}{\boxspacing}
\begin{Verbatim}[commandchars=\\\{\}]
\PY{n}{sta}\PY{o}{.}\PY{n}{mean}\PY{p}{(}\PY{n}{x}\PY{p}{)}
\PY{n}{sta}\PY{o}{.}\PY{n}{mode}\PY{p}{(}\PY{n}{x}\PY{p}{)}
\PY{n}{sta}\PY{o}{.}\PY{n}{median}\PY{p}{(}\PY{n}{x}\PY{p}{)}
\PY{n}{sta}\PY{o}{.}\PY{n}{median\PYZus{}high}\PY{p}{(}\PY{n}{x}\PY{p}{)}
\PY{n}{sta}\PY{o}{.}\PY{n}{median\PYZus{}low}\PY{p}{(}\PY{n}{x}\PY{p}{)}
\end{Verbatim}
\end{tcolorbox}

            \begin{tcolorbox}[breakable, size=fbox, boxrule=.5pt, pad at break*=1mm, opacityfill=0]
\prompt{Out}{outcolor}{4}{\boxspacing}
\begin{Verbatim}[commandchars=\\\{\}]
26.416666666666668
\end{Verbatim}
\end{tcolorbox}
        
            \begin{tcolorbox}[breakable, size=fbox, boxrule=.5pt, pad at break*=1mm, opacityfill=0]
\prompt{Out}{outcolor}{4}{\boxspacing}
\begin{Verbatim}[commandchars=\\\{\}]
25
\end{Verbatim}
\end{tcolorbox}
        
            \begin{tcolorbox}[breakable, size=fbox, boxrule=.5pt, pad at break*=1mm, opacityfill=0]
\prompt{Out}{outcolor}{4}{\boxspacing}
\begin{Verbatim}[commandchars=\\\{\}]
26.0
\end{Verbatim}
\end{tcolorbox}
        
            \begin{tcolorbox}[breakable, size=fbox, boxrule=.5pt, pad at break*=1mm, opacityfill=0]
\prompt{Out}{outcolor}{4}{\boxspacing}
\begin{Verbatim}[commandchars=\\\{\}]
27
\end{Verbatim}
\end{tcolorbox}
        
            \begin{tcolorbox}[breakable, size=fbox, boxrule=.5pt, pad at break*=1mm, opacityfill=0]
\prompt{Out}{outcolor}{4}{\boxspacing}
\begin{Verbatim}[commandchars=\\\{\}]
25
\end{Verbatim}
\end{tcolorbox}
        
    \hypertarget{ux9488ux5bf9xux5e8fux5217ux5229ux7528pythonux7684ux51fdux6570numpy.meanx-statistics.meanxux8ba1ux7b97ux5747ux503cux770bux770bux662fux5426ux76f8ux7b49ux5982ux679cux4e0dux7b49ux8bf7ux89e3ux91ca}{%
\paragraph{(2) 针对x序列,利用python的函数numpy.mean(x),statistics.mean(x)计算均值,看看是否相等,
如果不等,请解释。}\label{ux9488ux5bf9xux5e8fux5217ux5229ux7528pythonux7684ux51fdux6570numpy.meanx-statistics.meanxux8ba1ux7b97ux5747ux503cux770bux770bux662fux5426ux76f8ux7b49ux5982ux679cux4e0dux7b49ux8bf7ux89e3ux91ca}}

    \begin{tcolorbox}[breakable, size=fbox, boxrule=1pt, pad at break*=1mm,colback=cellbackground, colframe=cellborder]
\prompt{In}{incolor}{5}{\boxspacing}
\begin{Verbatim}[commandchars=\\\{\}]
\PY{n}{np}\PY{o}{.}\PY{n}{mean}\PY{p}{(}\PY{n}{x}\PY{p}{)}
\PY{n}{sta}\PY{o}{.}\PY{n}{mean}\PY{p}{(}\PY{n}{x}\PY{p}{)}
\PY{n}{x\PYZus{}array}\PY{o}{=}\PY{n}{np}\PY{o}{.}\PY{n}{array}\PY{p}{(}\PY{n}{x}\PY{p}{)}
\PY{n}{np}\PY{o}{.}\PY{n}{mean}\PY{p}{(}\PY{n}{x\PYZus{}array}\PY{p}{)}
\PY{n}{sta}\PY{o}{.}\PY{n}{mean}\PY{p}{(}\PY{n}{x\PYZus{}array}\PY{p}{)}
\end{Verbatim}
\end{tcolorbox}

            \begin{tcolorbox}[breakable, size=fbox, boxrule=.5pt, pad at break*=1mm, opacityfill=0]
\prompt{Out}{outcolor}{5}{\boxspacing}
\begin{Verbatim}[commandchars=\\\{\}]
26.416666666666668
\end{Verbatim}
\end{tcolorbox}
        
            \begin{tcolorbox}[breakable, size=fbox, boxrule=.5pt, pad at break*=1mm, opacityfill=0]
\prompt{Out}{outcolor}{5}{\boxspacing}
\begin{Verbatim}[commandchars=\\\{\}]
26.416666666666668
\end{Verbatim}
\end{tcolorbox}
        
            \begin{tcolorbox}[breakable, size=fbox, boxrule=.5pt, pad at break*=1mm, opacityfill=0]
\prompt{Out}{outcolor}{5}{\boxspacing}
\begin{Verbatim}[commandchars=\\\{\}]
26.416666666666668
\end{Verbatim}
\end{tcolorbox}
        
            \begin{tcolorbox}[breakable, size=fbox, boxrule=.5pt, pad at break*=1mm, opacityfill=0]
\prompt{Out}{outcolor}{5}{\boxspacing}
\begin{Verbatim}[commandchars=\\\{\}]
26
\end{Verbatim}
\end{tcolorbox}
        
    可以看到: 1. x序列利用python的函数numpy.mean(x),
statistics.mean(x)计算出的均值相等。 2.
将其转换为numpy.ndarray后计算出的均值不等

    \begin{tcolorbox}[breakable, size=fbox, boxrule=1pt, pad at break*=1mm,colback=cellbackground, colframe=cellborder]
\prompt{In}{incolor}{6}{\boxspacing}
\begin{Verbatim}[commandchars=\\\{\}]
\PY{n}{x\PYZus{}array}\PY{o}{.}\PY{n}{dtype}
\PY{n+nb}{type}\PY{p}{(}\PY{n}{x}\PY{p}{[}\PY{l+m+mi}{0}\PY{p}{]}\PY{p}{)}
\end{Verbatim}
\end{tcolorbox}

            \begin{tcolorbox}[breakable, size=fbox, boxrule=.5pt, pad at break*=1mm, opacityfill=0]
\prompt{Out}{outcolor}{6}{\boxspacing}
\begin{Verbatim}[commandchars=\\\{\}]
dtype('int32')
\end{Verbatim}
\end{tcolorbox}
        
            \begin{tcolorbox}[breakable, size=fbox, boxrule=.5pt, pad at break*=1mm, opacityfill=0]
\prompt{Out}{outcolor}{6}{\boxspacing}
\begin{Verbatim}[commandchars=\\\{\}]
int
\end{Verbatim}
\end{tcolorbox}
        
    其原因可能是numpy.ndarray和Python内置格式中的数据类型不同,statistic对numpy.ndarray
    求均值时由于int32类型直接作了取整操作。

    \hypertarget{ux5229ux7528pandas.dataframeux5c06xux8f6cux6362ux6210ux6570ux636eux8868dfux7136ux540eux5229ux7528ux6570ux636eux8868ux7684df.describeux7ed9ux51faxux7684ux63cfux8ff0ux6027ux7edfux8ba1ux5e76ux4e0escipy.stats.describexux7684ux8f93ux51faux7ed3ux679cux6bd4ux8f83-ux7279ux522bux5730ux6bd4ux8f83ux5176ux4e2dux7684ux65b9ux5deeux6807ux51c6ux5deeux5e76ux4e0enumpy.varxux6216numpy.stdx-ux6bd4ux8f83ux770bux770bux662fux5426ux4e00ux81f4ux5982ux679cux6709ux5deeux5f02ux8bf7ux89e3ux91ca}{%
\paragraph{(3)利用pandas.DataFrame将x转换成数据表df,然后利用数据表的df.describe()给出x的描述性统计;
并与scipy.stats.describe(x)的输出结果比较;
特别地比较其中的方差/标准差,并与numpy.var(x)
或numpy.std(x)比较看看是否一致,如果有差异,请解释。}\label{ux5229ux7528pandas.dataframeux5c06xux8f6cux6362ux6210ux6570ux636eux8868dfux7136ux540eux5229ux7528ux6570ux636eux8868ux7684df.describeux7ed9ux51faxux7684ux63cfux8ff0ux6027ux7edfux8ba1ux5e76ux4e0escipy.stats.describexux7684ux8f93ux51faux7ed3ux679cux6bd4ux8f83-ux7279ux522bux5730ux6bd4ux8f83ux5176ux4e2dux7684ux65b9ux5deeux6807ux51c6ux5deeux5e76ux4e0enumpy.varxux6216numpy.stdx-ux6bd4ux8f83ux770bux770bux662fux5426ux4e00ux81f4ux5982ux679cux6709ux5deeux5f02ux8bf7ux89e3ux91ca}}

    \begin{tcolorbox}[breakable, size=fbox, boxrule=1pt, pad at break*=1mm,colback=cellbackground, colframe=cellborder]
\prompt{In}{incolor}{7}{\boxspacing}
\begin{Verbatim}[commandchars=\\\{\}]
\PY{n}{df}\PY{o}{=}\PY{n}{pd}\PY{o}{.}\PY{n}{DataFrame}\PY{p}{(}\PY{p}{\PYZob{}}\PY{l+s+s2}{\PYZdq{}}\PY{l+s+s2}{X}\PY{l+s+s2}{\PYZdq{}}\PY{p}{:}\PY{n}{x}\PY{p}{\PYZcb{}}\PY{p}{)}
\PY{n}{df}\PY{o}{.}\PY{n}{describe}\PY{p}{(}\PY{p}{)}
\end{Verbatim}
\end{tcolorbox}

            \begin{tcolorbox}[breakable, size=fbox, boxrule=.5pt, pad at break*=1mm, opacityfill=0]
\prompt{Out}{outcolor}{7}{\boxspacing}
\begin{Verbatim}[commandchars=\\\{\}]
               X
count  12.000000
mean   26.416667
std     4.077841
min    20.000000
25\%    24.750000
50\%    26.000000
75\%    29.000000
max    35.000000
\end{Verbatim}
\end{tcolorbox}
        
    \begin{tcolorbox}[breakable, size=fbox, boxrule=1pt, pad at break*=1mm,colback=cellbackground, colframe=cellborder]
\prompt{In}{incolor}{8}{\boxspacing}
\begin{Verbatim}[commandchars=\\\{\}]
\PY{n}{stats}\PY{o}{.}\PY{n}{describe}\PY{p}{(}\PY{n}{x}\PY{p}{)}
\PY{n}{np}\PY{o}{.}\PY{n}{var}\PY{p}{(}\PY{n}{x}\PY{p}{)}
\PY{n}{np}\PY{o}{.}\PY{n}{var}\PY{p}{(}\PY{n}{x}\PY{p}{,}\PY{n}{ddof}\PY{o}{=}\PY{l+m+mi}{1}\PY{p}{)}
\end{Verbatim}
\end{tcolorbox}

            \begin{tcolorbox}[breakable, size=fbox, boxrule=.5pt, pad at break*=1mm, opacityfill=0]
\prompt{Out}{outcolor}{8}{\boxspacing}
\begin{Verbatim}[commandchars=\\\{\}]
DescribeResult(nobs=12, minmax=(20, 35), mean=26.416666666666668,
variance=16.628787878787875, skewness=0.37455184608977043,
kurtosis=-0.02881305099081022)
\end{Verbatim}
\end{tcolorbox}
        
            \begin{tcolorbox}[breakable, size=fbox, boxrule=.5pt, pad at break*=1mm, opacityfill=0]
\prompt{Out}{outcolor}{8}{\boxspacing}
\begin{Verbatim}[commandchars=\\\{\}]
15.243055555555552
\end{Verbatim}
\end{tcolorbox}
        
            \begin{tcolorbox}[breakable, size=fbox, boxrule=.5pt, pad at break*=1mm, opacityfill=0]
\prompt{Out}{outcolor}{8}{\boxspacing}
\begin{Verbatim}[commandchars=\\\{\}]
16.628787878787875
\end{Verbatim}
\end{tcolorbox}
        
    两者结果不同,原因是stats.describe()计算的是样本方差,np.var()计算的是总体方差。

    \begin{tcolorbox}[breakable, size=fbox, boxrule=1pt, pad at break*=1mm,colback=cellbackground, colframe=cellborder]
\prompt{In}{incolor}{9}{\boxspacing}
\begin{Verbatim}[commandchars=\\\{\}]
\PY{n}{df}\PY{o}{.}\PY{n}{describe}\PY{p}{(}\PY{p}{)}
\PY{n}{np}\PY{o}{.}\PY{n}{std}\PY{p}{(}\PY{n}{x}\PY{p}{)}
\PY{n}{np}\PY{o}{.}\PY{n}{std}\PY{p}{(}\PY{n}{x}\PY{p}{,}\PY{n}{ddof}\PY{o}{=}\PY{l+m+mi}{1}\PY{p}{)}
\end{Verbatim}
\end{tcolorbox}

            \begin{tcolorbox}[breakable, size=fbox, boxrule=.5pt, pad at break*=1mm, opacityfill=0]
\prompt{Out}{outcolor}{9}{\boxspacing}
\begin{Verbatim}[commandchars=\\\{\}]
               X
count  12.000000
mean   26.416667
std     4.077841
min    20.000000
25\%    24.750000
50\%    26.000000
75\%    29.000000
max    35.000000
\end{Verbatim}
\end{tcolorbox}
        
            \begin{tcolorbox}[breakable, size=fbox, boxrule=.5pt, pad at break*=1mm, opacityfill=0]
\prompt{Out}{outcolor}{9}{\boxspacing}
\begin{Verbatim}[commandchars=\\\{\}]
3.9042355917074922
\end{Verbatim}
\end{tcolorbox}
        
            \begin{tcolorbox}[breakable, size=fbox, boxrule=.5pt, pad at break*=1mm, opacityfill=0]
\prompt{Out}{outcolor}{9}{\boxspacing}
\begin{Verbatim}[commandchars=\\\{\}]
4.077841080619483
\end{Verbatim}
\end{tcolorbox}
        
    两者结果也不同,原因是df.describe()计算的是样本标准差,np.var()计算的是总体标准差。

    \hypertarget{hw-u2-2-ux9488ux5bf9ux95eeux9898hw-u2-1ux4e2dux7684ux6570ux636ex}{%
\subsubsection{HW-U2-2:
针对问题HW-U2-1中的数据x:}\label{hw-u2-2-ux9488ux5bf9ux95eeux9898hw-u2-1ux4e2dux7684ux6570ux636ex}}

    \hypertarget{ux5229ux7528stats.tmeanux8ba1ux7b97xux4e2dux8303ux56f4ux57282430ux5185ux7684ux503cux7684ux5747ux503c-ux5e76ux7528np.meanux8fdbux884cux9a8cux8bc1}{%
\paragraph{(1) 利用stats.tmean()计算x中范围在{[}24,30)内的值的均值,
并用np.mean()进行验证}\label{ux5229ux7528stats.tmeanux8ba1ux7b97xux4e2dux8303ux56f4ux57282430ux5185ux7684ux503cux7684ux5747ux503c-ux5e76ux7528np.meanux8fdbux884cux9a8cux8bc1}}

    \begin{tcolorbox}[breakable, size=fbox, boxrule=1pt, pad at break*=1mm,colback=cellbackground, colframe=cellborder]
\prompt{In}{incolor}{10}{\boxspacing}
\begin{Verbatim}[commandchars=\\\{\}]
\PY{n}{stats}\PY{o}{.}\PY{n}{tmean}\PY{p}{(}\PY{n}{x}\PY{p}{,}\PY{p}{[}\PY{l+m+mi}{24}\PY{p}{,}\PY{l+m+mi}{30}\PY{p}{]}\PY{p}{,}\PY{n}{inclusive}\PY{o}{=}\PY{p}{(}\PY{k+kc}{True}\PY{p}{,}\PY{k+kc}{False}\PY{p}{)}\PY{p}{)}
\PY{n}{np}\PY{o}{.}\PY{n}{mean}\PY{p}{(}\PY{p}{[}\PY{l+m+mi}{24}\PY{p}{,}\PY{l+m+mi}{25}\PY{p}{,} \PY{l+m+mi}{25}\PY{p}{,} \PY{l+m+mi}{25}\PY{p}{,} \PY{l+m+mi}{27}\PY{p}{,} \PY{l+m+mi}{27}\PY{p}{,} \PY{l+m+mi}{29}\PY{p}{,} \PY{l+m+mi}{29}\PY{p}{]}\PY{p}{)}
\PY{n}{np}\PY{o}{.}\PY{n}{mean}\PY{p}{(}\PY{n}{x}\PY{p}{)}
\end{Verbatim}
\end{tcolorbox}

            \begin{tcolorbox}[breakable, size=fbox, boxrule=.5pt, pad at break*=1mm, opacityfill=0]
\prompt{Out}{outcolor}{10}{\boxspacing}
\begin{Verbatim}[commandchars=\\\{\}]
26.375
\end{Verbatim}
\end{tcolorbox}
        
            \begin{tcolorbox}[breakable, size=fbox, boxrule=.5pt, pad at break*=1mm, opacityfill=0]
\prompt{Out}{outcolor}{10}{\boxspacing}
\begin{Verbatim}[commandchars=\\\{\}]
26.375
\end{Verbatim}
\end{tcolorbox}
        
            \begin{tcolorbox}[breakable, size=fbox, boxrule=.5pt, pad at break*=1mm, opacityfill=0]
\prompt{Out}{outcolor}{10}{\boxspacing}
\begin{Verbatim}[commandchars=\\\{\}]
26.416666666666668
\end{Verbatim}
\end{tcolorbox}
        
    \hypertarget{ux5229ux7528ux4e24ux79cdux65b9ux6cd5ux5bf9ux4e0aux9762ux7684ux6570ux636exux622aux53d6ux6700ux592710-ux548cux6700ux5c0f10ux540eux8ba1ux7b97ux5747ux503ca-stats.trim_meanb-stats.trimboth-np.meanux5e76ux5c06ux8ba1ux7b97ux7684ux7ed3ux679cux8fdbux884cux6bd4ux8f83}{%
\paragraph{(2)利用两种方法对上面的数据x截取最大10\%和最小10\%后计算均值:(a) stats.trim\_mean()(b) stats.trimboth,
np.mean(),并将计算的结果进行比较;}\label{ux5229ux7528ux4e24ux79cdux65b9ux6cd5ux5bf9ux4e0aux9762ux7684ux6570ux636exux622aux53d6ux6700ux592710-ux548cux6700ux5c0f10ux540eux8ba1ux7b97ux5747ux503ca-stats.trim_meanb-stats.trimboth-np.meanux5e76ux5c06ux8ba1ux7b97ux7684ux7ed3ux679cux8fdbux884cux6bd4ux8f83}}

    \begin{tcolorbox}[breakable, size=fbox, boxrule=1pt, pad at break*=1mm,colback=cellbackground, colframe=cellborder]
\prompt{In}{incolor}{11}{\boxspacing}
\begin{Verbatim}[commandchars=\\\{\}]
\PY{n}{stats}\PY{o}{.}\PY{n}{trim\PYZus{}mean}\PY{p}{(}\PY{n}{x}\PY{p}{,}\PY{l+m+mf}{0.1}\PY{p}{)}
\PY{n}{np}\PY{o}{.}\PY{n}{mean}\PY{p}{(}\PY{n}{stats}\PY{o}{.}\PY{n}{trimboth}\PY{p}{(}\PY{n}{x}\PY{p}{,}\PY{l+m+mf}{0.1}\PY{p}{)}\PY{p}{)}
\end{Verbatim}
\end{tcolorbox}

            \begin{tcolorbox}[breakable, size=fbox, boxrule=.5pt, pad at break*=1mm, opacityfill=0]
\prompt{Out}{outcolor}{11}{\boxspacing}
\begin{Verbatim}[commandchars=\\\{\}]
26.2
\end{Verbatim}
\end{tcolorbox}
        
            \begin{tcolorbox}[breakable, size=fbox, boxrule=.5pt, pad at break*=1mm, opacityfill=0]
\prompt{Out}{outcolor}{11}{\boxspacing}
\begin{Verbatim}[commandchars=\\\{\}]
26.2
\end{Verbatim}
\end{tcolorbox}
        
    \hypertarget{hw-u2-3time_series_19-covid-deaths-mar-15.csv-ux8fd9ux662fux622aux6b62315ux65e5ux7684ux6570ux636eux5168ux7403covid-19ux6b7bux4ea1ux6570ux636eux62a5ux544a}{%
\subsubsection{HW-U2-3:time\_series\_19-covid-Deaths-Mar-15.csv
这是截止3/15日的数据全球covid-19死亡数据报告。}\label{hw-u2-3time_series_19-covid-deaths-mar-15.csv-ux8fd9ux662fux622aux6b62315ux65e5ux7684ux6570ux636eux5168ux7403covid-19ux6b7bux4ea1ux6570ux636eux62a5ux544a}}

    \hypertarget{ux8bf7ux6839ux636eux8fd9ux4e2aux6570ux636eux6587ux4ef6ux753bux51fa122-315ux671fux95f4ux6574ux4e2aux4e2dux56fdux6bcfux65e5ux65b0ux62a5ux544aux7684ux6b7bux4ea1ux75c5ux4f8b}{%
\paragraph{(1)
请根据这个数据文件画出1/22-3/15期间整个中国每日新报告的死亡病例;}\label{ux8bf7ux6839ux636eux8fd9ux4e2aux6570ux636eux6587ux4ef6ux753bux51fa122-315ux671fux95f4ux6574ux4e2aux4e2dux56fdux6bcfux65e5ux65b0ux62a5ux544aux7684ux6b7bux4ea1ux75c5ux4f8b}}

    \begin{tcolorbox}[breakable, size=fbox, boxrule=1pt, pad at break*=1mm,colback=cellbackground, colframe=cellborder]
\prompt{In}{incolor}{35}{\boxspacing}
\begin{Verbatim}[commandchars=\\\{\}]
\PY{n}{df}\PY{o}{=}\PY{n}{pd}\PY{o}{.}\PY{n}{read\PYZus{}csv}\PY{p}{(}\PY{l+s+s1}{\PYZsq{}}\PY{l+s+s1}{time\PYZus{}series\PYZus{}19\PYZhy{}covid\PYZhy{}Deaths\PYZhy{}Mar\PYZhy{}15.csv}\PY{l+s+s1}{\PYZsq{}}\PY{p}{)}
\PY{n}{df}\PY{o}{=}\PY{n}{df}\PY{o}{.}\PY{n}{groupby}\PY{p}{(}\PY{p}{[}\PY{l+s+s2}{\PYZdq{}}\PY{l+s+s2}{Country/Region}\PY{l+s+s2}{\PYZdq{}}\PY{p}{]}\PY{p}{)}\PY{o}{.}\PY{n}{sum}\PY{p}{(}\PY{p}{)}
\PY{n}{df}\PY{o}{=}\PY{n}{df}\PY{o}{.}\PY{n}{loc}\PY{p}{[}\PY{l+s+s2}{\PYZdq{}}\PY{l+s+s2}{China}\PY{l+s+s2}{\PYZdq{}}\PY{p}{,}\PY{l+s+s2}{\PYZdq{}}\PY{l+s+s2}{1/22/20}\PY{l+s+s2}{\PYZdq{}}\PY{p}{:}\PY{l+s+s2}{\PYZdq{}}\PY{l+s+s2}{3/15/20}\PY{l+s+s2}{\PYZdq{}}\PY{p}{]}
\PY{n}{value}\PY{o}{=}\PY{p}{[}\PY{p}{]}
\PY{k}{for} \PY{n}{i} \PY{o+ow}{in} \PY{n+nb}{range}\PY{p}{(}\PY{n+nb}{len}\PY{p}{(}\PY{n}{df}\PY{p}{)}\PY{o}{\PYZhy{}}\PY{l+m+mi}{1}\PY{p}{)}\PY{p}{:}
    \PY{n}{value}\PY{o}{.}\PY{n}{append}\PY{p}{(}\PY{n}{df}\PY{o}{.}\PY{n}{values}\PY{p}{[}\PY{n}{i}\PY{o}{+}\PY{l+m+mi}{1}\PY{p}{]}\PY{o}{\PYZhy{}}\PY{n}{df}\PY{o}{.}\PY{n}{values}\PY{p}{[}\PY{n}{i}\PY{p}{]}\PY{p}{)}
\PY{n}{data}\PY{o}{=}\PY{n}{pd}\PY{o}{.}\PY{n}{Series}\PY{p}{(}\PY{n}{value}\PY{p}{,}\PY{n}{index}\PY{o}{=}\PY{n}{df}\PY{o}{.}\PY{n}{index}\PY{p}{[}\PY{l+m+mi}{1}\PY{p}{:}\PY{p}{]}\PY{p}{)}
\PY{n}{data}\PY{o}{.}\PY{n}{plot}\PY{p}{(}\PY{p}{)}
\end{Verbatim}
\end{tcolorbox}

            \begin{tcolorbox}[breakable, size=fbox, boxrule=.5pt, pad at break*=1mm, opacityfill=0]
\prompt{Out}{outcolor}{35}{\boxspacing}
\begin{Verbatim}[commandchars=\\\{\}]
<matplotlib.axes.\_subplots.AxesSubplot at 0x2a165cb9048>
\end{Verbatim}
\end{tcolorbox}
        
    \begin{center}
    \adjustimage{max size={0.9\linewidth}{0.9\paperheight}}{output_24_1.png}
    \end{center}
    { \hspace*{\fill} \\}
    
    \hypertarget{ux8ba1ux7b97122-315ux671fux95f4ux6574ux4e2aux4e2dux56fdux6bcfux65e5ux65b0ux62a5ux544aux7684ux6b7bux4ea1ux75c5ux4f8bux6570ux636eux7684skewness-kurtosis-fisher-ux503c}{%
\paragraph{(2)
计算1/22-3/15期间整个中国每日新报告的死亡病例数据的skewness , kurtosis
(fisher)
值;}\label{ux8ba1ux7b97122-315ux671fux95f4ux6574ux4e2aux4e2dux56fdux6bcfux65e5ux65b0ux62a5ux544aux7684ux6b7bux4ea1ux75c5ux4f8bux6570ux636eux7684skewness-kurtosis-fisher-ux503c}}

    \begin{tcolorbox}[breakable, size=fbox, boxrule=1pt, pad at break*=1mm,colback=cellbackground, colframe=cellborder]
\prompt{In}{incolor}{36}{\boxspacing}
\begin{Verbatim}[commandchars=\\\{\}]
\PY{n}{stats}\PY{o}{.}\PY{n}{skew}\PY{p}{(}\PY{n}{data}\PY{o}{.}\PY{n}{values}\PY{p}{)}
\PY{n}{stats}\PY{o}{.}\PY{n}{kurtosis}\PY{p}{(}\PY{n}{data}\PY{o}{.}\PY{n}{values}\PY{p}{)}
\end{Verbatim}
\end{tcolorbox}

            \begin{tcolorbox}[breakable, size=fbox, boxrule=.5pt, pad at break*=1mm, opacityfill=0]
\prompt{Out}{outcolor}{36}{\boxspacing}
\begin{Verbatim}[commandchars=\\\{\}]
1.351808123400179
\end{Verbatim}
\end{tcolorbox}
        
            \begin{tcolorbox}[breakable, size=fbox, boxrule=.5pt, pad at break*=1mm, opacityfill=0]
\prompt{Out}{outcolor}{36}{\boxspacing}
\begin{Verbatim}[commandchars=\\\{\}]
1.839410067721043
\end{Verbatim}
\end{tcolorbox}
        
    \hypertarget{ux5982ux679cux5c06ux6570ux636eux4e2dux53d8ux5316ux6bd4ux8f83ux5927ux7684ux65e5ux671fux6bd4ux59825-ux6216150ux4f5cux4e3aoutliersux53bbux6389ux91cdux65b0ux8ba1ux7b97skewness-kurtosisux6839ux636eux7ed3ux679cux8bc4ux4f30ux4e00ux4e0bux8fd9ux4e2aux6570ux636eux7684ux6b63ux6001ux6027}{%
\paragraph{(3) 如果将数据中变化比较大的日期(比如\textless{}=5,
或\textgreater{}=150)作为outliers去掉,重新计算skewness ,
kurtosis,根据结果评估一下这个数据的正态性。}\label{ux5982ux679cux5c06ux6570ux636eux4e2dux53d8ux5316ux6bd4ux8f83ux5927ux7684ux65e5ux671fux6bd4ux59825-ux6216150ux4f5cux4e3aoutliersux53bbux6389ux91cdux65b0ux8ba1ux7b97skewness-kurtosisux6839ux636eux7ed3ux679cux8bc4ux4f30ux4e00ux4e0bux8fd9ux4e2aux6570ux636eux7684ux6b63ux6001ux6027}}

    \begin{tcolorbox}[breakable, size=fbox, boxrule=1pt, pad at break*=1mm,colback=cellbackground, colframe=cellborder]
\prompt{In}{incolor}{29}{\boxspacing}
\begin{Verbatim}[commandchars=\\\{\}]
\PY{n}{data1}\PY{o}{=}\PY{n}{data}\PY{p}{[}\PY{n}{data}\PY{o}{.}\PY{n}{values}\PY{o}{\PYZgt{}}\PY{l+m+mi}{5}\PY{p}{]}
\PY{n}{stats}\PY{o}{.}\PY{n}{skew}\PY{p}{(}\PY{n}{data1}\PY{p}{)}
\PY{n}{stats}\PY{o}{.}\PY{n}{kurtosis}\PY{p}{(}\PY{n}{data1}\PY{p}{)}
\end{Verbatim}
\end{tcolorbox}

            \begin{tcolorbox}[breakable, size=fbox, boxrule=.5pt, pad at break*=1mm, opacityfill=0]
\prompt{Out}{outcolor}{29}{\boxspacing}
\begin{Verbatim}[commandchars=\\\{\}]
1.3543571077255594
\end{Verbatim}
\end{tcolorbox}
        
            \begin{tcolorbox}[breakable, size=fbox, boxrule=.5pt, pad at break*=1mm, opacityfill=0]
\prompt{Out}{outcolor}{29}{\boxspacing}
\begin{Verbatim}[commandchars=\\\{\}]
1.827630952069665
\end{Verbatim}
\end{tcolorbox}
        
    \begin{tcolorbox}[breakable, size=fbox, boxrule=1pt, pad at break*=1mm,colback=cellbackground, colframe=cellborder]
\prompt{In}{incolor}{38}{\boxspacing}
\begin{Verbatim}[commandchars=\\\{\}]
\PY{n}{data2}\PY{o}{=}\PY{n}{data}\PY{p}{[}\PY{n}{data}\PY{o}{.}\PY{n}{values}\PY{o}{\PYZlt{}}\PY{l+m+mi}{150}\PY{p}{]}
\PY{n}{stats}\PY{o}{.}\PY{n}{skew}\PY{p}{(}\PY{n}{data2}\PY{p}{)}
\PY{n}{stats}\PY{o}{.}\PY{n}{kurtosis}\PY{p}{(}\PY{n}{data2}\PY{p}{)}
\end{Verbatim}
\end{tcolorbox}

            \begin{tcolorbox}[breakable, size=fbox, boxrule=.5pt, pad at break*=1mm, opacityfill=0]
\prompt{Out}{outcolor}{38}{\boxspacing}
\begin{Verbatim}[commandchars=\\\{\}]
0.709423012821765
\end{Verbatim}
\end{tcolorbox}
        
            \begin{tcolorbox}[breakable, size=fbox, boxrule=.5pt, pad at break*=1mm, opacityfill=0]
\prompt{Out}{outcolor}{38}{\boxspacing}
\begin{Verbatim}[commandchars=\\\{\}]
-0.5646552353924768
\end{Verbatim}
\end{tcolorbox}
        
    可以看到,去除较小值后,数据的正态性变化不明显。去除较大值后,数据正态性明显增加,
    
    skewness和kurtosis都显著更接近0

    \hypertarget{hw-u2-4-ux53d8ux5f02ux7cfbux6570coefficient-of-variation-cv-ux548cux71b5-entropy}{%
\subsubsection{HW-U2-4, 变异系数(coefficient of variation, cv )和熵
(entropy)}\label{hw-u2-4-ux53d8ux5f02ux7cfbux6570coefficient-of-variation-cv-ux548cux71b5-entropy}}

    \hypertarget{ux8bf7ux8ba1ux7b97ux4e0aux4e00ux9898ux6570ux636eux4e2dux7684122-315ux65e5ux671fux95f4ux4e2dux56fdux6bcfux65e5ux65b0ux589eux6b7bux4ea1ux75c5ux4f8bux7684cv}{%
\paragraph{(1)请计算上一题数据中的1/22-3/15日期间中国每日新增死亡病例的cv}\label{ux8bf7ux8ba1ux7b97ux4e0aux4e00ux9898ux6570ux636eux4e2dux7684122-315ux65e5ux671fux95f4ux4e2dux56fdux6bcfux65e5ux65b0ux589eux6b7bux4ea1ux75c5ux4f8bux7684cv}}

    \begin{tcolorbox}[breakable, size=fbox, boxrule=1pt, pad at break*=1mm,colback=cellbackground, colframe=cellborder]
\prompt{In}{incolor}{41}{\boxspacing}
\begin{Verbatim}[commandchars=\\\{\}]
\PY{n}{cv}\PY{o}{=}\PY{n}{np}\PY{o}{.}\PY{n}{std}\PY{p}{(}\PY{n}{data}\PY{p}{,}\PY{n}{ddof}\PY{o}{=}\PY{l+m+mi}{1}\PY{p}{)}\PY{o}{/}\PY{n}{np}\PY{o}{.}\PY{n}{mean}\PY{p}{(}\PY{n}{data}\PY{p}{)}
\PY{n}{cv}
\end{Verbatim}
\end{tcolorbox}

            \begin{tcolorbox}[breakable, size=fbox, boxrule=.5pt, pad at break*=1mm, opacityfill=0]
\prompt{Out}{outcolor}{41}{\boxspacing}
\begin{Verbatim}[commandchars=\\\{\}]
0.9059307542663461
\end{Verbatim}
\end{tcolorbox}
        
    \hypertarget{ux8bf7ux6839ux636eentropyux7684ux5b9aux4e49ux516cux5f0fux4ee5ux53capython-ux51fdux6570scipy.stats.entropy-ux5206ux522bux8ba1ux7b97ux6982ux7387p14112-16-16-13ux7684ux71b5-ux5bf9ux6570ux4ee52ux4e3aux5e95}{%
\paragraph{(2)请根据entropy的定义公式,以及Python函数scipy.stats.entropy() 分别计算概率P=\{1/4,1/12, 1/6, 1/6 ,1/3\}的熵(对数以2为底)}\label{ux8bf7ux6839ux636eentropyux7684ux5b9aux4e49ux516cux5f0fux4ee5ux53capython-ux51fdux6570scipy.stats.entropy-ux5206ux522bux8ba1ux7b97ux6982ux7387p14112-16-16-13ux7684ux71b5-ux5bf9ux6570ux4ee52ux4e3aux5e95}}

    \begin{tcolorbox}[breakable, size=fbox, boxrule=1pt, pad at break*=1mm,colback=cellbackground, colframe=cellborder]
\prompt{In}{incolor}{83}{\boxspacing}
\begin{Verbatim}[commandchars=\\\{\}]
\PY{n}{p}\PY{o}{=}\PY{p}{[}\PY{l+m+mi}{1}\PY{o}{/}\PY{l+m+mi}{4}\PY{p}{,}\PY{l+m+mi}{1}\PY{o}{/}\PY{l+m+mi}{12}\PY{p}{,}\PY{l+m+mi}{1}\PY{o}{/}\PY{l+m+mi}{6}\PY{p}{,}\PY{l+m+mi}{1}\PY{o}{/}\PY{l+m+mi}{6}\PY{p}{,}\PY{l+m+mi}{1}\PY{o}{/}\PY{l+m+mi}{3}\PY{p}{]}
\PY{n}{probability}\PY{o}{=}\PY{p}{[}\PY{p}{]}
\PY{n}{prob}\PY{o}{=}\PY{p}{(}\PY{n}{pd}\PY{o}{.}\PY{n}{Series}\PY{p}{(}\PY{n}{p}\PY{p}{)}\PY{o}{.}\PY{n}{value\PYZus{}counts}\PY{p}{(}\PY{p}{)}\PY{o}{.}\PY{n}{values}\PY{p}{)}\PY{o}{/}\PY{n+nb}{len}\PY{p}{(}\PY{n}{p}\PY{p}{)}
\PY{n}{prob}
\end{Verbatim}
\end{tcolorbox}

            \begin{tcolorbox}[breakable, size=fbox, boxrule=.5pt, pad at break*=1mm, opacityfill=0]
\prompt{Out}{outcolor}{83}{\boxspacing}
\begin{Verbatim}[commandchars=\\\{\}]
array([0.4, 0.2, 0.2, 0.2])
\end{Verbatim}
\end{tcolorbox}
        
    \begin{tcolorbox}[breakable, size=fbox, boxrule=1pt, pad at break*=1mm,colback=cellbackground, colframe=cellborder]
\prompt{In}{incolor}{85}{\boxspacing}
\begin{Verbatim}[commandchars=\\\{\}]
\PY{n}{stats}\PY{o}{.}\PY{n}{entropy}\PY{p}{(}\PY{n}{prob}\PY{p}{,}\PY{n}{base}\PY{o}{=}\PY{l+m+mi}{2}\PY{p}{)} 
\PY{n}{result}\PY{o}{=}\PY{l+m+mi}{0}
\PY{k}{for} \PY{n}{pro} \PY{o+ow}{in} \PY{n}{prob}\PY{p}{:}
    \PY{n}{result}\PY{o}{\PYZhy{}}\PY{o}{=}\PY{n}{pro}\PY{o}{*}\PY{n}{math}\PY{o}{.}\PY{n}{log2}\PY{p}{(}\PY{n}{pro}\PY{p}{)}
\PY{n}{result}
\end{Verbatim}
\end{tcolorbox}

            \begin{tcolorbox}[breakable, size=fbox, boxrule=.5pt, pad at break*=1mm, opacityfill=0]
\prompt{Out}{outcolor}{85}{\boxspacing}
\begin{Verbatim}[commandchars=\\\{\}]
1.9219280948873625
\end{Verbatim}
\end{tcolorbox}
        
            \begin{tcolorbox}[breakable, size=fbox, boxrule=.5pt, pad at break*=1mm, opacityfill=0]
\prompt{Out}{outcolor}{85}{\boxspacing}
\begin{Verbatim}[commandchars=\\\{\}]
1.9219280948873623
\end{Verbatim}
\end{tcolorbox}
        
    \hypertarget{hw-u2-5-grouped-data-ux6709ux4e00ux627930ux53eaux5b9eux9a8crats-ux91cdux91cfux5206ux5e03ux5982ux4e0bux8868-ux8bf7}{%
\subsubsection{HW-U2-5, Grouped Data 有一批30只实验rats, 重量分布如下表,
请}\label{hw-u2-5-grouped-data-ux6709ux4e00ux627930ux53eaux5b9eux9a8crats-ux91cdux91cfux5206ux5e03ux5982ux4e0bux8868-ux8bf7}}

    \begin{tcolorbox}[breakable, size=fbox, boxrule=1pt, pad at break*=1mm,colback=cellbackground, colframe=cellborder]
\prompt{In}{incolor}{124}{\boxspacing}
\begin{Verbatim}[commandchars=\\\{\}]
\PY{n}{weight}\PY{o}{=}\PY{n}{np}\PY{o}{.}\PY{n}{array}\PY{p}{(}\PY{p}{[}\PY{l+m+mi}{275}\PY{p}{,}\PY{l+m+mi}{225}\PY{p}{,}\PY{l+m+mi}{175}\PY{p}{,}\PY{l+m+mi}{125}\PY{p}{]}\PY{p}{)}
\PY{n}{freq}\PY{o}{=}\PY{n}{np}\PY{o}{.}\PY{n}{array}\PY{p}{(}\PY{p}{[}\PY{l+m+mi}{5}\PY{p}{,}\PY{l+m+mi}{10}\PY{p}{,}\PY{l+m+mi}{5}\PY{p}{,}\PY{l+m+mi}{10}\PY{p}{]}\PY{p}{)}
\end{Verbatim}
\end{tcolorbox}

    \hypertarget{ux6839ux636emean-mode-median-std-varux7684ux516cux5f0fux8ba1ux7b97ux8fd9ux4e9bux6837ux672cux7684ux63cfux8ff0ux6027ux7edfux8ba1ux91cf}{%
\paragraph{(1) 根据mean, mode, median, std,
var的公式计算这些样本的描述性统计量;}\label{ux6839ux636emean-mode-median-std-varux7684ux516cux5f0fux8ba1ux7b97ux8fd9ux4e9bux6837ux672cux7684ux63cfux8ff0ux6027ux7edfux8ba1ux91cf}}

    \begin{tcolorbox}[breakable, size=fbox, boxrule=1pt, pad at break*=1mm,colback=cellbackground, colframe=cellborder]
\prompt{In}{incolor}{129}{\boxspacing}
\begin{Verbatim}[commandchars=\\\{\}]
\PY{c+c1}{\PYZsh{} mean}
\PY{n}{my\PYZus{}mean}\PY{o}{=}\PY{l+m+mi}{0}
\PY{k}{for} \PY{n}{i} \PY{o+ow}{in} \PY{n+nb}{range}\PY{p}{(}\PY{n+nb}{len}\PY{p}{(}\PY{n}{weight}\PY{p}{)}\PY{p}{)}\PY{p}{:}
    \PY{n}{my\PYZus{}mean}\PY{o}{+}\PY{o}{=}\PY{n}{weight}\PY{p}{[}\PY{n}{i}\PY{p}{]}\PY{o}{*}\PY{n}{freq}\PY{p}{[}\PY{n}{i}\PY{p}{]}
\PY{n}{my\PYZus{}mean}\PY{o}{=}\PY{n}{my\PYZus{}mean}\PY{o}{/}\PY{n}{freq}\PY{o}{.}\PY{n}{sum}\PY{p}{(}\PY{p}{)}
\PY{n}{my\PYZus{}mean}
\end{Verbatim}
\end{tcolorbox}

            \begin{tcolorbox}[breakable, size=fbox, boxrule=.5pt, pad at break*=1mm, opacityfill=0]
\prompt{Out}{outcolor}{129}{\boxspacing}
\begin{Verbatim}[commandchars=\\\{\}]
191.66666666666666
\end{Verbatim}
\end{tcolorbox}
        
    \begin{tcolorbox}[breakable, size=fbox, boxrule=1pt, pad at break*=1mm,colback=cellbackground, colframe=cellborder]
\prompt{In}{incolor}{133}{\boxspacing}
\begin{Verbatim}[commandchars=\\\{\}]
\PY{c+c1}{\PYZsh{} mode}
\PY{n}{my\PYZus{}mode}\PY{o}{=}\PY{n}{weight}\PY{p}{[}\PY{n}{np}\PY{o}{.}\PY{n}{argmax}\PY{p}{(}\PY{n}{freq}\PY{p}{)}\PY{p}{]}
\PY{n}{my\PYZus{}mode}
\end{Verbatim}
\end{tcolorbox}

            \begin{tcolorbox}[breakable, size=fbox, boxrule=.5pt, pad at break*=1mm, opacityfill=0]
\prompt{Out}{outcolor}{133}{\boxspacing}
\begin{Verbatim}[commandchars=\\\{\}]
225
\end{Verbatim}
\end{tcolorbox}
        
    \begin{tcolorbox}[breakable, size=fbox, boxrule=1pt, pad at break*=1mm,colback=cellbackground, colframe=cellborder]
\prompt{In}{incolor}{179}{\boxspacing}
\begin{Verbatim}[commandchars=\\\{\}]
\PY{c+c1}{\PYZsh{} median}
\PY{n}{tmp}\PY{o}{=}\PY{n}{pd}\PY{o}{.}\PY{n}{Series}\PY{p}{(}\PY{n}{weight}\PY{p}{,}\PY{n}{index}\PY{o}{=}\PY{n}{freq}\PY{p}{)}\PY{o}{.}\PY{n}{sort\PYZus{}values}\PY{p}{(}\PY{p}{)}
\PY{n}{weight\PYZus{}sorted}\PY{o}{=}\PY{n}{tmp}\PY{o}{.}\PY{n}{values}
\PY{n}{freq\PYZus{}sorted}\PY{o}{=}\PY{n}{tmp}\PY{o}{.}\PY{n}{index}
\PY{n}{weight\PYZus{}str}\PY{o}{=}\PY{l+s+s1}{\PYZsq{}}\PY{l+s+s1}{\PYZsq{}}

\PY{k}{for} \PY{n}{i} \PY{o+ow}{in} \PY{n+nb}{range}\PY{p}{(}\PY{n+nb}{len}\PY{p}{(}\PY{n}{weight}\PY{p}{)}\PY{p}{)}\PY{p}{:}
    \PY{n}{weight\PYZus{}str}\PY{o}{+}\PY{o}{=}\PY{p}{(}\PY{n+nb}{str}\PY{p}{(}\PY{n}{weight\PYZus{}sorted}\PY{p}{[}\PY{n}{i}\PY{p}{]}\PY{p}{)}\PY{o}{+}\PY{l+s+s1}{\PYZsq{}}\PY{l+s+s1}{,}\PY{l+s+s1}{\PYZsq{}}\PY{p}{)}\PY{o}{*}\PY{n}{freq\PYZus{}sorted}\PY{p}{[}\PY{n}{i}\PY{p}{]}

\PY{n}{weight\PYZus{}list}\PY{o}{=}\PY{n}{weight\PYZus{}str}\PY{o}{.}\PY{n}{strip}\PY{p}{(}\PY{l+s+s1}{\PYZsq{}}\PY{l+s+s1}{,}\PY{l+s+s1}{\PYZsq{}}\PY{p}{)}\PY{o}{.}\PY{n}{split}\PY{p}{(}\PY{l+s+s1}{\PYZsq{}}\PY{l+s+s1}{,}\PY{l+s+s1}{\PYZsq{}}\PY{p}{)} 
\PY{n}{weight\PYZus{}list}\PY{o}{=}\PY{n}{np}\PY{o}{.}\PY{n}{array}\PY{p}{(}\PY{n}{weight\PYZus{}list}\PY{p}{,}\PY{n}{dtype}\PY{o}{=}\PY{n+nb}{int}\PY{p}{)}
\PY{n}{weight\PYZus{}list}
\PY{n}{length}\PY{o}{=}\PY{n+nb}{len}\PY{p}{(}\PY{n}{weight\PYZus{}list}\PY{p}{)}

\PY{k}{if} \PY{n}{length}\PY{o}{\PYZpc{}}\PY{k}{2}==1:
    \PY{n}{my\PYZus{}median}\PY{o}{=}\PY{n}{weight\PYZus{}list}\PY{p}{[}\PY{n}{length}\PY{o}{/}\PY{l+m+mi}{2}\PY{p}{]}
\PY{k}{else} \PY{p}{:}
    \PY{n}{pos}\PY{o}{=}\PY{n+nb}{int}\PY{p}{(}\PY{n}{length}\PY{o}{/}\PY{l+m+mi}{2}\PY{p}{)}
    \PY{n}{my\PYZus{}median}\PY{o}{=}\PY{p}{(}\PY{n}{weight\PYZus{}list}\PY{p}{[}\PY{n}{pos}\PY{o}{\PYZhy{}}\PY{l+m+mi}{1}\PY{p}{]}\PY{o}{+}\PY{n}{weight\PYZus{}list}\PY{p}{[}\PY{n}{pos}\PY{p}{]}\PY{p}{)}\PY{o}{/}\PY{l+m+mi}{2}
\PY{n}{my\PYZus{}median}
\end{Verbatim}
\end{tcolorbox}

            \begin{tcolorbox}[breakable, size=fbox, boxrule=.5pt, pad at break*=1mm, opacityfill=0]
\prompt{Out}{outcolor}{179}{\boxspacing}
\begin{Verbatim}[commandchars=\\\{\}]
array([125, 125, 125, 125, 125, 125, 125, 125, 125, 125, 175, 175, 175,
       175, 175, 225, 225, 225, 225, 225, 225, 225, 225, 225, 225, 275,
       275, 275, 275, 275])
\end{Verbatim}
\end{tcolorbox}
        
            \begin{tcolorbox}[breakable, size=fbox, boxrule=.5pt, pad at break*=1mm, opacityfill=0]
\prompt{Out}{outcolor}{179}{\boxspacing}
\begin{Verbatim}[commandchars=\\\{\}]
200.0
\end{Verbatim}
\end{tcolorbox}
        
    \begin{tcolorbox}[breakable, size=fbox, boxrule=1pt, pad at break*=1mm,colback=cellbackground, colframe=cellborder]
\prompt{In}{incolor}{139}{\boxspacing}
\begin{Verbatim}[commandchars=\\\{\}]
\PY{c+c1}{\PYZsh{} var}
\PY{n}{my\PYZus{}var}\PY{o}{=}\PY{l+m+mi}{0}
\PY{k}{for} \PY{n}{i} \PY{o+ow}{in} \PY{n+nb}{range}\PY{p}{(}\PY{n+nb}{len}\PY{p}{(}\PY{n}{weight}\PY{p}{)}\PY{p}{)}\PY{p}{:}
    \PY{n}{my\PYZus{}var}\PY{o}{+}\PY{o}{=}\PY{p}{(}\PY{n}{weight}\PY{p}{[}\PY{n}{i}\PY{p}{]}\PY{o}{\PYZhy{}}\PY{n}{my\PYZus{}mean}\PY{p}{)}\PY{o}{*}\PY{o}{*}\PY{l+m+mi}{2}\PY{o}{*}\PY{n}{freq}\PY{p}{[}\PY{n}{i}\PY{p}{]}
\PY{n}{my\PYZus{}var}\PY{o}{=}\PY{n}{my\PYZus{}var}\PY{o}{/}\PY{p}{(}\PY{n}{freq}\PY{o}{.}\PY{n}{sum}\PY{p}{(}\PY{p}{)}\PY{o}{\PYZhy{}}\PY{l+m+mi}{1}\PY{p}{)}
\PY{n}{my\PYZus{}var}
\end{Verbatim}
\end{tcolorbox}

            \begin{tcolorbox}[breakable, size=fbox, boxrule=.5pt, pad at break*=1mm, opacityfill=0]
\prompt{Out}{outcolor}{139}{\boxspacing}
\begin{Verbatim}[commandchars=\\\{\}]
3160.9195402298856
\end{Verbatim}
\end{tcolorbox}
        
    \begin{tcolorbox}[breakable, size=fbox, boxrule=1pt, pad at break*=1mm,colback=cellbackground, colframe=cellborder]
\prompt{In}{incolor}{140}{\boxspacing}
\begin{Verbatim}[commandchars=\\\{\}]
\PY{c+c1}{\PYZsh{} std}
\PY{n}{my\PYZus{}std}\PY{o}{=}\PY{n}{np}\PY{o}{.}\PY{n}{sqrt}\PY{p}{(}\PY{n}{my\PYZus{}var}\PY{p}{)}
\PY{n}{my\PYZus{}std}
\end{Verbatim}
\end{tcolorbox}

            \begin{tcolorbox}[breakable, size=fbox, boxrule=.5pt, pad at break*=1mm, opacityfill=0]
\prompt{Out}{outcolor}{140}{\boxspacing}
\begin{Verbatim}[commandchars=\\\{\}]
56.22205563860046
\end{Verbatim}
\end{tcolorbox}
        
    \hypertarget{ux6839ux636edemoux7a0bux5e8fux7684mean_-mode_-median_-std_-var_-ux7a0bux5e8fux8ba1ux7b97ux5e76ux4f5cux5bf9ux6bd4}{%
\paragraph{(2) 根据demo程序的mean\_, mode\_, median\_, std\_, var\_
程序计算,并作对比;}\label{ux6839ux636edemoux7a0bux5e8fux7684mean_-mode_-median_-std_-var_-ux7a0bux5e8fux8ba1ux7b97ux5e76ux4f5cux5bf9ux6bd4}}

    \begin{tcolorbox}[breakable, size=fbox, boxrule=1pt, pad at break*=1mm,colback=cellbackground, colframe=cellborder]
\prompt{In}{incolor}{121}{\boxspacing}
\begin{Verbatim}[commandchars=\\\{\}]
\PY{c+c1}{\PYZsh{}\PYZsh{}\PYZsh{} calculating the mean, std, var of grouped data}
\PY{k+kn}{import} \PY{n+nn}{numpy} \PY{k}{as} \PY{n+nn}{np}

\PY{k}{def} \PY{n+nf}{mean\PYZus{}}\PY{p}{(}\PY{n}{val}\PY{p}{,} \PY{n}{freq}\PY{p}{)}\PY{p}{:}
    \PY{k}{return} \PY{n}{np}\PY{o}{.}\PY{n}{average}\PY{p}{(}\PY{n}{val}\PY{p}{,} \PY{n}{weights} \PY{o}{=} \PY{n}{freq}\PY{p}{)}

\PY{k}{def} \PY{n+nf}{median\PYZus{}}\PY{p}{(}\PY{n}{val}\PY{p}{,} \PY{n}{freq}\PY{p}{)}\PY{p}{:}
    \PY{n+nb}{ord} \PY{o}{=} \PY{n}{np}\PY{o}{.}\PY{n}{argsort}\PY{p}{(}\PY{n}{val}\PY{p}{)}
    \PY{n}{cdf} \PY{o}{=} \PY{n}{np}\PY{o}{.}\PY{n}{cumsum}\PY{p}{(}\PY{n}{freq}\PY{p}{[}\PY{n+nb}{ord}\PY{p}{]}\PY{p}{)}
    \PY{k}{return} \PY{n}{val}\PY{p}{[}\PY{n+nb}{ord}\PY{p}{]}\PY{p}{[}\PY{n}{np}\PY{o}{.}\PY{n}{searchsorted}\PY{p}{(}\PY{n}{cdf}\PY{p}{,} \PY{n}{cdf}\PY{p}{[}\PY{o}{\PYZhy{}}\PY{l+m+mi}{1}\PY{p}{]} \PY{o}{/}\PY{o}{/} \PY{l+m+mi}{2}\PY{p}{)}\PY{p}{]}

\PY{k}{def} \PY{n+nf}{mode\PYZus{}}\PY{p}{(}\PY{n}{val}\PY{p}{,} \PY{n}{freq}\PY{p}{)}\PY{p}{:} \PY{c+c1}{\PYZsh{}in the strictest sense, assuming unique mode}
    \PY{k}{return} \PY{n}{val}\PY{p}{[}\PY{n}{np}\PY{o}{.}\PY{n}{argmax}\PY{p}{(}\PY{n}{freq}\PY{p}{)}\PY{p}{]}

\PY{k}{def} \PY{n+nf}{var\PYZus{}}\PY{p}{(}\PY{n}{val}\PY{p}{,} \PY{n}{freq}\PY{p}{)}\PY{p}{:}
    \PY{n}{avg} \PY{o}{=} \PY{n}{mean\PYZus{}}\PY{p}{(}\PY{n}{val}\PY{p}{,} \PY{n}{freq}\PY{p}{)}
    \PY{n}{dev} \PY{o}{=} \PY{n}{freq} \PY{o}{*} \PY{p}{(}\PY{n}{val} \PY{o}{\PYZhy{}} \PY{n}{avg}\PY{p}{)} \PY{o}{*}\PY{o}{*} \PY{l+m+mi}{2}
    \PY{k}{return} \PY{n}{dev}\PY{o}{.}\PY{n}{sum}\PY{p}{(}\PY{p}{)} \PY{o}{/} \PY{p}{(}\PY{n}{freq}\PY{o}{.}\PY{n}{sum}\PY{p}{(}\PY{p}{)} \PY{o}{\PYZhy{}} \PY{l+m+mi}{1}\PY{p}{)}

\PY{k}{def} \PY{n+nf}{std\PYZus{}}\PY{p}{(}\PY{n}{val}\PY{p}{,} \PY{n}{freq}\PY{p}{)}\PY{p}{:}
    \PY{k}{return} \PY{n}{np}\PY{o}{.}\PY{n}{sqrt}\PY{p}{(}\PY{n}{var\PYZus{}}\PY{p}{(}\PY{n}{val}\PY{p}{,} \PY{n}{freq}\PY{p}{)}\PY{p}{)}
\end{Verbatim}
\end{tcolorbox}

    \begin{tcolorbox}[breakable, size=fbox, boxrule=1pt, pad at break*=1mm,colback=cellbackground, colframe=cellborder]
\prompt{In}{incolor}{122}{\boxspacing}
\begin{Verbatim}[commandchars=\\\{\}]
\PY{n}{mean\PYZus{}}\PY{p}{(}\PY{n}{weight}\PY{p}{,}\PY{n}{freq}\PY{p}{)}
\PY{n}{median\PYZus{}}\PY{p}{(}\PY{n}{weight}\PY{p}{,}\PY{n}{freq}\PY{p}{)}
\PY{n}{mode\PYZus{}}\PY{p}{(}\PY{n}{weight}\PY{p}{,}\PY{n}{freq}\PY{p}{)}
\PY{n}{var\PYZus{}}\PY{p}{(}\PY{n}{weight}\PY{p}{,}\PY{n}{freq}\PY{p}{)}
\PY{n}{std\PYZus{}}\PY{p}{(}\PY{n}{weight}\PY{p}{,}\PY{n}{freq}\PY{p}{)}
\end{Verbatim}
\end{tcolorbox}

            \begin{tcolorbox}[breakable, size=fbox, boxrule=.5pt, pad at break*=1mm, opacityfill=0]
\prompt{Out}{outcolor}{122}{\boxspacing}
\begin{Verbatim}[commandchars=\\\{\}]
191.66666666666666
\end{Verbatim}
\end{tcolorbox}
        
            \begin{tcolorbox}[breakable, size=fbox, boxrule=.5pt, pad at break*=1mm, opacityfill=0]
\prompt{Out}{outcolor}{122}{\boxspacing}
\begin{Verbatim}[commandchars=\\\{\}]
175
\end{Verbatim}
\end{tcolorbox}
        
            \begin{tcolorbox}[breakable, size=fbox, boxrule=.5pt, pad at break*=1mm, opacityfill=0]
\prompt{Out}{outcolor}{122}{\boxspacing}
\begin{Verbatim}[commandchars=\\\{\}]
225
\end{Verbatim}
\end{tcolorbox}
        
            \begin{tcolorbox}[breakable, size=fbox, boxrule=.5pt, pad at break*=1mm, opacityfill=0]
\prompt{Out}{outcolor}{122}{\boxspacing}
\begin{Verbatim}[commandchars=\\\{\}]
3160.9195402298856
\end{Verbatim}
\end{tcolorbox}
        
            \begin{tcolorbox}[breakable, size=fbox, boxrule=.5pt, pad at break*=1mm, opacityfill=0]
\prompt{Out}{outcolor}{122}{\boxspacing}
\begin{Verbatim}[commandchars=\\\{\}]
56.22205563860046
\end{Verbatim}
\end{tcolorbox}
        
    两者的mean,mode,var,std都相同,但是median\_取了偶数个数据中中位数的前面一个,
    
    而用公式计算出的求了最中间两个数的平均值,所以答案有差异


    % Add a bibliography block to the postdoc
    
    
    
\end{document}
